\documentclass{report}
\usepackage[utf8]{inputenc}
\usepackage{amsmath}
\usepackage{amsfonts}
\usepackage{hyperref}
\usepackage{tcolorbox}
\usepackage{breqn}
\usepackage{adjustbox}
\usepackage{changepage}
\usepackage{rotating}
\usepackage{algorithm}
\usepackage{algpseudocode}
\usepackage{ntheorem}
\usepackage[table,xcdraw]{xcolor}
\usepackage{longtable}
\usepackage{listings}

% Definition
\newtheorem{definition}{Definition}{\bfseries}{\itshape}
\newtheorem*{definition*}{Definition}{\bfseries}{\itshape}

% Theorem
\newtheorem{theorem}{Theorem}{\bfseries}{\itshape}

% Concept
\newtheorem*{concept}{}{\bfseries}{\itshape}

\title{Obtaining Linear Approximation Tables of AES and DES}
\author{Ana Clara Zoppi Serpa\\ Prof. Dr. Ricardo Dahab \\ Dr. Jorge Nakahara Jr.}
\date{\today}

\begin{document}

\maketitle

\tableofcontents

\chapter{Obtaining Linear Approximation Tables of AES and DES}

Linear Cryptanalysis (LC) was invented by Matsui \cite{Matsui1993LinearCM}. He describes how to apply it to the DES \cite{DES-FIPS} cipher, showing it is capable of breaking 8-round DES with $2^{21}$ known-plaintexts and 16-round DES with $2^{47}$ known-plaintexts. LAT -- Linear Approximation Tables --- of each DES $S$-Box are a core component for the attack. In this chapter, we focus on discussing how to obtain LATs for DES and AES \cite{AES-FIPS} and on exposing preliminary concepts related to LC.

We assume the reader is familiar with:
\begin{itemize}
    \item The DES cipher structure, available in \textcolor{red}{Chapter X}
    \item The AES cipher structure, available in \textcolor{red}{Chapter X}
    \item The Differential Cryptanalysis, DDTs and joint DDTs overview, available in \textcolor{red}{Chapter X}
\end{itemize}

\section{Notation}
\begin{itemize}
    \item $X_i$: a random variable
    \item $p_i$: the probability of $X_i$ being equal to $0$ ($Pr[X_i = 0]$)
    \item $\epsilon_i$: bias of a random variable $X_i$
    \item $\pi_S$: an hypothetical $S$-Box
    \item $n$: input size of an $S$-Box
    \item $m$: output size of an $S$-Box
    \item $X = (x_1, ..., x_n)$: $n$-tuple containing the inputs of $\pi_S$
    \item $Y = (y_1, ..., y_m)$: $m$-tuple containing the outputs of $\pi_s$
    \item $K$: cipher key
    \item $\gamma, \alpha, \beta$: bit masks
    \item $X \cdot \gamma$: dot product of $X$ and $\gamma$, i.e application of the bit mask $\gamma$ to the variable $X$
\end{itemize}

\section{Acronyms}
\begin{itemize}
    \item DES: Data Encryption Standard
    \item AES: Advanced Encryption Standard
    \item LC: Linear Cryptanalysis
    \item DC: Differential Cryptanalysis
    \item DDT: Difference Distribution Table
    \item LAT: Linear Approximation Table
\end{itemize}

\section{Preliminaries}

\subsection{Linear Cryptanalysis overview}

The goal of an LC attack is to find a probabilistic linear relationship between a subset of plaintext bits and a subset of state bits that immediately precede the substitutions performed in the last round of the cipher. In other words, \emph{the attacker seeks a subset of bits whose XOR behaves non-randomly}, i.e, taking value 0 with probability bounded away from $\frac{1}{2}$.

Assuming the attacker has a large number of plaintext and ciphertext pairs, all encrypted with the same key $K$, the attack works by decrypting each ciphertext, using all possible candidate keys for the last round of the iterated cipher. For each of these candidate keys, if there is a probabilistic linear relationship as aforementioned, the candidate key's counter is incremented by 1. At the end of the process, the correct key should have a frequency count which allows us to distinguish it from the other candidates.

Let $X_1, X_2, ...$ be independent random values taking on values from $\{0,1\}$, and $p_1, p_2, ...$ be the probabilities of each of them being equal to zero, i.e, $Pr[X_1 = 0] = p_1, Pr[X_2 = 0] = p_2$ and so forth. 
\begin{concept}[Bias \cite{Matsui1993LinearCM}]
The \emph{bias} of a random variable taking on the values $0$ and $1$ is defined as $\epsilon_i = p_i - \frac{1}{2}$. Consequently, $-1/2 \leq \epsilon_i \leq 1/2$, since $0 \leq p_i \leq 1$.
\end{concept}

\begin{concept}[Piling-Up Lemma \cite{Matsui1993LinearCM}]
The bias of a random variable $X_{i_1} \oplus X_{i_2} \oplus ... \oplus X_{i_k}$ is $\epsilon_{i_1, i_2, ..., i_k} = 2^{k-1} \prod_{j=1}^{k} \epsilon_{i_j}$.
\end{concept}

The Linear Cryptanalysis attack exploits sums of random variables, e.g $S$-boxes inputs and outputs, their biases, and, extending the concept to the whole iterated cipher, extracts the key. 

Considering an example $S$-box $\pi_S : \{0,1\}^n \rightarrow \{0,1\}^m$, if an input $n$-tuple $X = (x_1, ..., x_n)$ is chosen randomly, each coordinate $x_i$ is a random variable $X_i$ taking values from $\{0,1 \}$. These variables are independent. A randomly chosen output tuple $Y = (y_1, ..., y_m)$, similarly, defines a random variable $Y_j$, also taking values from $\{0,1 \}$. However, they are not necessarily independent on each other, nor they are independent of the $X_i$ variables.

%If $(y_1, ..., y_m) \neq \pi_s(x_1, ..., x_n)$ then 

%$$Pr[X_1 = x_1, ..., X_n = x_n, Y_1 = y_1, ..., Y_m = y_m] = 0.$$ 

%However, if $(y_1, ..., y_m) = \pi_s(x_1, ..., x_n)$, then 

%$$Pr[X_1 = x_1, ..., X_n = x_n, Y_1 = y_1, ..., Y_m = y_m] = 2^{-n},$$ 

%since $$Pr[X_1 = x_1, ..., X_n = x_n] = 2^{-n}$$

%and $$Pr[Y_1 = y_1, ..., Y_m = y_m | X_1 = x_1, ..., X_n = x_n] = 1.$$

It is possible to compute the bias of a random variable of the form $X_{i_1} \oplus ... \oplus X_{i_z} \oplus Y_{j_1} \oplus ... \oplus Y_{j_l}$, from the Piling-up Lemma. When such bias is bounded away from $\frac{1}{2}$, a Linear Cryptanalysis attack can be mounted, if the amount of required plaintexts for the attack (which is inversely proportional to the bias) does not surpass the cipher's codebook, i.e if the final bias value is not too small.

\subsection{Bit masks and Linear Approximation Tables}

\begin{concept}[Bit mask \cite{Matsui1993LinearCM}]
A bit mask $\gamma$ selects fixed specific bits of a variable $X$. For example, $\gamma = [i]$ refers to selecting only the $i$-th bit, $\gamma = [i, j, k]$ refers to selecting the $i$-th, $j$-th and $k$-th bits, and so forth, and XORing them. Formally, the result of the mask selection for a variable $X$ can be obtained by means of the \emph{dot product} $X \cdot \gamma$.
\end{concept}

For example, let $X = 1111_2$ and $\gamma = [1, 2] = 0110_2$. Then

$$X \cdot \gamma = (0\cdot1) \oplus (1\cdot1) \oplus (1\cdot1) \oplus (0\cdot1) = 0_2,$$

because $\gamma$ selects bits $1$ and $2$, considering that the starting bit, $0$, is the least significant bit, and XORs them. Bits 0 and 3 are not selected. Bit masks are used by Matsui \cite{Matsui1993LinearCM} to represent linear relationships (or approximations) between subsets of bits of the cipher. As another example, we show the usage of bit masks to represent linear relationship in the logical XOR of two variables.

Let $A = (..., a_1, a_0)$, $B = (..., b_1, b_0)$ and $C = A \oplus B$. Each $i$-th bit of $C$ depends linearly (and individually) on the $i$-th bits of $A$ and $B$, due to the bitwise XOR. In other words, $c_i = a_i \oplus b_i$. Equivalently, using bit mask notation, $C \cdot \gamma = A \cdot \gamma \oplus B \cdot \gamma$, with $\gamma = [i]$. Note that, when dealing with bit masks, we have switched the indexing of bits: we start from left to right, instead of from right to left. Also, the bit indices start at 0. We do this to follow the notation presented by Matsui \cite{Matsui1993LinearCM}.

\begin{concept}[Linear Approximation Table of an $S$-Box]
For an $n$-bit input $S$-Box $S$, the LAT of $S$ is the table $\mathcal{L}$ such that  $\mathcal{L}[\alpha][\beta] = \# \{ x \in \mathbb{Z}_2^n : \alpha \cdot x = \beta \cdot S(x)\} - 2^{n-1}$, where $\alpha$ and $\beta$ are masks. In other words, a LAT entry $[\alpha, \beta]$ counts how far the parity of $\alpha \cdot x \oplus \beta \cdot S(x)$ deviates (or not) from $2^{n-1}$.
\end{concept}

\begin{concept}[Trivial and non-trivial linear relations]
A trivial linear relation possesses bias equal to zero. A non-trivial linear relation possesses non-zero bias. In Linear Cryptanalysis, we seek non-trivial linear relations.
\end{concept}

\begin{concept}[Trivial and non-trivial masks]
We say that $\alpha = 0$ is a trivial mask, whilst $\alpha \neq 0$ is a non-trivial mask. In LC, we are interested in LAT entries with non-trivial input masks.
\end{concept}

Each entry $\mathcal{L}[\alpha][\beta]$ indicates whether there is a non-trivial linear relation between $\alpha$ input bits and $\beta$ output bits of $S$.

\begin{concept}[Linear Uniformity]
The linear uniformity of an $S$-Box is 

$$\gamma_{max} = max_{\alpha\not= 0} |LAT(\alpha,\beta)|,$$ 
where $|x|$ denotes the absolute value of $x$. It shows the largest entry in the LAT of a given $S$-Box.
\end{concept}

\section{Computing a LAT}

Let $S$ be an $n$-bit input and $m$-bit output $S$-Box. In order to obtain its LAT $\mathcal{L}$, Algorithm \ref{alg:get-lat} can be used. $2^{n} \times 2^{m} \times 2^{n}$ $S$-Box lookups are necessary, and $2 \times 2^{n} \times 2^{m} \times 2^{n}$ mask selection steps. Therefore, the computational complexity is $O(2^{2n + m})$ $S$-Box lookups, considering $S$-Box lookups and mask selection operations to be constant time.

For a DES $S$-Box, $n = 6$ and $m = 4$, resulting in $2^{16}$ run time complexity. For an AES $S$-Box, $n = m = 8$, resulting in $2^{24}$ complexity. In the case of AES and DES, since we are dealing with integer values, the mask selection can be performed with logical AND, XOR and bit shift operators.

\begin{algorithm}[H]
\caption{Obtaining the LAT of an $S$-box}
\label{alg:get-lat}
\begin{algorithmic}[1]
    \State Initialize $\mathcal{L}$ with $2^{n-1}$ (the bias) in all entries
    \For{$\alpha = 0, ..., 2^n - 1$}
        \For{$\beta = 0, ..., 2^m - 1$}
            \For {$x = 0, ..., 2^n - 1$}
                \State $l \gets \alpha \cdot x$ \Comment{Select bits according to the input mask}
                \State $r \gets \beta \cdot S(x)$ \Comment{Select bits according to the output mask}
                \If {$l = r$}
                    \State $\mathcal{L}[\alpha][\beta] \gets \mathcal{L}[\alpha][\beta] + 1$
                \EndIf
            \EndFor
        \EndFor
    \EndFor
\State \textbf{return} $\mathcal{L}$
\end{algorithmic}
\end{algorithm}

It is also possible to compute LATs using the Walsh-Hadamard Transform of a Boolean Function, as explained in \cite{Anne2016}.

\subsection{Practical run time remarks}
In a personal computer with a 2.6 GHz Intel Core i7 6-Core processor, a LAT for a DES $S$-Box could be obtained in less than 2 seconds, and a LAT for one AES $S$-Box could be obtained in less than 2 minutes.

\section{LATs for DES}

Table \ref{tbl:lu-des} shows the linear uniformitiy of each DES $S$-Box. Note that $S_5$ presents the largest linear uniformity, which signals it contains the most effective linear relationship to be exploited in an LC attack. For this reason, Matsui \cite{Matsui1993LinearCM} exploits $S_5$ to mount his attack.
Tables \ref{tbl:lat1} to \ref{tbl:lat8} show the LATs for $S$-Boxes $S_1$ to $S_8$, respectively. They are also available as \texttt{.csv} files at \textcolor{red}{Git link}, together with the code used to obtain them.

\begin{table}[H]
\centering
\begin{tabular}{|l|l|l|l|l|l|l|l|l|}
\hline
$S$-Box           & $S_1$ & $S_2$ & $S_3$ & $S_4$ & $S_5$ & $S_6$ & $S_7$ & $S_8$ \\ \hline
Linear uniformity & 18    & 16    & 16    & 16    & 20    & 14    & 18    & 16    \\ \hline
\end{tabular}
\caption{Linear Uniformities of each DES $S$-Box}
\label{tbl:lu-des}
\end{table}

% Please add the following required packages to your document preamble:
% \usepackage{longtable}
% Note: It may be necessary to compile the document several times to get a multi-page table to line up properly
\begin{longtable}[c]{|l|l|l|l|l|l|l|l|l|l|l|l|l|l|l|l|l|}
\hline
            & \textbf{0} & \textbf{1} & \textbf{2} & \textbf{3} & \textbf{4} & \textbf{5} & \textbf{6} & \textbf{7} & \textbf{8} & \textbf{9} & \textbf{10} & \textbf{11} & \textbf{12} & \textbf{13} & \textbf{14} & \textbf{15}  \\ \hline
\endfirsthead
%
\endhead
%
\textbf{0}  & 32         & 0          & 0          & 0          & 0          & 0          & 0          & 0          & 0          & 0          & 0           & 0           & 0           & 0           & 0           & 0            \\ \hline
\textbf{1}  & 0          & 0          & 0          & 0          & 0          & 0          & 0          & 0          & 0          & 0          & 0           & 0           & 0           & 0           & 0           & 0            \\ \hline
\textbf{2}  & 0          & -2         & -2         & -4         & -2         & 0          & -4         & 6          & 2          & 0          & 0           & 6           & 4           & -2          & -6          & 4            \\ \hline
\textbf{3}  & 0          & -2         & -2         & -4         & -2         & 0          & -4         & 6          & 2          & 8          & 0           & -2          & 4           & 6           & -6          & -4           \\ \hline
\textbf{4}  & 0          & 2          & -2         & -4         & -2         & 0          & -4         & -6         & -2         & 4          & 8           & 2           & 0           & -2          & -6          & 12           \\ \hline
\textbf{5}  & 0          & -2         & -2         & 0          & -2         & -4         & -4         & -2         & 2          & -4         & -4          & 2           & 4           & -10         & -2          & -4           \\ \hline
\textbf{6}  & 0          & 0          & 0          & 4          & 0          & 4          & 0          & 0          & 0          & -4         & 4           & 4           & 0           & 0           & -4          & -8           \\ \hline
\textbf{7}  & 0          & -4         & 0          & 8          & 0          & 0          & 0          & 4          & 4          & -4         & -8          & -4          & 4           & 0           & 0           & 0            \\ \hline
\textbf{8}  & 0          & 4          & -2         & 6          & -6         & -6         & 0          & -4         & -4         & -4         & 2           & -2          & 2           & -2          & 0           & 0            \\ \hline
\textbf{9}  & 0          & 0          & 6          & -6         & -2         & -6         & 4          & -4         & 0          & -4         & -2          & 6           & 2           & -6          & 0           & -4           \\ \hline
\textbf{10} & 0          & -2         & 0          & 2          & 0          & 6          & 8          & 2          & -2         & 0          & -2          & 4           & -2          & 0           & -2          & 4            \\ \hline
\textbf{11} & 0          & 2          & -8         & -2         & -4         & -10        & 4          & 2          & -6         & 8          & 2           & 4           & -2          & -4          & -2          & 0            \\ \hline
\textbf{12} & 0          & -2         & 0          & 6          & 0          & 2          & 0          & 2          & 2          & 0          & 6           & -4          & 2           & -4          & 6           & 0            \\ \hline
\textbf{13} & 0          & 6          & 0          & 6          & 4          & -2         & -4         & -2         & 2          & 0          & 6           & 4           & -2          & 8           & -6          & -4           \\ \hline
\textbf{14} & 0          & 0          & -2         & -2         & 2          & 2          & 0          & 0          & 4          & 4          & 6           & -2          & 2           & 2           & -4          & 4            \\ \hline
\textbf{15} & 0          & 0          & -2         & 6          & -2         & -2         & 4          & -4         & -4         & -4         & -2          & -2          & -2          & -2          & 0           & 0            \\ \hline
\textbf{16} & 0          & 2          & 2          & 0          & -2         & 0          & 4          & -6         & 0          & 6          & 2           & -4          & 6           & -4          & -4          & \textbf{-18} \\ \hline
\textbf{17} & 0          & 2          & -2         & -4         & 2          & -4         & -4         & 10         & -4         & 2          & 2           & -4          & -2          & -4          & 0           & -6           \\ \hline
\textbf{18} & 0          & 4          & 0          & 0          & -4         & 4          & 0          & 4          & -6         & 2          & 2           & 6           & 2           & 6           & 6           & -10          \\ \hline
\textbf{19} & 0          & 4          & -4         & -4         & 0          & 0          & -8         & -12        & -2         & -2         & -6          & 6           & 2           & 6           & 2           & 2            \\ \hline
\textbf{20} & 0          & 4          & 0          & 4          & -8         & -4         & 4          & 0          & 2          & 6          & -2          & 2           & 6           & 2           & -2          & 2            \\ \hline
\textbf{21} & 0          & 0          & 4          & -4         & -4         & 4          & 4          & -4         & 10         & 2          & 2           & 2           & -6          & 2           & 6           & -2           \\ \hline
\textbf{22} & 0          & 6          & 2          & 0          & 2          & -4         & 0          & 2          & 4          & 2          & 2           & 0           & -2          & 0           & 0           & 2            \\ \hline
\textbf{23} & 0          & 2          & 6          & -8         & 6          & 4          & 0          & -2         & -12        & -2         & -2          & 0           & -6          & 0           & 0           & -2           \\ \hline
\textbf{24} & 0          & 2          & 8          & 2          & 0          & 6          & 4          & 2          & 4          & -2         & 4           & 6           & 0           & -2          & -4          & 2            \\ \hline
\textbf{25} & 0          & -2         & 4          & -6         & 0          & -6         & 0          & 2          & 4          & -6         & 8           & 6           & 0           & 2           & 0           & -6           \\ \hline
\textbf{26} & 0          & 0          & -6         & 2          & -2         & -2         & 4          & 4          & -2         & -2         & 0           & 0           & -4          & 4           & 2           & 2            \\ \hline
\textbf{27} & 0          & 4          & 6          & 2          & -10        & 2          & -8         & 4          & -2         & -6         & 4           & 0           & 4           & 0           & -2          & 2            \\ \hline
\textbf{28} & 0          & -4         & 2          & 2          & 2          & -6         & 0          & -4         & -2         & -2         & 4           & 0           & 0           & 4           & 2           & 2            \\ \hline
\textbf{29} & 0          & 4          & -2         & -2         & 2          & -6         & -4         & 0          & 2          & 2          & -4          & 0           & -12         & 0           & -6          & -6           \\ \hline
\textbf{30} & 0          & 2          & 0          & -2         & 4          & -2         & 0          & -2         & 0          & 6          & -4          & -2          & 0           & -2          & 0           & 2            \\ \hline
\textbf{31} & 0          & 2          & -4         & 2          & -4         & -2         & 4          & 2          & 4          & -6         & 4           & -2          & -4          & 2           & 0           & 2            \\ \hline
\textbf{32} & 0          & 0          & 0          & 0          & 0          & 0          & 0          & 0          & 0          & 0          & 0           & 0           & 0           & 0           & 0           & 0            \\ \hline
\textbf{33} & 0          & 0          & 0          & 0          & 0          & 0          & 0          & 0          & 0          & 0          & 0           & 0           & 0           & 0           & 0           & 0            \\ \hline
\textbf{34} & 0          & 2          & -2         & 0          & 2          & 0          & 0          & 6          & -2         & 0          & -4          & 6           & 4           & 10          & 10          & 0            \\ \hline
\textbf{35} & 0          & 2          & -2         & 0          & 2          & 0          & 0          & 6          & 6          & 0          & 4           & -10         & -4          & -6          & 2           & 0            \\ \hline
\textbf{36} & 0          & 2          & -6         & -8         & 2          & 4          & 4          & 2          & 2          & 0          & 0           & 2           & 0           & 6           & -10         & 0            \\ \hline
\textbf{37} & 0          & -2         & 2          & 4          & 2          & 0          & -4         & -2         & -2         & 0          & 4           & 2           & -4          & 6           & -6          & 0            \\ \hline
\textbf{38} & 0          & 4          & 4          & -4         & 8          & -8         & 4          & 0          & 0          & -8         & 0           & -4          & 0           & 4           & 0           & 0            \\ \hline
\textbf{39} & 0          & 0          & -4         & -8         & -8         & 4          & -4         & -4         & 4          & -8         & -4          & -4          & 4           & 4           & -4          & 0            \\ \hline
\textbf{40} & 0          & 4          & -2         & -2         & -2         & -2         & -4         & 0          & 4          & 4          & 2           & 6           & -2          & -6          & 12          & 4            \\ \hline
\textbf{41} & 0          & 0          & -2         & -6         & 2          & -2         & -8         & 8          & 0          & -4         & -2          & -2          & 6           & -2          & -4          & 0            \\ \hline
\textbf{42} & 0          & 2          & 0          & -2         & 0          & 2          & 0          & -2         & 2          & -8         & -6          & -4          & 2           & 0           & 2           & -4           \\ \hline
\textbf{43} & 0          & -10        & 0          & 2          & -4         & 2          & 4          & 6          & -2         & 0          & -10         & 4           & 2           & -4          & -6          & 0            \\ \hline
\textbf{44} & 0          & 6          & -4         & 2          & 8          & 2          & 4          & 6          & -2         & -4         & -2          & 12          & -2          & -8          & -2          & 0            \\ \hline
\textbf{45} & 0          & -2         & -4         & 2          & -4         & -2         & 0          & 2          & -2         & -4         & -2          & 4           & -6          & 4           & 2           & -4           \\ \hline
\textbf{46} & 0          & -4         & 2          & 6          & 6          & -6         & -8         & 4          & -4         & 0          & 2           & 6           & 6           & 2           & 4           & 0            \\ \hline
\textbf{47} & 0          & -4         & 2          & -2         & 2          & -10        & 12         & 0          & -4         & 0          & 2           & -2          & 10          & 6           & 0           & 4            \\ \hline
\textbf{48} & 0          & -2         & -2         & 0          & -2         & 4          & 0          & 2          & 0          & 2          & 6           & 4           & 6           & 0           & 0           & -2           \\ \hline
\textbf{49} & 0          & -2         & 2          & 4          & 2          & 0          & 0          & -6         & -4         & -2         & -2          & -4          & -2          & 0           & -4          & 2            \\ \hline
\textbf{50} & 0          & -4         & -4         & -4         & 0          & 0          & 0          & 4          & -2         & -2         & -6          & -2          & -6          & 6           & 2           & 2            \\ \hline
\textbf{51} & 0          & -4         & 0          & 0          & 4          & -4         & 0          & -4         & -6         & 2          & 2           & -2          & 2           & -2          & -2          & -2           \\ \hline
\textbf{52} & 0          & 8          & -8         & 8          & 4          & 4          & 0          & 0          & -2         & -2         & 2           & -6          & 6           & 6           & -2          & -2           \\ \hline
\textbf{53} & 0          & 4          & -4         & 0          & -8         & -4         & 0          & -4         & -2         & 2          & -2          & 2           & 2           & -2          & -2          & 2            \\ \hline
\textbf{54} & 0          & 6          & 2          & -8         & 2          & -4         & 8          & 2          & 4          & -6         & 2           & 0           & 6           & 0           & 0           & 2            \\ \hline
\textbf{55} & 0          & 2          & -10        & 0          & 6          & 4          & 8          & -2         & 4          & 6          & -2          & 0           & 2           & 0           & 0           & -2           \\ \hline
\textbf{56} & 0          & -10        & 4          & 2          & -4         & -2         & 4          & -2         & 4          & 2          & 0           & 6           & -4          & 6           & -4          & -2           \\ \hline
\textbf{57} & 0          & 2          & 0          & -6         & -4         & 2          & 0          & -2         & -4         & 6          & -4          & -2          & 4           & 2           & 8           & -2           \\ \hline
\textbf{58} & 0          & 0          & 6          & -2         & 6          & 6          & 0          & 0          & 2          & 2          & 0           & 0           & 8           & 0           & 2           & 2            \\ \hline
\textbf{59} & 0          & 4          & 2          & -2         & -2         & 10         & 4          & 0          & -14        & -2         & 4           & 0           & 0           & -4          & -2          & 2            \\ \hline
\textbf{60} & 0          & 0          & 10         & -2         & -6         & -2         & 0          & 8          & -6         & 6          & 0           & -8          & -4          & 4           & -2          & 2            \\ \hline
\textbf{61} & 0          & 8          & -2         & 2          & -6         & -2         & 4          & 4          & -2         & -6         & 0           & 0           & 0           & 0           & -2          & 2            \\ \hline
\textbf{62} & 0          & 2          & 0          & -2         & -8         & 2          & 4          & 2          & 0          & -2         & 4           & -2          & -4          & 2           & 4           & -2           \\ \hline
\textbf{63} & 0          & -14        & -12        & -6         & 0          & 2          & 0          & -2         & -4         & -6         & 12          & -2          & 0           & -2          & 4           & -2           \\ \hline
\caption{LAT for $S_1$ of DES}
\label{tbl:lat1}
\end{longtable}
% Please add the following required packages to your document preamble:
% \usepackage{longtable}
% Note: It may be necessary to compile the document several times to get a multi-page table to line up properly
\begin{longtable}[c]{|l|l|l|l|l|l|l|l|l|l|l|l|l|l|l|l|l|}
\hline
            & \textbf{0} & \textbf{1} & \textbf{2} & \textbf{3} & \textbf{4} & \textbf{5} & \textbf{6} & \textbf{7} & \textbf{8} & \textbf{9} & \textbf{10} & \textbf{11}  & \textbf{12} & \textbf{13} & \textbf{14} & \textbf{15}  \\ \hline
\endfirsthead
%
\endhead
%
\textbf{0}  & 32         & 0          & 0          & 0          & 0          & 0          & 0          & 0          & 0          & 0          & 0           & 0            & 0           & 0           & 0           & 0            \\ \hline
\textbf{1}  & 0          & 0          & 0          & 0          & 0          & 0          & 0          & 0          & 0          & 0          & 0           & 0            & 0           & 0           & 0           & 0            \\ \hline
\textbf{2}  & 0          & 0          & 0          & 4          & 0          & 0          & -4         & 0          & 0          & 0          & -4          & 0            & 0           & 0           & 0           & 4            \\ \hline
\textbf{3}  & 0          & 0          & 4          & -8         & 0          & 8          & 0          & -4         & 0          & 0          & -8          & -4           & 0           & 8           & -4          & 8            \\ \hline
\textbf{4}  & 0          & -2         & 2          & 4          & 2          & 0          & 4          & 6          & 0          & 6          & -2          & 0            & 2           & 0           & 0           & 10           \\ \hline
\textbf{5}  & 0          & 2          & 2          & 0          & 2          & -4         & 4          & -6         & 0          & -6         & 6           & 4            & -6          & 4           & 0           & -2           \\ \hline
\textbf{6}  & 0          & -2         & -2         & -4         & -2         & -4         & 0          & -2         & -4         & 2          & 2           & 0            & 2           & 0           & 4           & 10           \\ \hline
\textbf{7}  & 0          & 2          & 2          & -4         & -2         & 0          & 4          & -2         & -4         & 6          & 6           & 0            & -6          & -4          & 0           & 2            \\ \hline
\textbf{8}  & 0          & 0          & 2          & 2          & -2         & 2          & 4          & 0          & -2         & -6         & -4          & 0            & 0           & 0           & 10          & -6           \\ \hline
\textbf{9}  & 0          & 0          & -2         & -2         & -2         & 2          & 8          & 4          & -2         & 2          & 0           & -4           & 0           & 8           & -10         & -2           \\ \hline
\textbf{10} & 0          & -4         & 2          & 2          & 2          & 2          & -4         & -8         & 2          & 2          & 4           & 0            & 0           & 4           & -6          & 2            \\ \hline
\textbf{11} & 0          & 4          & 2          & 10         & 2          & 2          & 4          & 0          & 2          & 2          & 4           & 0            & 0           & -4          & 2           & 2            \\ \hline
\textbf{12} & 0          & -2         & 4          & -2         & 0          & 2          & 0          & 6          & -2         & 0          & 2           & 0            & 2           & 0           & -6          & -4           \\ \hline
\textbf{13} & 0          & -6         & 0          & -2         & 0          & 6          & 4          & -10        & 6          & -4         & 6           & 0            & 2           & -4          & -2          & 4            \\ \hline
\textbf{14} & 0          & -6         & 0          & 2          & 0          & -2         & -8         & 6          & -2         & 4          & 2           & 0            & 2           & 4           & -2          & 0            \\ \hline
\textbf{15} & 0          & -2         & 0          & -2         & 0          & 2          & 0          & 10         & 6          & 8          & 2           & 4            & 2           & 0           & -2          & 4            \\ \hline
\textbf{16} & 0          & 0          & -4         & 4          & 0          & 0          & 0          & 0          & 0          & 0          & 4           & 4            & 4           & -12         & -4          & -12 \\ \hline
\textbf{17} & 0          & 0          & 0          & 0          & 0          & -8         & 4          & 4          & 0          & 0          & 8           & 0            & -4          & 4           & 8           & 0            \\ \hline
\textbf{18} & 0          & 0          & 0          & 4          & 0          & 8          & 0          & 4          & 0          & 0          & -4          & 8            & -4          & -12         & 0           & 12           \\ \hline
\textbf{19} & 0          & 0          & -8         & 4          & 0          & 8          & 8          & 4          & 0          & 0          & -4          & 0            & 4           & -4          & 8           & -4           \\ \hline
\textbf{20} & 0          & -2         & -6         & -4         & -2         & 4          & -4         & -2         & -4         & 2          & 2           & -4           & 6           & -4          & 0           & 2            \\ \hline
\textbf{21} & 0          & 10         & -2         & -4         & -2         & 0          & 0          & -2         & 4          & 6          & 6           & -4           & -2          & 0           & 4           & 2            \\ \hline
\textbf{22} & 0          & -2         & -6         & 0          & 2          & 0          & 4          & 2          & 8          & -2         & 2           & 0            & 6           & 4           & 0           & -2           \\ \hline
\textbf{23} & 0          & 10         & 2          & 4          & 2          & 4          & -4         & -2         & 0          & 2          & 2           & -4           & -2          & 0           & 0           & 2            \\ \hline
\textbf{24} & 0          & -4         & 2          & -2         & 2          & 2          & 4          & 4          & -2         & -2         & 4           & 4            & 0           & 4           & -2          & 2            \\ \hline
\textbf{25} & 0          & 4          & -6         & 6          & 2          & 2          & 4          & -4         & -2         & -2         & 4           & -4           & 8           & 4           & -2          & 2            \\ \hline
\textbf{26} & 0          & 8          & -2         & 2          & -2         & 2          & 0          & 0          & 2          & 6          & 0           & 0            & 0           & 0           & 2           & -2           \\ \hline
\textbf{27} & 0          & -8         & 10         & 6          & -2         & 2          & 4          & -4         & 2          & -2         & -4          & -4           & -8          & 0           & -2          & -6           \\ \hline
\textbf{28} & 0          & 2          & 8          & -2         & 0          & -2         & 0          & 2          & 2          & 0          & -6          & 4            & 10          & -4          & 2           & 0            \\ \hline
\textbf{29} & 0          & -2         & 0          & 2          & 0          & -6         & 0          & -2         & 2          & 4          & 2           & 0            & 10          & 0           & 2           & 4            \\ \hline
\textbf{30} & 0          & -2         & 0          & 6          & 0          & 2          & 4          & -2         & 2          & 4          & -2          & 0            & 2           & 0           & 2           & 0            \\ \hline
\textbf{31} & 0          & 2          & -4         & 6          & 0          & -2         & -8         & -2         & -14        & 0          & 2           & 0            & 2           & 4           & -2          & 0            \\ \hline
\textbf{32} & 0          & 0          & 0          & 0          & 0          & 0          & 0          & 0          & 0          & 0          & 0           & 0            & 0           & 0           & 0           & 0            \\ \hline
\textbf{33} & 0          & 0          & 0          & 0          & 0          & 0          & 0          & 0          & 0          & 0          & 0           & 0            & 0           & 0           & 0           & 0            \\ \hline
\textbf{34} & 0          & 0          & 0          & 4          & 0          & -8         & 4          & 0          & 0          & 0          & -4          & \textbf{-16} & 0           & -8          & -8          & 4            \\ \hline
\textbf{35} & 0          & 0          & -4         & 0          & 0          & 0          & 0          & 4          & 0          & 0          & 0           & 4            & 0           & 0           & -4          & 0            \\ \hline
\textbf{36} & 0          & -2         & -2         & 0          & -2         & -4         & 4          & -10        & 0          & 6          & -6          & -4           & 6           & 4           & 8           & 2            \\ \hline
\textbf{37} & 0          & -6         & -2         & 4          & -2         & 0          & 4          & 2          & 0          & 2          & 2           & -8           & -2          & 0           & 8           & -2           \\ \hline
\textbf{38} & 0          & 6          & -6         & 0          & -14        & 0          & 0          & -2         & -4         & -6         & -2          & 4            & -2          & -4          & -4          & 2            \\ \hline
\textbf{39} & 0          & 2          & 6          & 0          & 2          & -4         & -4         & -2         & -4         & 6          & -6          & 4            & 6           & 0           & 0           & -6           \\ \hline
\textbf{40} & 0          & -4         & 2          & -2         & 2          & 2          & 0          & -8         & 2          & 2          & 0           & 8            & 8           & -4          & -6          & -2           \\ \hline
\textbf{41} & 0          & -4         & -2         & -6         & 2          & 2          & 4          & -4         & -6         & 2          & -4          & -4           & 0           & -4          & -2          & -6           \\ \hline
\textbf{42} & 0          & 8          & 2          & -2         & 6          & -6         & 0          & 0          & -2         & 2          & 0           & 0            & 0           & 0           & -6          & -2           \\ \hline
\textbf{43} & 0          & 0          & -6         & -2         & 6          & 10         & 0          & 0          & -10        & 10         & 0           & 0            & -8          & 0           & 2           & -2           \\ \hline
\textbf{44} & 0          & -6         & 0          & 6          & 0          & -2         & -4         & -2         & 2          & -8         & 2           & 4            & -2          & 0           & 2           & 8            \\ \hline
\textbf{45} & 0          & -2         & -4         & -2         & 0          & -6         & 0          & -10        & 2          & 4          & -2          & 4            & -10         & -4          & -2          & 0            \\ \hline
\textbf{46} & 0          & -2         & -4         & 2          & 8          & 2          & 4          & -2         & -6         & -4         & -6          & 4            & -2          & 4           & -2          & 4            \\ \hline
\textbf{47} & 0          & -6         & 4          & -2         & -8         & -2         & 4          & 2          & -6         & 0          & 10          & 0            & 6           & 0           & -2          & 0            \\ \hline
\textbf{48} & 0          & 0          & -4         & 4          & -4         & -4         & 4          & 4          & 0          & 0          & -4          & -4           & 0           & 0           & -8          & 0            \\ \hline
\textbf{49} & 0          & 0          & 0          & 0          & 4          & -4         & 0          & 0          & 0          & 0          & 0           & -8           & 0           & -8          & -4          & 4            \\ \hline
\textbf{50} & 0          & 0          & 0          & 4          & 4          & 4          & 4          & 0          & 0          & 0          & 4           & 0            & 0           & 0           & -4          & 0            \\ \hline
\textbf{51} & 0          & 0          & 0          & -4         & -4         & -4         & -4         & 0          & 0          & 0          & -4          & 0            & 0           & 0           & 4           & 0            \\ \hline
\textbf{52} & 0          & -10        & -2         & -8         & 6          & 4          & -8         & 2          & 4          & 2          & 6           & -8           & -2          & -4          & 4           & -2           \\ \hline
\textbf{53} & 0          & -6         & 2          & 0          & -2         & 0          & 4          & 2          & -4         & -2         & -6          & 0            & -2          & 0           & 0           & -2           \\ \hline
\textbf{54} & 0          & -2         & -2         & 4          & -6         & 0          & 8          & -2         & 0          & 6          & 6           & 4            & -2          & 4           & -4          & 2            \\ \hline
\textbf{55} & 0          & 2          & -2         & -8         & -14        & 4          & 0          & 2          & 8          & 2          & -2          & 0            & -2          & 0           & -4          & -2           \\ \hline
\textbf{56} & 0          & 0          & 2          & 2          & -6         & -2         & -4         & 0          & 2          & -2         & 0           & -4           & -4          & -4          & 2           & 2            \\ \hline
\textbf{57} & 0          & 8          & 10         & -6         & 2          & 6          & 4          & 0          & -6         & -10        & 8           & -4           & 4           & -4          & 2           & 2            \\ \hline
\textbf{58} & 0          & -4         & -2         & 6          & -2         & -2         & -8         & 4          & -2         & -2         & 4           & 0            & -4          & 0           & -2          & -2           \\ \hline
\textbf{59} & 0          & -4         & 2          & 2          & -10        & 6          & -4         & 0          & -10        & -2         & 0           & -4           & 4           & 0           & 2           & 2            \\ \hline
\textbf{60} & 0          & -2         & -4         & -2         & 4          & -2         & 0          & -2         & -2         & 0          & 2           & 0            & 2           & -8          & -2          & 0            \\ \hline
\textbf{61} & 0          & 2          & 4          & 10         & -4         & 10         & -8         & -6         & 6          & 4          & 2           & -4           & 2           & 4           & -2          & -4           \\ \hline
\textbf{62} & 0          & 2          & -12        & -2         & 4          & 2          & -4         & 2          & 6          & -12        & -2          & -4           & 2           & 4           & -2          & 0            \\ \hline
\textbf{63} & 0          & -2         & -8         & -2         & -4         & -2         & 0          & -6         & -2         & 0          & 2           & 4            & 2           & 0           & 2           & 0            \\ \hline
\caption{LAT for $S_2$ of DES}
\label{tbl:lat2}
\end{longtable}
\input{lat3}
\input{lat4}
\input{lat5}
\input{lat6}
\input{lat7}
\input{lat8}

\section{LATs for AES}

For the AES $S$-Box (and its inverse), the linear uniformity is $16$. The LATs for AES are too large to be displayed in this document with good visualization quality, therefore, we make them available as \texttt{.csv} files at \textcolor{red}{Git link}, together with the code used to obtain them. Figures \ref{fig:lat-aes} and \ref{fig:lat-aes-inv} show one possible visual representation (heatmap) of the AES LATs.

\begin{figure}
    \centering
    \includegraphics[scale=0.45]{lat_aes.png}
    \caption{LAT heatmap for AES $S$-Box}
    \label{fig:lat-aes}
\end{figure}

\begin{figure}
    \centering
    \includegraphics[scale=0.45]{lat_aes_inv.png}
    \caption{LAT heatmap for AES inverse $S$-Box}
    \label{fig:lat-aes-inv}
\end{figure}

\section{LC versus DC}
Linear Cryptanalysis is more efficient in breaking DES than Differential Cryptanalysis \cite{Shamir}, since it is capable of breaking the full 16-round DES in less steps. Furthermore, note that, in LC, we do not need \emph{joint LATs} or other means of restricting $S$-Box activation, because the usage of bit masks allows the attacker a better control of the propagation of the linear approximations. 

This suggests, as mentioned in \cite{Susan}, that the DES designers took DC in account, but not LC. Also, Matsui \cite{Matsui1993LinearCM} is able to mount an attack taking advantage of the linear uniformity, while Biham and Shamir \cite{Shamir} cannot take advantage of the differential uniformity of the DES $S$-Boxes and must find a suitable characteristic for their attack by other means, due to the difficulties incurred by the DES design against DC. See \cite{Coppersmith1994} and \textcolor{red}{Chapter X} of this work for more about DC on DES. 

Table \ref{tbl:dc-versus-lc} briefly compares the core concepts of DC and LC. Figures \ref{fig:diftrail} and \ref{fig:lintrail} show how differences (respectively linear relations) propagate through the DES $F$ function and the $S$-Boxes.

% Please add the following required packages to your document preamble:
% \usepackage{graphicx}
\begin{table}[H]
\centering
\resizebox{\textwidth}{!}{%
\begin{tabular}{c|c}
\hline
\multicolumn{1}{|c|}{\textbf{Differential Cryptanalysis Concepts}}                                                                & \multicolumn{1}{c|}{\textbf{Linear Cryptanalysis Concepts}}                                                                                                    \\ \hline
Chosen-plaintext attack (CPA)                                                                                                     & Known-plaintext attack (KPA)                                                                                                                                   \\ \hline
\begin{tabular}[c]{@{}c@{}}Invented by \\ Biham and Shamir \cite{Shamir}\end{tabular}                            & \begin{tabular}[c]{@{}c@{}}Invented by \\ Matsui \cite{Matsui1993LinearCM}\end{tabular}                                                                         \\ \hline
\begin{tabular}[c]{@{}c@{}}Difference: \\ $X' = X \oplus X^*$\end{tabular}                                                        & \begin{tabular}[c]{@{}c@{}}Bitmask: \\ $X \cdot \gamma$\end{tabular}                                                                                           \\ \hline
DDT (Difference Distribution Table)                                                                                                                               & LAT (Linear Approximation Table)                                                                                                                                                            \\ \hline
\begin{tabular}[c]{@{}c@{}}Active $S$-Box: \\ input difference $\neq 0$\end{tabular}                                              & \begin{tabular}[c]{@{}c@{}}Active $S$-Box: \\ output mask $\neq 0$\end{tabular}                                                                                \\ \hline
\begin{tabular}[c]{@{}c@{}}Distinguisher: \\ differential characteristic\end{tabular}                                             & \begin{tabular}[c]{@{}c@{}}Distinguisher: \\ linear relation (or approximation)\end{tabular}                                                                   \\ \hline
\begin{tabular}[c]{@{}c@{}}Probability of an $r$-round \\ differential characteristic: \\ $p = \prod_{i=1}^r p_i$\end{tabular}    & \begin{tabular}[c]{@{}c@{}}Bias of an $r$-round \\ linear relation: \\ $\epsilon = 2^{r-1}\cdot \prod_{i=1}^r \epsilon_i$, \\ from the Piling-up Lemma\end{tabular} \\ \hline
\begin{tabular}[c]{@{}c@{}}Chosen-plaintexts \\ for a characteristic with \\ probability $p$: \\ approximately $1/p$\end{tabular} & \begin{tabular}[c]{@{}c@{}}Known-plaintexts \\ for a linear relation with \\ bias $\epsilon$: \\ approximately $1/\epsilon^2$\end{tabular}                     \\ \hline
\begin{tabular}[c]{@{}c@{}}Differential trail \\ (follow non-zero differences)\end{tabular}                                       & \begin{tabular}[c]{@{}c@{}}Linear trail \\ (follow non-zero masks)\end{tabular}                                                                               
\end{tabular}%
}
\caption{Comparing DC and LC concepts}
\label{tbl:dc-versus-lc}
\end{table}

\begin{figure}[H]
    \centering
    \includegraphics[scale=0.5]{Shamir-DES.png}
    \caption{Propagation of differences in DC (differential trails)}
    \label{fig:diftrail}
\end{figure}

\begin{figure}[H]
    \centering
    \includegraphics[scale=0.5]{Matsui-DES.png}
    \caption{Propagation of linear relations in LC (linear trails)}
    \label{fig:lintrail}
\end{figure}

\section{Conclusions}
In this chapter, we gave a brief overview of Linear Cryptanalysis and explained how to obtain LATs --- Linear Approximation Tables. We also presented LATs for DES $S$-Boxes and AES $S$-Boxes, as well as their linear uniformities. We leave the application of an actual LC attack as a future work.

\bibliographystyle{plain}
\bibliography{refs}

\end{document}
