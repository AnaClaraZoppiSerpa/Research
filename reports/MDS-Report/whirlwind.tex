% Whirlwind

Matrices (\ref{mat:whirlwind-m0}) and (\ref{mat:whirlwind-m1}) are used in the Whirlwind hash function, with the irreducible polynomial $p(x) = x^4+x+1$. The inverses are, respectively, (\ref{mat:whirlwind-m0-inv}) and (\ref{mat:whirlwind-m1-inv}). In \cite{Whirlwind2010}, it is claimed that the matrix is MDS. However, in our work, we have found singular submatrices, which leads us to believe, according to Theorem \ref{teo:mds}, that it is not MDS.

For example, the smallest singular submatrix we have found (matrix (\ref{mat:singular-whirlwind-m0})) for (\ref{mat:whirlwind-m0}) is $2 \times 2$. It is clear that the determinant is $(x+1)(x^2+x) - x(x^2+1) = x^3+x^2+x^2+x - (x^3+x) = x^3+x-x^3-x = 0$.

\begin{equation}\label{mat:singular-whirlwind-m0}
\begin{bmatrix}
3_x & 5_x\\
2_x & 6_x
\end{bmatrix}
=
\begin{bmatrix}
x+1 & x^2+1\\
x & x^2+x
\end{bmatrix}
\end{equation}

There are too many singular submatrices to list, therefore, we list which rows and columns should be removed from the original matrix in order to obtain them, in Table \ref{tbl:m0-singular}. Our row/column index starts at $0$ and ends at $7$.

\begin{footnotesize}
\begin{longtable}[c]{|l|l|}
\hline
\textbf{Rows to be removed} & \textbf{Columns to be removed} \\ \hline
\endfirsthead
\endhead
(0, 3)
&(0, 7)
\\ \hline
(0, 3)
&(3, 4)
\\ \hline
(0, 5)
&(0, 7)
\\ \hline
(0, 5)
&(2, 5)
\\ \hline
(0, 6)
&(2, 5)
\\ \hline
(0, 6)
&(3, 4)
\\ \hline
(0, 7)
&(0, 3)
\\ \hline
(0, 7)
&(0, 5)
\\ \hline
(0, 7)
&(2, 4)
\\ \hline
(0, 7)
&(2, 7)
\\ \hline
(0, 7)
&(3, 5)
\\ \hline
(0, 7)
&(4, 7)
\\ \hline
(1, 2)
&(1, 6)
\\ \hline
(1, 2)
&(2, 5)
\\ \hline
(1, 4)
&(1, 6)
\\ \hline
(1, 4)
&(3, 4)
\\ \hline
(1, 6)
&(1, 2)
\\ \hline
(1, 6)
&(1, 4)
\\ \hline
(1, 6)
&(2, 4)
\\ \hline
(1, 6)
&(3, 5)
\\ \hline
(1, 6)
&(3, 6)
\\ \hline
(1, 6)
&(5, 6)
\\ \hline
(1, 7)
&(2, 5)
\\ \hline
(1, 7)
&(3, 4)
\\ \hline
(2, 4)
&(0, 7)
\\ \hline
(2, 4)
&(1, 6)
\\ \hline
(2, 5)
&(0, 5)
\\ \hline
(2, 5)
&(0, 6)
\\ \hline
(2, 5)
&(1, 2)
\\ \hline
(2, 5)
&(1, 7)
\\ \hline
(2, 5)
&(2, 7)
\\ \hline
(2, 5)
&(5, 6)
\\ \hline
(2, 7)
&(0, 7)
\\ \hline
(2, 7)
&(2, 5)
\\ \hline
(3, 4)
&(0, 3)
\\ \hline
(3, 4)
&(0, 6)
\\ \hline
(3, 4)
&(1, 4)
\\ \hline
(3, 4)
&(1, 7)
\\ \hline
(3, 4)
&(3, 6)
\\ \hline
(3, 4)
&(4, 7)
\\ \hline
(3, 5)
&(0, 7)
\\ \hline
(3, 5)
&(1, 6)
\\ \hline
(3, 6)
&(1, 6)
\\ \hline
(3, 6)
&(3, 4)
\\ \hline
(4, 7)
&(0, 7)
\\ \hline
(4, 7)
&(3, 4)
\\ \hline
(5, 6)
&(1, 6)
\\ \hline
(5, 6)
&(2, 5)
\\ \hline
(0, 1, 2)
&(0, 3, 7)
\\ \hline
(0, 1, 2)
&(3, 4, 6)
\\ \hline
(0, 1, 3)
&(1, 2, 6)
\\ \hline
(0, 1, 3)
&(2, 5, 7)
\\ \hline
(0, 1, 4)
&(1, 3, 7)
\\ \hline
(0, 1, 5)
&(0, 2, 6)
\\ \hline
(0, 1, 6)
&(0, 5, 7)
\\ \hline
(0, 1, 6)
&(1, 2, 4)
\\ \hline
(0, 1, 6)
&(3, 5, 6)
\\ \hline
(0, 1, 7)
&(0, 3, 5)
\\ \hline
(0, 1, 7)
&(1, 4, 6)
\\ \hline
(0, 1, 7)
&(2, 4, 7)
\\ \hline
(0, 2, 3)
&(1, 2, 5)
\\ \hline
(0, 2, 3)
&(1, 4, 6)
\\ \hline
(0, 2, 4)
&(1, 2, 4)
\\ \hline
(0, 2, 4)
&(2, 3, 7)
\\ \hline
(0, 2, 5)
&(0, 5, 6)
\\ \hline
(0, 2, 5)
&(1, 2, 7)
\\ \hline
(0, 2, 5)
&(3, 4, 5)
\\ \hline
(0, 2, 5)
&(4, 6, 7)
\\ \hline
(0, 2, 6)
&(0, 1, 5)
\\ \hline
(0, 2, 6)
&(0, 3, 6)
\\ \hline
(0, 2, 7)
&(0, 3, 5)
\\ \hline
(0, 2, 7)
&(1, 6, 7)
\\ \hline
(0, 2, 7)
&(2, 4, 7)
\\ \hline
(0, 2, 7)
&(4, 5, 6)
\\ \hline
(0, 3, 4)
&(0, 3, 6)
\\ \hline
(0, 3, 4)
&(1, 2, 3)
\\ \hline
(0, 3, 4)
&(1, 4, 7)
\\ \hline
(0, 3, 5)
&(0, 1, 7)
\\ \hline
(0, 3, 5)
&(0, 2, 7)
\\ \hline
(0, 3, 5)
&(0, 3, 7)
\\ \hline
(0, 3, 5)
&(0, 4, 7)
\\ \hline
(0, 3, 5)
&(0, 5, 7)
\\ \hline
(0, 3, 5)
&(0, 6, 7)
\\ \hline
(0, 3, 5)
&(1, 3, 5)
\\ \hline
(0, 3, 6)
&(0, 2, 6)
\\ \hline
(0, 3, 6)
&(0, 3, 4)
\\ \hline
(0, 3, 6)
&(1, 3, 4)
\\ \hline
(0, 3, 6)
&(2, 3, 4)
\\ \hline
(0, 3, 6)
&(3, 4, 5)
\\ \hline
(0, 3, 6)
&(3, 4, 6)
\\ \hline
(0, 3, 6)
&(3, 4, 7)
\\ \hline
(0, 3, 7)
&(0, 1, 2)
\\ \hline
(0, 3, 7)
&(0, 3, 5)
\\ \hline
(0, 3, 7)
&(2, 4, 7)
\\ \hline
(0, 4, 5)
&(3, 5, 7)
\\ \hline
(0, 4, 6)
&(0, 5, 6)
\\ \hline
(0, 4, 6)
&(3, 6, 7)
\\ \hline
(0, 4, 7)
&(0, 3, 5)
\\ \hline
(0, 4, 7)
&(2, 4, 7)
\\ \hline
(0, 4, 7)
&(5, 6, 7)
\\ \hline
(0, 5, 6)
&(0, 2, 5)
\\ \hline
(0, 5, 6)
&(0, 4, 6)
\\ \hline
(0, 5, 6)
&(1, 2, 5)
\\ \hline
(0, 5, 6)
&(2, 3, 5)
\\ \hline
(0, 5, 6)
&(2, 4, 5)
\\ \hline
(0, 5, 6)
&(2, 5, 6)
\\ \hline
(0, 5, 6)
&(2, 5, 7)
\\ \hline
(0, 5, 7)
&(0, 1, 6)
\\ \hline
(0, 5, 7)
&(0, 3, 5)
\\ \hline
(0, 5, 7)
&(1, 2, 3)
\\ \hline
(0, 5, 7)
&(2, 4, 7)
\\ \hline
(0, 6, 7)
&(0, 3, 5)
\\ \hline
(0, 6, 7)
&(1, 3, 6)
\\ \hline
(0, 6, 7)
&(2, 4, 7)
\\ \hline
(1, 2, 3)
&(0, 3, 4)
\\ \hline
(1, 2, 3)
&(0, 5, 7)
\\ \hline
(1, 2, 4)
&(0, 1, 6)
\\ \hline
(1, 2, 4)
&(0, 2, 4)
\\ \hline
(1, 2, 4)
&(1, 2, 6)
\\ \hline
(1, 2, 4)
&(1, 3, 6)
\\ \hline
(1, 2, 4)
&(1, 4, 6)
\\ \hline
(1, 2, 4)
&(1, 5, 6)
\\ \hline
(1, 2, 4)
&(1, 6, 7)
\\ \hline
(1, 2, 5)
&(0, 2, 3)
\\ \hline
(1, 2, 5)
&(0, 5, 6)
\\ \hline
(1, 2, 5)
&(1, 2, 7)
\\ \hline
(1, 2, 6)
&(0, 1, 3)
\\ \hline
(1, 2, 6)
&(1, 2, 4)
\\ \hline
(1, 2, 6)
&(3, 5, 6)
\\ \hline
(1, 2, 7)
&(0, 2, 5)
\\ \hline
(1, 2, 7)
&(1, 2, 5)
\\ \hline
(1, 2, 7)
&(1, 3, 7)
\\ \hline
(1, 2, 7)
&(2, 3, 5)
\\ \hline
(1, 2, 7)
&(2, 4, 5)
\\ \hline
(1, 2, 7)
&(2, 5, 6)
\\ \hline
(1, 2, 7)
&(2, 5, 7)
\\ \hline
(1, 3, 4)
&(0, 3, 6)
\\ \hline
(1, 3, 4)
&(1, 4, 7)
\\ \hline
(1, 3, 4)
&(2, 4, 5)
\\ \hline
(1, 3, 4)
&(5, 6, 7)
\\ \hline
(1, 3, 5)
&(0, 3, 5)
\\ \hline
(1, 3, 5)
&(2, 3, 6)
\\ \hline
(1, 3, 6)
&(0, 6, 7)
\\ \hline
(1, 3, 6)
&(1, 2, 4)
\\ \hline
(1, 3, 6)
&(3, 5, 6)
\\ \hline
(1, 3, 6)
&(4, 5, 7)
\\ \hline
(1, 3, 7)
&(0, 1, 4)
\\ \hline
(1, 3, 7)
&(1, 2, 7)
\\ \hline
(1, 4, 5)
&(2, 4, 6)
\\ \hline
(1, 4, 6)
&(0, 1, 7)
\\ \hline
(1, 4, 6)
&(0, 2, 3)
\\ \hline
(1, 4, 6)
&(1, 2, 4)
\\ \hline
(1, 4, 6)
&(3, 5, 6)
\\ \hline
(1, 4, 7)
&(0, 3, 4)
\\ \hline
(1, 4, 7)
&(1, 3, 4)
\\ \hline
(1, 4, 7)
&(1, 5, 7)
\\ \hline
(1, 4, 7)
&(2, 3, 4)
\\ \hline
(1, 4, 7)
&(3, 4, 5)
\\ \hline
(1, 4, 7)
&(3, 4, 6)
\\ \hline
(1, 4, 7)
&(3, 4, 7)
\\ \hline
(1, 5, 6)
&(1, 2, 4)
\\ \hline
(1, 5, 6)
&(3, 5, 6)
\\ \hline
(1, 5, 6)
&(4, 6, 7)
\\ \hline
(1, 5, 7)
&(1, 4, 7)
\\ \hline
(1, 5, 7)
&(2, 6, 7)
\\ \hline
(1, 6, 7)
&(0, 2, 7)
\\ \hline
(1, 6, 7)
&(1, 2, 4)
\\ \hline
(1, 6, 7)
&(3, 5, 6)
\\ \hline
(2, 3, 4)
&(0, 3, 6)
\\ \hline
(2, 3, 4)
&(1, 4, 7)
\\ \hline
(2, 3, 4)
&(2, 5, 7)
\\ \hline
(2, 3, 5)
&(0, 5, 6)
\\ \hline
(2, 3, 5)
&(1, 2, 7)
\\ \hline
(2, 3, 5)
&(3, 4, 6)
\\ \hline
(2, 3, 6)
&(1, 3, 5)
\\ \hline
(2, 3, 7)
&(0, 2, 4)
\\ \hline
(2, 4, 5)
&(0, 5, 6)
\\ \hline
(2, 4, 5)
&(1, 2, 7)
\\ \hline
(2, 4, 5)
&(1, 3, 4)
\\ \hline
(2, 4, 6)
&(1, 4, 5)
\\ \hline
(2, 4, 6)
&(2, 4, 7)
\\ \hline
(2, 4, 7)
&(0, 1, 7)
\\ \hline
(2, 4, 7)
&(0, 2, 7)
\\ \hline
(2, 4, 7)
&(0, 3, 7)
\\ \hline
(2, 4, 7)
&(0, 4, 7)
\\ \hline
(2, 4, 7)
&(0, 5, 7)
\\ \hline
(2, 4, 7)
&(0, 6, 7)
\\ \hline
(2, 4, 7)
&(2, 4, 6)
\\ \hline
(2, 5, 6)
&(0, 5, 6)
\\ \hline
(2, 5, 6)
&(1, 2, 7)
\\ \hline
(2, 5, 6)
&(4, 5, 7)
\\ \hline
(2, 5, 7)
&(0, 1, 3)
\\ \hline
(2, 5, 7)
&(0, 5, 6)
\\ \hline
(2, 5, 7)
&(1, 2, 7)
\\ \hline
(2, 5, 7)
&(2, 3, 4)
\\ \hline
(2, 6, 7)
&(1, 5, 7)
\\ \hline
(3, 4, 5)
&(0, 2, 5)
\\ \hline
(3, 4, 5)
&(0, 3, 6)
\\ \hline
(3, 4, 5)
&(1, 4, 7)
\\ \hline
(3, 4, 6)
&(0, 1, 2)
\\ \hline
(3, 4, 6)
&(0, 3, 6)
\\ \hline
(3, 4, 6)
&(1, 4, 7)
\\ \hline
(3, 4, 6)
&(2, 3, 5)
\\ \hline
(3, 4, 7)
&(0, 3, 6)
\\ \hline
(3, 4, 7)
&(1, 4, 7)
\\ \hline
(3, 4, 7)
&(4, 5, 6)
\\ \hline
(3, 5, 6)
&(0, 1, 6)
\\ \hline
(3, 5, 6)
&(1, 2, 6)
\\ \hline
(3, 5, 6)
&(1, 3, 6)
\\ \hline
(3, 5, 6)
&(1, 4, 6)
\\ \hline
(3, 5, 6)
&(1, 5, 6)
\\ \hline
(3, 5, 6)
&(1, 6, 7)
\\ \hline
(3, 5, 6)
&(3, 5, 7)
\\ \hline
(3, 5, 7)
&(0, 4, 5)
\\ \hline
(3, 5, 7)
&(3, 5, 6)
\\ \hline
(3, 6, 7)
&(0, 4, 6)
\\ \hline
(4, 5, 6)
&(0, 2, 7)
\\ \hline
(4, 5, 6)
&(3, 4, 7)
\\ \hline
(4, 5, 7)
&(1, 3, 6)
\\ \hline
(4, 5, 7)
&(2, 5, 6)
\\ \hline
(4, 6, 7)
&(0, 2, 5)
\\ \hline
(4, 6, 7)
&(1, 5, 6)
\\ \hline
(5, 6, 7)
&(0, 4, 7)
\\ \hline
(5, 6, 7)
&(1, 3, 4)
\\ \hline
(0, 1, 2, 3)
&(0, 3, 5, 6)
\\ \hline
(0, 1, 2, 3)
&(1, 2, 4, 7)
\\ \hline
(0, 1, 2, 4)
&(0, 4, 6, 7)
\\ \hline
(0, 1, 2, 4)
&(2, 4, 5, 6)
\\ \hline
(0, 1, 2, 5)
&(1, 2, 4, 6)
\\ \hline
(0, 1, 2, 5)
&(1, 3, 5, 6)
\\ \hline
(0, 1, 2, 6)
&(0, 4, 5, 6)
\\ \hline
(0, 1, 2, 6)
&(1, 3, 4, 5)
\\ \hline
(0, 1, 2, 6)
&(2, 3, 5, 7)
\\ \hline
(0, 1, 2, 6)
&(2, 4, 6, 7)
\\ \hline
(0, 1, 3, 4)
&(0, 2, 4, 7)
\\ \hline
(0, 1, 3, 4)
&(0, 3, 5, 7)
\\ \hline
(0, 1, 3, 5)
&(1, 5, 6, 7)
\\ \hline
(0, 1, 3, 5)
&(3, 4, 5, 7)
\\ \hline
(0, 1, 3, 7)
&(0, 2, 4, 5)
\\ \hline
(0, 1, 3, 7)
&(1, 4, 5, 7)
\\ \hline
(0, 1, 3, 7)
&(2, 3, 4, 6)
\\ \hline
(0, 1, 3, 7)
&(3, 5, 6, 7)
\\ \hline
(0, 1, 4, 6)
&(1, 2, 3, 5)
\\ \hline
(0, 1, 4, 6)
&(3, 4, 5, 7)
\\ \hline
(0, 1, 4, 7)
&(0, 2, 5, 6)
\\ \hline
(0, 1, 4, 7)
&(0, 3, 6, 7)
\\ \hline
(0, 1, 4, 7)
&(1, 3, 5, 6)
\\ \hline
(0, 1, 5, 6)
&(0, 2, 4, 7)
\\ \hline
(0, 1, 5, 6)
&(1, 2, 6, 7)
\\ \hline
(0, 1, 5, 6)
&(1, 3, 4, 7)
\\ \hline
(0, 1, 5, 7)
&(0, 2, 3, 4)
\\ \hline
(0, 1, 5, 7)
&(2, 4, 5, 6)
\\ \hline
(0, 2, 3, 4)
&(0, 1, 5, 7)
\\ \hline
(0, 2, 3, 4)
&(0, 4, 5, 6)
\\ \hline
(0, 2, 3, 4)
&(1, 3, 6, 7)
\\ \hline
(0, 2, 3, 4)
&(2, 4, 6, 7)
\\ \hline
(0, 2, 3, 6)
&(0, 4, 6, 7)
\\ \hline
(0, 2, 3, 6)
&(2, 4, 5, 6)
\\ \hline
(0, 2, 3, 7)
&(0, 3, 4, 6)
\\ \hline
(0, 2, 3, 7)
&(1, 3, 4, 7)
\\ \hline
(0, 2, 4, 5)
&(0, 1, 3, 7)
\\ \hline
(0, 2, 4, 5)
&(1, 5, 6, 7)
\\ \hline
(0, 2, 4, 6)
&(0, 3, 4, 7)
\\ \hline
(0, 2, 4, 6)
&(1, 2, 5, 6)
\\ \hline
(0, 2, 4, 7)
&(0, 1, 3, 4)
\\ \hline
(0, 2, 4, 7)
&(0, 1, 5, 6)
\\ \hline
(0, 2, 4, 7)
&(2, 3, 5, 6)
\\ \hline
(0, 2, 4, 7)
&(3, 4, 6, 7)
\\ \hline
(0, 2, 5, 6)
&(0, 1, 4, 7)
\\ \hline
(0, 2, 5, 6)
&(1, 2, 3, 6)
\\ \hline
(0, 2, 5, 6)
&(1, 4, 5, 6)
\\ \hline
(0, 2, 5, 6)
&(2, 3, 4, 7)
\\ \hline
(0, 2, 6, 7)
&(1, 2, 3, 5)
\\ \hline
(0, 2, 6, 7)
&(3, 4, 5, 7)
\\ \hline
(0, 3, 4, 5)
&(1, 2, 4, 6)
\\ \hline
(0, 3, 4, 5)
&(1, 2, 5, 7)
\\ \hline
(0, 3, 4, 5)
&(2, 3, 4, 7)
\\ \hline
(0, 3, 4, 6)
&(0, 2, 3, 7)
\\ \hline
(0, 3, 4, 6)
&(0, 4, 5, 7)
\\ \hline
(0, 3, 4, 6)
&(1, 2, 4, 5)
\\ \hline
(0, 3, 4, 6)
&(1, 2, 6, 7)
\\ \hline
(0, 3, 4, 7)
&(0, 2, 4, 6)
\\ \hline
(0, 3, 4, 7)
&(1, 3, 5, 7)
\\ \hline
(0, 3, 5, 6)
&(0, 1, 2, 3)
\\ \hline
(0, 3, 5, 6)
&(4, 5, 6, 7)
\\ \hline
(0, 3, 5, 7)
&(0, 1, 3, 4)
\\ \hline
(0, 3, 5, 7)
&(1, 2, 4, 5)
\\ \hline
(0, 3, 5, 7)
&(1, 2, 6, 7)
\\ \hline
(0, 3, 5, 7)
&(3, 4, 6, 7)
\\ \hline
(0, 3, 6, 7)
&(0, 1, 4, 7)
\\ \hline
(0, 3, 6, 7)
&(1, 2, 4, 6)
\\ \hline
(0, 3, 6, 7)
&(1, 2, 5, 7)
\\ \hline
(0, 4, 5, 6)
&(0, 1, 2, 6)
\\ \hline
(0, 4, 5, 6)
&(0, 2, 3, 4)
\\ \hline
(0, 4, 5, 7)
&(0, 3, 4, 6)
\\ \hline
(0, 4, 5, 7)
&(1, 3, 4, 7)
\\ \hline
(0, 4, 6, 7)
&(0, 1, 2, 4)
\\ \hline
(0, 4, 6, 7)
&(0, 2, 3, 6)
\\ \hline
(0, 4, 6, 7)
&(1, 3, 4, 5)
\\ \hline
(0, 4, 6, 7)
&(2, 3, 5, 7)
\\ \hline
(1, 2, 3, 5)
&(0, 1, 4, 6)
\\ \hline
(1, 2, 3, 5)
&(0, 2, 6, 7)
\\ \hline
(1, 2, 3, 5)
&(1, 4, 5, 7)
\\ \hline
(1, 2, 3, 5)
&(3, 5, 6, 7)
\\ \hline
(1, 2, 3, 6)
&(0, 2, 5, 6)
\\ \hline
(1, 2, 3, 6)
&(1, 2, 5, 7)
\\ \hline
(1, 2, 3, 7)
&(1, 5, 6, 7)
\\ \hline
(1, 2, 3, 7)
&(3, 4, 5, 7)
\\ \hline
(1, 2, 4, 5)
&(0, 3, 4, 6)
\\ \hline
(1, 2, 4, 5)
&(0, 3, 5, 7)
\\ \hline
(1, 2, 4, 5)
&(2, 3, 5, 6)
\\ \hline
(1, 2, 4, 6)
&(0, 1, 2, 5)
\\ \hline
(1, 2, 4, 6)
&(0, 3, 4, 5)
\\ \hline
(1, 2, 4, 6)
&(0, 3, 6, 7)
\\ \hline
(1, 2, 4, 6)
&(2, 5, 6, 7)
\\ \hline
(1, 2, 4, 7)
&(0, 1, 2, 3)
\\ \hline
(1, 2, 4, 7)
&(4, 5, 6, 7)
\\ \hline
(1, 2, 5, 6)
&(0, 2, 4, 6)
\\ \hline
(1, 2, 5, 6)
&(1, 3, 5, 7)
\\ \hline
(1, 2, 5, 7)
&(0, 3, 4, 5)
\\ \hline
(1, 2, 5, 7)
&(0, 3, 6, 7)
\\ \hline
(1, 2, 5, 7)
&(1, 2, 3, 6)
\\ \hline
(1, 2, 5, 7)
&(1, 4, 5, 6)
\\ \hline
(1, 2, 6, 7)
&(0, 1, 5, 6)
\\ \hline
(1, 2, 6, 7)
&(0, 3, 4, 6)
\\ \hline
(1, 2, 6, 7)
&(0, 3, 5, 7)
\\ \hline
(1, 3, 4, 5)
&(0, 1, 2, 6)
\\ \hline
(1, 3, 4, 5)
&(0, 4, 6, 7)
\\ \hline
(1, 3, 4, 7)
&(0, 1, 5, 6)
\\ \hline
(1, 3, 4, 7)
&(0, 2, 3, 7)
\\ \hline
(1, 3, 4, 7)
&(0, 4, 5, 7)
\\ \hline
(1, 3, 4, 7)
&(2, 3, 5, 6)
\\ \hline
(1, 3, 5, 6)
&(0, 1, 2, 5)
\\ \hline
(1, 3, 5, 6)
&(0, 1, 4, 7)
\\ \hline
(1, 3, 5, 6)
&(2, 3, 4, 7)
\\ \hline
(1, 3, 5, 6)
&(2, 5, 6, 7)
\\ \hline
(1, 3, 5, 7)
&(0, 3, 4, 7)
\\ \hline
(1, 3, 5, 7)
&(1, 2, 5, 6)
\\ \hline
(1, 3, 6, 7)
&(0, 2, 3, 4)
\\ \hline
(1, 3, 6, 7)
&(2, 4, 5, 6)
\\ \hline
(1, 4, 5, 6)
&(0, 2, 5, 6)
\\ \hline
(1, 4, 5, 6)
&(1, 2, 5, 7)
\\ \hline
(1, 4, 5, 7)
&(0, 1, 3, 7)
\\ \hline
(1, 4, 5, 7)
&(1, 2, 3, 5)
\\ \hline
(1, 5, 6, 7)
&(0, 1, 3, 5)
\\ \hline
(1, 5, 6, 7)
&(0, 2, 4, 5)
\\ \hline
(1, 5, 6, 7)
&(1, 2, 3, 7)
\\ \hline
(1, 5, 6, 7)
&(2, 3, 4, 6)
\\ \hline
(2, 3, 4, 6)
&(0, 1, 3, 7)
\\ \hline
(2, 3, 4, 6)
&(1, 5, 6, 7)
\\ \hline
(2, 3, 4, 7)
&(0, 2, 5, 6)
\\ \hline
(2, 3, 4, 7)
&(0, 3, 4, 5)
\\ \hline
(2, 3, 4, 7)
&(1, 3, 5, 6)
\\ \hline
(2, 3, 5, 6)
&(0, 2, 4, 7)
\\ \hline
(2, 3, 5, 6)
&(1, 2, 4, 5)
\\ \hline
(2, 3, 5, 6)
&(1, 3, 4, 7)
\\ \hline
(2, 3, 5, 7)
&(0, 1, 2, 6)
\\ \hline
(2, 3, 5, 7)
&(0, 4, 6, 7)
\\ \hline
(2, 4, 5, 6)
&(0, 1, 2, 4)
\\ \hline
(2, 4, 5, 6)
&(0, 1, 5, 7)
\\ \hline
(2, 4, 5, 6)
&(0, 2, 3, 6)
\\ \hline
(2, 4, 5, 6)
&(1, 3, 6, 7)
\\ \hline
(2, 4, 6, 7)
&(0, 1, 2, 6)
\\ \hline
(2, 4, 6, 7)
&(0, 2, 3, 4)
\\ \hline
(2, 5, 6, 7)
&(1, 2, 4, 6)
\\ \hline
(2, 5, 6, 7)
&(1, 3, 5, 6)
\\ \hline
(3, 4, 5, 7)
&(0, 1, 3, 5)
\\ \hline
(3, 4, 5, 7)
&(0, 1, 4, 6)
\\ \hline
(3, 4, 5, 7)
&(0, 2, 6, 7)
\\ \hline
(3, 4, 5, 7)
&(1, 2, 3, 7)
\\ \hline
(3, 4, 6, 7)
&(0, 2, 4, 7)
\\ \hline
(3, 4, 6, 7)
&(0, 3, 5, 7)
\\ \hline
(3, 5, 6, 7)
&(0, 1, 3, 7)
\\ \hline
(3, 5, 6, 7)
&(1, 2, 3, 5)
\\ \hline
(4, 5, 6, 7)
&(0, 3, 5, 6)
\\ \hline
(4, 5, 6, 7)
&(1, 2, 4, 7)
\\ \hline
(0, 1, 2, 3, 4)
&(0, 2, 5, 6, 7)
\\ \hline
(0, 1, 2, 3, 4)
&(1, 2, 3, 5, 6)
\\ \hline
(0, 1, 2, 3, 5)
&(0, 2, 3, 4, 7)
\\ \hline
(0, 1, 2, 3, 5)
&(1, 3, 4, 6, 7)
\\ \hline
(0, 1, 2, 3, 6)
&(0, 1, 3, 4, 7)
\\ \hline
(0, 1, 2, 3, 6)
&(0, 2, 4, 5, 7)
\\ \hline
(0, 1, 2, 3, 7)
&(0, 1, 2, 5, 6)
\\ \hline
(0, 1, 2, 3, 7)
&(1, 3, 4, 5, 6)
\\ \hline
(0, 1, 2, 4, 5)
&(1, 2, 3, 5, 7)
\\ \hline
(0, 1, 2, 4, 6)
&(0, 1, 2, 4, 7)
\\ \hline
(0, 1, 2, 4, 6)
&(1, 2, 3, 6, 7)
\\ \hline
(0, 1, 2, 4, 7)
&(0, 1, 2, 4, 6)
\\ \hline
(0, 1, 2, 4, 7)
&(0, 2, 3, 4, 5)
\\ \hline
(0, 1, 2, 4, 7)
&(0, 2, 3, 4, 7)
\\ \hline
(0, 1, 2, 4, 7)
&(0, 2, 3, 5, 7)
\\ \hline
(0, 1, 2, 4, 7)
&(0, 2, 4, 5, 7)
\\ \hline
(0, 1, 2, 4, 7)
&(0, 3, 4, 5, 7)
\\ \hline
(0, 1, 2, 4, 7)
&(2, 3, 4, 5, 7)
\\ \hline
(0, 1, 2, 5, 6)
&(0, 1, 2, 3, 7)
\\ \hline
(0, 1, 2, 5, 6)
&(0, 2, 3, 5, 6)
\\ \hline
(0, 1, 2, 5, 6)
&(1, 2, 4, 5, 7)
\\ \hline
(0, 1, 2, 5, 7)
&(0, 1, 4, 6, 7)
\\ \hline
(0, 1, 2, 5, 7)
&(0, 2, 3, 5, 6)
\\ \hline
(0, 1, 2, 5, 7)
&(1, 2, 4, 5, 7)
\\ \hline
(0, 1, 2, 5, 7)
&(3, 4, 5, 6, 7)
\\ \hline
(0, 1, 2, 6, 7)
&(0, 2, 3, 5, 6)
\\ \hline
(0, 1, 2, 6, 7)
&(1, 2, 4, 5, 7)
\\ \hline
(0, 1, 2, 6, 7)
&(1, 3, 4, 6, 7)
\\ \hline
(0, 1, 3, 4, 5)
&(0, 2, 3, 4, 6)
\\ \hline
(0, 1, 3, 4, 6)
&(0, 1, 5, 6, 7)
\\ \hline
(0, 1, 3, 4, 6)
&(0, 3, 4, 5, 6)
\\ \hline
(0, 1, 3, 4, 6)
&(1, 2, 3, 4, 7)
\\ \hline
(0, 1, 3, 4, 6)
&(2, 4, 5, 6, 7)
\\ \hline
(0, 1, 3, 4, 7)
&(0, 1, 2, 3, 6)
\\ \hline
(0, 1, 3, 4, 7)
&(0, 3, 4, 5, 6)
\\ \hline
(0, 1, 3, 4, 7)
&(1, 2, 3, 4, 7)
\\ \hline
(0, 1, 3, 5, 6)
&(0, 1, 3, 5, 7)
\\ \hline
(0, 1, 3, 5, 6)
&(1, 2, 3, 4, 5)
\\ \hline
(0, 1, 3, 5, 6)
&(1, 2, 3, 4, 6)
\\ \hline
(0, 1, 3, 5, 6)
&(1, 2, 3, 5, 6)
\\ \hline
(0, 1, 3, 5, 6)
&(1, 2, 4, 5, 6)
\\ \hline
(0, 1, 3, 5, 6)
&(1, 3, 4, 5, 6)
\\ \hline
(0, 1, 3, 5, 6)
&(2, 3, 4, 5, 6)
\\ \hline
(0, 1, 3, 5, 7)
&(0, 1, 3, 5, 6)
\\ \hline
(0, 1, 3, 5, 7)
&(0, 2, 3, 6, 7)
\\ \hline
(0, 1, 3, 6, 7)
&(0, 2, 5, 6, 7)
\\ \hline
(0, 1, 3, 6, 7)
&(0, 3, 4, 5, 6)
\\ \hline
(0, 1, 3, 6, 7)
&(1, 2, 3, 4, 7)
\\ \hline
(0, 1, 4, 5, 6)
&(1, 3, 5, 6, 7)
\\ \hline
(0, 1, 4, 5, 7)
&(0, 2, 4, 6, 7)
\\ \hline
(0, 1, 4, 6, 7)
&(0, 1, 2, 5, 7)
\\ \hline
(0, 1, 4, 6, 7)
&(0, 3, 4, 5, 6)
\\ \hline
(0, 1, 4, 6, 7)
&(1, 2, 3, 4, 7)
\\ \hline
(0, 1, 5, 6, 7)
&(0, 1, 3, 4, 6)
\\ \hline
(0, 1, 5, 6, 7)
&(0, 2, 3, 5, 6)
\\ \hline
(0, 1, 5, 6, 7)
&(1, 2, 4, 5, 7)
\\ \hline
(0, 2, 3, 4, 5)
&(0, 1, 2, 4, 7)
\\ \hline
(0, 2, 3, 4, 5)
&(0, 3, 5, 6, 7)
\\ \hline
(0, 2, 3, 4, 5)
&(1, 3, 4, 5, 6)
\\ \hline
(0, 2, 3, 4, 6)
&(0, 1, 3, 4, 5)
\\ \hline
(0, 2, 3, 4, 6)
&(0, 2, 3, 5, 6)
\\ \hline
(0, 2, 3, 4, 7)
&(0, 1, 2, 3, 5)
\\ \hline
(0, 2, 3, 4, 7)
&(0, 1, 2, 4, 7)
\\ \hline
(0, 2, 3, 4, 7)
&(0, 3, 5, 6, 7)
\\ \hline
(0, 2, 3, 5, 6)
&(0, 1, 2, 5, 6)
\\ \hline
(0, 2, 3, 5, 6)
&(0, 1, 2, 5, 7)
\\ \hline
(0, 2, 3, 5, 6)
&(0, 1, 2, 6, 7)
\\ \hline
(0, 2, 3, 5, 6)
&(0, 1, 5, 6, 7)
\\ \hline
(0, 2, 3, 5, 6)
&(0, 2, 3, 4, 6)
\\ \hline
(0, 2, 3, 5, 6)
&(0, 2, 5, 6, 7)
\\ \hline
(0, 2, 3, 5, 6)
&(1, 2, 5, 6, 7)
\\ \hline
(0, 2, 3, 5, 7)
&(0, 1, 2, 4, 7)
\\ \hline
(0, 2, 3, 5, 7)
&(0, 3, 5, 6, 7)
\\ \hline
(0, 2, 3, 5, 7)
&(1, 4, 5, 6, 7)
\\ \hline
(0, 2, 3, 5, 7)
&(2, 3, 4, 5, 6)
\\ \hline
(0, 2, 3, 6, 7)
&(0, 1, 3, 5, 7)
\\ \hline
(0, 2, 4, 5, 6)
&(0, 3, 4, 5, 6)
\\ \hline
(0, 2, 4, 5, 6)
&(2, 3, 5, 6, 7)
\\ \hline
(0, 2, 4, 5, 7)
&(0, 1, 2, 3, 6)
\\ \hline
(0, 2, 4, 5, 7)
&(0, 1, 2, 4, 7)
\\ \hline
(0, 2, 4, 5, 7)
&(0, 3, 5, 6, 7)
\\ \hline
(0, 2, 4, 5, 7)
&(1, 2, 3, 4, 5)
\\ \hline
(0, 2, 4, 6, 7)
&(0, 1, 4, 5, 7)
\\ \hline
(0, 2, 4, 6, 7)
&(1, 2, 4, 6, 7)
\\ \hline
(0, 2, 5, 6, 7)
&(0, 1, 2, 3, 4)
\\ \hline
(0, 2, 5, 6, 7)
&(0, 1, 3, 6, 7)
\\ \hline
(0, 2, 5, 6, 7)
&(0, 2, 3, 5, 6)
\\ \hline
(0, 2, 5, 6, 7)
&(1, 2, 4, 5, 7)
\\ \hline
(0, 3, 4, 5, 6)
&(0, 1, 3, 4, 6)
\\ \hline
(0, 3, 4, 5, 6)
&(0, 1, 3, 4, 7)
\\ \hline
(0, 3, 4, 5, 6)
&(0, 1, 3, 6, 7)
\\ \hline
(0, 3, 4, 5, 6)
&(0, 1, 4, 6, 7)
\\ \hline
(0, 3, 4, 5, 6)
&(0, 2, 4, 5, 6)
\\ \hline
(0, 3, 4, 5, 6)
&(0, 3, 4, 6, 7)
\\ \hline
(0, 3, 4, 5, 6)
&(1, 3, 4, 6, 7)
\\ \hline
(0, 3, 4, 5, 7)
&(0, 1, 2, 4, 7)
\\ \hline
(0, 3, 4, 5, 7)
&(0, 3, 5, 6, 7)
\\ \hline
(0, 3, 4, 5, 7)
&(2, 4, 5, 6, 7)
\\ \hline
(0, 3, 4, 6, 7)
&(0, 3, 4, 5, 6)
\\ \hline
(0, 3, 4, 6, 7)
&(1, 2, 3, 4, 7)
\\ \hline
(0, 3, 4, 6, 7)
&(1, 4, 5, 6, 7)
\\ \hline
(0, 3, 5, 6, 7)
&(0, 2, 3, 4, 5)
\\ \hline
(0, 3, 5, 6, 7)
&(0, 2, 3, 4, 7)
\\ \hline
(0, 3, 5, 6, 7)
&(0, 2, 3, 5, 7)
\\ \hline
(0, 3, 5, 6, 7)
&(0, 2, 4, 5, 7)
\\ \hline
(0, 3, 5, 6, 7)
&(0, 3, 4, 5, 7)
\\ \hline
(0, 3, 5, 6, 7)
&(1, 3, 5, 6, 7)
\\ \hline
(0, 3, 5, 6, 7)
&(2, 3, 4, 5, 7)
\\ \hline
(0, 4, 5, 6, 7)
&(1, 2, 3, 4, 6)
\\ \hline
(0, 4, 5, 6, 7)
&(1, 2, 5, 6, 7)
\\ \hline
(1, 2, 3, 4, 5)
&(0, 1, 3, 5, 6)
\\ \hline
(1, 2, 3, 4, 5)
&(0, 2, 4, 5, 7)
\\ \hline
(1, 2, 3, 4, 5)
&(1, 2, 4, 6, 7)
\\ \hline
(1, 2, 3, 4, 6)
&(0, 1, 3, 5, 6)
\\ \hline
(1, 2, 3, 4, 6)
&(0, 4, 5, 6, 7)
\\ \hline
(1, 2, 3, 4, 6)
&(1, 2, 4, 6, 7)
\\ \hline
(1, 2, 3, 4, 6)
&(2, 3, 4, 5, 7)
\\ \hline
(1, 2, 3, 4, 7)
&(0, 1, 3, 4, 6)
\\ \hline
(1, 2, 3, 4, 7)
&(0, 1, 3, 4, 7)
\\ \hline
(1, 2, 3, 4, 7)
&(0, 1, 3, 6, 7)
\\ \hline
(1, 2, 3, 4, 7)
&(0, 1, 4, 6, 7)
\\ \hline
(1, 2, 3, 4, 7)
&(0, 3, 4, 6, 7)
\\ \hline
(1, 2, 3, 4, 7)
&(1, 2, 3, 5, 7)
\\ \hline
(1, 2, 3, 4, 7)
&(1, 3, 4, 6, 7)
\\ \hline
(1, 2, 3, 5, 6)
&(0, 1, 2, 3, 4)
\\ \hline
(1, 2, 3, 5, 6)
&(0, 1, 3, 5, 6)
\\ \hline
(1, 2, 3, 5, 6)
&(1, 2, 4, 6, 7)
\\ \hline
(1, 2, 3, 5, 7)
&(0, 1, 2, 4, 5)
\\ \hline
(1, 2, 3, 5, 7)
&(1, 2, 3, 4, 7)
\\ \hline
(1, 2, 3, 6, 7)
&(0, 1, 2, 4, 6)
\\ \hline
(1, 2, 4, 5, 6)
&(0, 1, 3, 5, 6)
\\ \hline
(1, 2, 4, 5, 6)
&(1, 2, 4, 6, 7)
\\ \hline
(1, 2, 4, 5, 6)
&(3, 4, 5, 6, 7)
\\ \hline
(1, 2, 4, 5, 7)
&(0, 1, 2, 5, 6)
\\ \hline
(1, 2, 4, 5, 7)
&(0, 1, 2, 5, 7)
\\ \hline
(1, 2, 4, 5, 7)
&(0, 1, 2, 6, 7)
\\ \hline
(1, 2, 4, 5, 7)
&(0, 1, 5, 6, 7)
\\ \hline
(1, 2, 4, 5, 7)
&(0, 2, 5, 6, 7)
\\ \hline
(1, 2, 4, 5, 7)
&(1, 2, 5, 6, 7)
\\ \hline
(1, 2, 4, 5, 7)
&(1, 3, 4, 5, 7)
\\ \hline
(1, 2, 4, 6, 7)
&(0, 2, 4, 6, 7)
\\ \hline
(1, 2, 4, 6, 7)
&(1, 2, 3, 4, 5)
\\ \hline
(1, 2, 4, 6, 7)
&(1, 2, 3, 4, 6)
\\ \hline
(1, 2, 4, 6, 7)
&(1, 2, 3, 5, 6)
\\ \hline
(1, 2, 4, 6, 7)
&(1, 2, 4, 5, 6)
\\ \hline
(1, 2, 4, 6, 7)
&(1, 3, 4, 5, 6)
\\ \hline
(1, 2, 4, 6, 7)
&(2, 3, 4, 5, 6)
\\ \hline
(1, 2, 5, 6, 7)
&(0, 2, 3, 5, 6)
\\ \hline
(1, 2, 5, 6, 7)
&(0, 4, 5, 6, 7)
\\ \hline
(1, 2, 5, 6, 7)
&(1, 2, 4, 5, 7)
\\ \hline
(1, 3, 4, 5, 6)
&(0, 1, 2, 3, 7)
\\ \hline
(1, 3, 4, 5, 6)
&(0, 1, 3, 5, 6)
\\ \hline
(1, 3, 4, 5, 6)
&(0, 2, 3, 4, 5)
\\ \hline
(1, 3, 4, 5, 6)
&(1, 2, 4, 6, 7)
\\ \hline
(1, 3, 4, 5, 7)
&(1, 2, 4, 5, 7)
\\ \hline
(1, 3, 4, 5, 7)
&(2, 3, 4, 6, 7)
\\ \hline
(1, 3, 4, 6, 7)
&(0, 1, 2, 3, 5)
\\ \hline
(1, 3, 4, 6, 7)
&(0, 1, 2, 6, 7)
\\ \hline
(1, 3, 4, 6, 7)
&(0, 3, 4, 5, 6)
\\ \hline
(1, 3, 4, 6, 7)
&(1, 2, 3, 4, 7)
\\ \hline
(1, 3, 5, 6, 7)
&(0, 1, 4, 5, 6)
\\ \hline
(1, 3, 5, 6, 7)
&(0, 3, 5, 6, 7)
\\ \hline
(1, 4, 5, 6, 7)
&(0, 2, 3, 5, 7)
\\ \hline
(1, 4, 5, 6, 7)
&(0, 3, 4, 6, 7)
\\ \hline
(2, 3, 4, 5, 6)
&(0, 1, 3, 5, 6)
\\ \hline
(2, 3, 4, 5, 6)
&(0, 2, 3, 5, 7)
\\ \hline
(2, 3, 4, 5, 6)
&(1, 2, 4, 6, 7)
\\ \hline
(2, 3, 4, 5, 7)
&(0, 1, 2, 4, 7)
\\ \hline
(2, 3, 4, 5, 7)
&(0, 3, 5, 6, 7)
\\ \hline
(2, 3, 4, 5, 7)
&(1, 2, 3, 4, 6)
\\ \hline
(2, 3, 4, 6, 7)
&(1, 3, 4, 5, 7)
\\ \hline
(2, 3, 5, 6, 7)
&(0, 2, 4, 5, 6)
\\ \hline
(2, 4, 5, 6, 7)
&(0, 1, 3, 4, 6)
\\ \hline
(2, 4, 5, 6, 7)
&(0, 3, 4, 5, 7)
\\ \hline
(3, 4, 5, 6, 7)
&(0, 1, 2, 5, 7)
\\ \hline
(3, 4, 5, 6, 7)
&(1, 2, 4, 5, 6)
\\ \hline
(0, 1, 2, 3, 4, 7)
&(0, 1, 3, 4, 6, 7)
\\ \hline
(0, 1, 2, 3, 4, 7)
&(0, 2, 3, 4, 5, 7)
\\ \hline
(0, 1, 2, 3, 5, 6)
&(0, 1, 2, 5, 6, 7)
\\ \hline
(0, 1, 2, 3, 5, 6)
&(1, 2, 3, 4, 5, 6)
\\ \hline
(0, 1, 2, 4, 5, 7)
&(0, 1, 2, 5, 6, 7)
\\ \hline
(0, 1, 2, 4, 5, 7)
&(0, 2, 3, 4, 5, 7)
\\ \hline
(0, 1, 2, 4, 6, 7)
&(0, 2, 3, 4, 5, 7)
\\ \hline
(0, 1, 2, 4, 6, 7)
&(1, 2, 3, 4, 5, 6)
\\ \hline
(0, 1, 2, 5, 6, 7)
&(0, 1, 2, 3, 5, 6)
\\ \hline
(0, 1, 2, 5, 6, 7)
&(0, 1, 2, 4, 5, 7)
\\ \hline
(0, 1, 2, 5, 6, 7)
&(0, 2, 3, 4, 5, 6)
\\ \hline
(0, 1, 2, 5, 6, 7)
&(0, 2, 3, 5, 6, 7)
\\ \hline
(0, 1, 2, 5, 6, 7)
&(1, 2, 3, 4, 5, 7)
\\ \hline
(0, 1, 2, 5, 6, 7)
&(1, 2, 4, 5, 6, 7)
\\ \hline
(0, 1, 3, 4, 5, 6)
&(0, 1, 3, 4, 6, 7)
\\ \hline
(0, 1, 3, 4, 5, 6)
&(1, 2, 3, 4, 5, 6)
\\ \hline
(0, 1, 3, 4, 6, 7)
&(0, 1, 2, 3, 4, 7)
\\ \hline
(0, 1, 3, 4, 6, 7)
&(0, 1, 3, 4, 5, 6)
\\ \hline
(0, 1, 3, 4, 6, 7)
&(0, 2, 3, 4, 5, 6)
\\ \hline
(0, 1, 3, 4, 6, 7)
&(0, 3, 4, 5, 6, 7)
\\ \hline
(0, 1, 3, 4, 6, 7)
&(1, 2, 3, 4, 5, 7)
\\ \hline
(0, 1, 3, 4, 6, 7)
&(1, 2, 3, 4, 6, 7)
\\ \hline
(0, 1, 3, 5, 6, 7)
&(0, 2, 3, 4, 5, 7)
\\ \hline
(0, 1, 3, 5, 6, 7)
&(1, 2, 3, 4, 5, 6)
\\ \hline
(0, 2, 3, 4, 5, 6)
&(0, 1, 2, 5, 6, 7)
\\ \hline
(0, 2, 3, 4, 5, 6)
&(0, 1, 3, 4, 6, 7)
\\ \hline
(0, 2, 3, 4, 5, 7)
&(0, 1, 2, 3, 4, 7)
\\ \hline
(0, 2, 3, 4, 5, 7)
&(0, 1, 2, 4, 5, 7)
\\ \hline
(0, 2, 3, 4, 5, 7)
&(0, 1, 2, 4, 6, 7)
\\ \hline
(0, 2, 3, 4, 5, 7)
&(0, 1, 3, 5, 6, 7)
\\ \hline
(0, 2, 3, 4, 5, 7)
&(0, 2, 3, 5, 6, 7)
\\ \hline
(0, 2, 3, 4, 5, 7)
&(0, 3, 4, 5, 6, 7)
\\ \hline
(0, 2, 3, 5, 6, 7)
&(0, 1, 2, 5, 6, 7)
\\ \hline
(0, 2, 3, 5, 6, 7)
&(0, 2, 3, 4, 5, 7)
\\ \hline
(0, 3, 4, 5, 6, 7)
&(0, 1, 3, 4, 6, 7)
\\ \hline
(0, 3, 4, 5, 6, 7)
&(0, 2, 3, 4, 5, 7)
\\ \hline
(1, 2, 3, 4, 5, 6)
&(0, 1, 2, 3, 5, 6)
\\ \hline
(1, 2, 3, 4, 5, 6)
&(0, 1, 2, 4, 6, 7)
\\ \hline
(1, 2, 3, 4, 5, 6)
&(0, 1, 3, 4, 5, 6)
\\ \hline
(1, 2, 3, 4, 5, 6)
&(0, 1, 3, 5, 6, 7)
\\ \hline
(1, 2, 3, 4, 5, 6)
&(1, 2, 3, 4, 6, 7)
\\ \hline
(1, 2, 3, 4, 5, 6)
&(1, 2, 4, 5, 6, 7)
\\ \hline
(1, 2, 3, 4, 5, 7)
&(0, 1, 2, 5, 6, 7)
\\ \hline
(1, 2, 3, 4, 5, 7)
&(0, 1, 3, 4, 6, 7)
\\ \hline
(1, 2, 3, 4, 6, 7)
&(0, 1, 3, 4, 6, 7)
\\ \hline
(1, 2, 3, 4, 6, 7)
&(1, 2, 3, 4, 5, 6)
\\ \hline
(1, 2, 4, 5, 6, 7)
&(0, 1, 2, 5, 6, 7)
\\ \hline
(1, 2, 4, 5, 6, 7)
&(1, 2, 3, 4, 5, 6)
\\ \hline
\caption{Whirlwind $m_0$ (\ref{mat:whirlwind-m0}) singular submatrices}\label{tbl:m0-singular}
\end{longtable}
\end{footnotesize}


For (\ref{mat:whirlwind-m1}), an example of singular submatrix is (\ref{mat:singular-whirlwind-m1}), and the list of rows and columns to be removed to yield all singular submatrices is given by Table \ref{tbl:m1-singular}.

\begin{equation}\label{mat:singular-whirlwind-m1}
\begin{bmatrix}
8_x & 7_x\\
3_x & e_x
\end{bmatrix}
=
\begin{bmatrix}
x^3 & x^2 + x + 1\\
x + 1 & x^3 + x^2 + x
\end{bmatrix}
\end{equation}

\begin{footnotesize}
\begin{longtable}[c]{|l|l|}
\hline
\textbf{Rows to be removed} & \textbf{Columns to be removed} \\ \hline
%\endfirsthead
\endhead
(0, 2)
&(1, 5)
\\ \hline
(0, 2)
&(3, 7)
\\ \hline
(0, 4)
&(0, 5)
\\ \hline
(0, 4)
&(0, 7)
\\ \hline
(0, 4)
&(1, 3)
\\ \hline
(0, 4)
&(1, 4)
\\ \hline
(0, 4)
&(3, 4)
\\ \hline
(0, 4)
&(5, 7)
\\ \hline
(0, 5)
&(0, 4)
\\ \hline
(0, 5)
&(1, 5)
\\ \hline
(0, 7)
&(0, 4)
\\ \hline
(0, 7)
&(3, 7)
\\ \hline
(1, 3)
&(0, 4)
\\ \hline
(1, 3)
&(2, 6)
\\ \hline
(1, 4)
&(0, 4)
\\ \hline
(1, 4)
&(1, 5)
\\ \hline
(1, 5)
&(0, 2)
\\ \hline
(1, 5)
&(0, 5)
\\ \hline
(1, 5)
&(1, 4)
\\ \hline
(1, 5)
&(1, 6)
\\ \hline
(1, 5)
&(2, 5)
\\ \hline
(1, 5)
&(4, 6)
\\ \hline
(1, 6)
&(1, 5)
\\ \hline
(1, 6)
&(2, 6)
\\ \hline
(2, 5)
&(1, 5)
\\ \hline
(2, 5)
&(2, 6)
\\ \hline
(2, 6)
&(1, 3)
\\ \hline
(2, 6)
&(1, 6)
\\ \hline
(2, 6)
&(2, 5)
\\ \hline
(2, 6)
&(2, 7)
\\ \hline
(2, 6)
&(3, 6)
\\ \hline
(2, 6)
&(5, 7)
\\ \hline
(2, 7)
&(2, 6)
\\ \hline
(2, 7)
&(3, 7)
\\ \hline
(3, 4)
&(0, 4)
\\ \hline
(3, 4)
&(3, 7)
\\ \hline
(3, 6)
&(2, 6)
\\ \hline
(3, 6)
&(3, 7)
\\ \hline
(3, 7)
&(0, 2)
\\ \hline
(3, 7)
&(0, 7)
\\ \hline
(3, 7)
&(2, 7)
\\ \hline
(3, 7)
&(3, 4)
\\ \hline
(3, 7)
&(3, 6)
\\ \hline
(3, 7)
&(4, 6)
\\ \hline
(4, 6)
&(1, 5)
\\ \hline
(4, 6)
&(3, 7)
\\ \hline
(5, 7)
&(0, 4)
\\ \hline
(5, 7)
&(2, 6)
\\ \hline
(0, 1, 2)
&(0, 3, 5)
\\ \hline
(0, 1, 2)
&(0, 3, 6)
\\ \hline
(0, 1, 2)
&(0, 5, 6)
\\ \hline
(0, 1, 2)
&(1, 2, 4)
\\ \hline
(0, 1, 2)
&(1, 2, 7)
\\ \hline
(0, 1, 2)
&(1, 4, 7)
\\ \hline
(0, 1, 2)
&(2, 4, 7)
\\ \hline
(0, 1, 2)
&(3, 5, 6)
\\ \hline
(0, 1, 3)
&(0, 3, 5)
\\ \hline
(0, 1, 3)
&(0, 3, 6)
\\ \hline
(0, 1, 3)
&(0, 5, 6)
\\ \hline
(0, 1, 3)
&(1, 2, 4)
\\ \hline
(0, 1, 3)
&(1, 2, 7)
\\ \hline
(0, 1, 3)
&(1, 4, 7)
\\ \hline
(0, 1, 3)
&(2, 4, 7)
\\ \hline
(0, 1, 3)
&(3, 5, 6)
\\ \hline
(0, 1, 4)
&(0, 5, 7)
\\ \hline
(0, 1, 4)
&(1, 3, 4)
\\ \hline
(0, 1, 5)
&(0, 2, 5)
\\ \hline
(0, 1, 5)
&(1, 4, 6)
\\ \hline
(0, 1, 6)
&(3, 5, 7)
\\ \hline
(0, 1, 7)
&(2, 4, 6)
\\ \hline
(0, 2, 3)
&(0, 3, 5)
\\ \hline
(0, 2, 3)
&(0, 3, 6)
\\ \hline
(0, 2, 3)
&(0, 5, 6)
\\ \hline
(0, 2, 3)
&(1, 2, 4)
\\ \hline
(0, 2, 3)
&(1, 2, 7)
\\ \hline
(0, 2, 3)
&(1, 4, 7)
\\ \hline
(0, 2, 3)
&(2, 4, 7)
\\ \hline
(0, 2, 3)
&(3, 5, 6)
\\ \hline
(0, 2, 4)
&(0, 5, 7)
\\ \hline
(0, 2, 4)
&(1, 3, 4)
\\ \hline
(0, 2, 4)
&(1, 6, 7)
\\ \hline
(0, 2, 4)
&(2, 5, 6)
\\ \hline
(0, 2, 5)
&(0, 1, 5)
\\ \hline
(0, 2, 5)
&(1, 2, 5)
\\ \hline
(0, 2, 5)
&(1, 3, 5)
\\ \hline
(0, 2, 5)
&(1, 4, 5)
\\ \hline
(0, 2, 5)
&(1, 5, 6)
\\ \hline
(0, 2, 5)
&(1, 5, 7)
\\ \hline
(0, 2, 6)
&(0, 4, 7)
\\ \hline
(0, 2, 6)
&(1, 3, 6)
\\ \hline
(0, 2, 6)
&(2, 5, 7)
\\ \hline
(0, 2, 6)
&(3, 4, 5)
\\ \hline
(0, 2, 7)
&(0, 3, 7)
\\ \hline
(0, 2, 7)
&(1, 3, 7)
\\ \hline
(0, 2, 7)
&(2, 3, 7)
\\ \hline
(0, 2, 7)
&(3, 4, 7)
\\ \hline
(0, 2, 7)
&(3, 5, 7)
\\ \hline
(0, 2, 7)
&(3, 6, 7)
\\ \hline
(0, 3, 4)
&(0, 5, 7)
\\ \hline
(0, 3, 4)
&(1, 3, 4)
\\ \hline
(0, 3, 4)
&(2, 4, 6)
\\ \hline
(0, 3, 5)
&(0, 1, 2)
\\ \hline
(0, 3, 5)
&(0, 1, 3)
\\ \hline
(0, 3, 5)
&(0, 2, 3)
\\ \hline
(0, 3, 5)
&(1, 2, 3)
\\ \hline
(0, 3, 5)
&(4, 5, 6)
\\ \hline
(0, 3, 5)
&(4, 5, 7)
\\ \hline
(0, 3, 5)
&(4, 6, 7)
\\ \hline
(0, 3, 5)
&(5, 6, 7)
\\ \hline
(0, 3, 6)
&(0, 1, 2)
\\ \hline
(0, 3, 6)
&(0, 1, 3)
\\ \hline
(0, 3, 6)
&(0, 2, 3)
\\ \hline
(0, 3, 6)
&(1, 2, 3)
\\ \hline
(0, 3, 6)
&(4, 5, 6)
\\ \hline
(0, 3, 6)
&(4, 5, 7)
\\ \hline
(0, 3, 6)
&(4, 6, 7)
\\ \hline
(0, 3, 6)
&(5, 6, 7)
\\ \hline
(0, 3, 7)
&(0, 2, 7)
\\ \hline
(0, 3, 7)
&(1, 5, 7)
\\ \hline
(0, 3, 7)
&(3, 4, 6)
\\ \hline
(0, 4, 5)
&(0, 5, 7)
\\ \hline
(0, 4, 5)
&(1, 3, 4)
\\ \hline
(0, 4, 6)
&(0, 5, 7)
\\ \hline
(0, 4, 6)
&(1, 2, 6)
\\ \hline
(0, 4, 6)
&(1, 3, 4)
\\ \hline
(0, 4, 6)
&(2, 3, 5)
\\ \hline
(0, 4, 7)
&(0, 2, 6)
\\ \hline
(0, 4, 7)
&(0, 5, 7)
\\ \hline
(0, 4, 7)
&(1, 3, 4)
\\ \hline
(0, 5, 6)
&(0, 1, 2)
\\ \hline
(0, 5, 6)
&(0, 1, 3)
\\ \hline
(0, 5, 6)
&(0, 2, 3)
\\ \hline
(0, 5, 6)
&(1, 2, 3)
\\ \hline
(0, 5, 6)
&(4, 5, 6)
\\ \hline
(0, 5, 6)
&(4, 5, 7)
\\ \hline
(0, 5, 6)
&(4, 6, 7)
\\ \hline
(0, 5, 6)
&(5, 6, 7)
\\ \hline
(0, 5, 7)
&(0, 1, 4)
\\ \hline
(0, 5, 7)
&(0, 2, 4)
\\ \hline
(0, 5, 7)
&(0, 3, 4)
\\ \hline
(0, 5, 7)
&(0, 4, 5)
\\ \hline
(0, 5, 7)
&(0, 4, 6)
\\ \hline
(0, 5, 7)
&(0, 4, 7)
\\ \hline
(0, 6, 7)
&(1, 3, 5)
\\ \hline
(1, 2, 3)
&(0, 3, 5)
\\ \hline
(1, 2, 3)
&(0, 3, 6)
\\ \hline
(1, 2, 3)
&(0, 5, 6)
\\ \hline
(1, 2, 3)
&(1, 2, 4)
\\ \hline
(1, 2, 3)
&(1, 2, 7)
\\ \hline
(1, 2, 3)
&(1, 4, 7)
\\ \hline
(1, 2, 3)
&(2, 4, 7)
\\ \hline
(1, 2, 3)
&(3, 5, 6)
\\ \hline
(1, 2, 4)
&(0, 1, 2)
\\ \hline
(1, 2, 4)
&(0, 1, 3)
\\ \hline
(1, 2, 4)
&(0, 2, 3)
\\ \hline
(1, 2, 4)
&(1, 2, 3)
\\ \hline
(1, 2, 4)
&(4, 5, 6)
\\ \hline
(1, 2, 4)
&(4, 5, 7)
\\ \hline
(1, 2, 4)
&(4, 6, 7)
\\ \hline
(1, 2, 4)
&(5, 6, 7)
\\ \hline
(1, 2, 5)
&(0, 2, 5)
\\ \hline
(1, 2, 5)
&(1, 4, 6)
\\ \hline
(1, 2, 5)
&(3, 5, 7)
\\ \hline
(1, 2, 6)
&(0, 4, 6)
\\ \hline
(1, 2, 6)
&(1, 3, 6)
\\ \hline
(1, 2, 6)
&(2, 5, 7)
\\ \hline
(1, 2, 7)
&(0, 1, 2)
\\ \hline
(1, 2, 7)
&(0, 1, 3)
\\ \hline
(1, 2, 7)
&(0, 2, 3)
\\ \hline
(1, 2, 7)
&(1, 2, 3)
\\ \hline
(1, 2, 7)
&(4, 5, 6)
\\ \hline
(1, 2, 7)
&(4, 5, 7)
\\ \hline
(1, 2, 7)
&(4, 6, 7)
\\ \hline
(1, 2, 7)
&(5, 6, 7)
\\ \hline
(1, 3, 4)
&(0, 1, 4)
\\ \hline
(1, 3, 4)
&(0, 2, 4)
\\ \hline
(1, 3, 4)
&(0, 3, 4)
\\ \hline
(1, 3, 4)
&(0, 4, 5)
\\ \hline
(1, 3, 4)
&(0, 4, 6)
\\ \hline
(1, 3, 4)
&(0, 4, 7)
\\ \hline
(1, 3, 5)
&(0, 2, 5)
\\ \hline
(1, 3, 5)
&(0, 6, 7)
\\ \hline
(1, 3, 5)
&(1, 4, 6)
\\ \hline
(1, 3, 5)
&(3, 4, 7)
\\ \hline
(1, 3, 6)
&(0, 2, 6)
\\ \hline
(1, 3, 6)
&(1, 2, 6)
\\ \hline
(1, 3, 6)
&(2, 3, 6)
\\ \hline
(1, 3, 6)
&(2, 4, 6)
\\ \hline
(1, 3, 6)
&(2, 5, 6)
\\ \hline
(1, 3, 6)
&(2, 6, 7)
\\ \hline
(1, 3, 7)
&(0, 2, 7)
\\ \hline
(1, 3, 7)
&(1, 5, 6)
\\ \hline
(1, 3, 7)
&(2, 4, 5)
\\ \hline
(1, 3, 7)
&(3, 4, 6)
\\ \hline
(1, 4, 5)
&(0, 2, 5)
\\ \hline
(1, 4, 5)
&(1, 4, 6)
\\ \hline
(1, 4, 6)
&(0, 1, 5)
\\ \hline
(1, 4, 6)
&(1, 2, 5)
\\ \hline
(1, 4, 6)
&(1, 3, 5)
\\ \hline
(1, 4, 6)
&(1, 4, 5)
\\ \hline
(1, 4, 6)
&(1, 5, 6)
\\ \hline
(1, 4, 6)
&(1, 5, 7)
\\ \hline
(1, 4, 7)
&(0, 1, 2)
\\ \hline
(1, 4, 7)
&(0, 1, 3)
\\ \hline
(1, 4, 7)
&(0, 2, 3)
\\ \hline
(1, 4, 7)
&(1, 2, 3)
\\ \hline
(1, 4, 7)
&(4, 5, 6)
\\ \hline
(1, 4, 7)
&(4, 5, 7)
\\ \hline
(1, 4, 7)
&(4, 6, 7)
\\ \hline
(1, 4, 7)
&(5, 6, 7)
\\ \hline
(1, 5, 6)
&(0, 2, 5)
\\ \hline
(1, 5, 6)
&(1, 3, 7)
\\ \hline
(1, 5, 6)
&(1, 4, 6)
\\ \hline
(1, 5, 7)
&(0, 2, 5)
\\ \hline
(1, 5, 7)
&(0, 3, 7)
\\ \hline
(1, 5, 7)
&(1, 4, 6)
\\ \hline
(1, 5, 7)
&(2, 3, 4)
\\ \hline
(1, 6, 7)
&(0, 2, 4)
\\ \hline
(2, 3, 4)
&(1, 5, 7)
\\ \hline
(2, 3, 5)
&(0, 4, 6)
\\ \hline
(2, 3, 6)
&(1, 3, 6)
\\ \hline
(2, 3, 6)
&(2, 5, 7)
\\ \hline
(2, 3, 7)
&(0, 2, 7)
\\ \hline
(2, 3, 7)
&(3, 4, 6)
\\ \hline
(2, 4, 5)
&(1, 3, 7)
\\ \hline
(2, 4, 6)
&(0, 1, 7)
\\ \hline
(2, 4, 6)
&(0, 3, 4)
\\ \hline
(2, 4, 6)
&(1, 3, 6)
\\ \hline
(2, 4, 6)
&(2, 5, 7)
\\ \hline
(2, 4, 7)
&(0, 1, 2)
\\ \hline
(2, 4, 7)
&(0, 1, 3)
\\ \hline
(2, 4, 7)
&(0, 2, 3)
\\ \hline
(2, 4, 7)
&(1, 2, 3)
\\ \hline
(2, 4, 7)
&(4, 5, 6)
\\ \hline
(2, 4, 7)
&(4, 5, 7)
\\ \hline
(2, 4, 7)
&(4, 6, 7)
\\ \hline
(2, 4, 7)
&(5, 6, 7)
\\ \hline
(2, 5, 6)
&(0, 2, 4)
\\ \hline
(2, 5, 6)
&(1, 3, 6)
\\ \hline
(2, 5, 6)
&(2, 5, 7)
\\ \hline
(2, 5, 7)
&(0, 2, 6)
\\ \hline
(2, 5, 7)
&(1, 2, 6)
\\ \hline
(2, 5, 7)
&(2, 3, 6)
\\ \hline
(2, 5, 7)
&(2, 4, 6)
\\ \hline
(2, 5, 7)
&(2, 5, 6)
\\ \hline
(2, 5, 7)
&(2, 6, 7)
\\ \hline
(2, 6, 7)
&(1, 3, 6)
\\ \hline
(2, 6, 7)
&(2, 5, 7)
\\ \hline
(3, 4, 5)
&(0, 2, 6)
\\ \hline
(3, 4, 6)
&(0, 3, 7)
\\ \hline
(3, 4, 6)
&(1, 3, 7)
\\ \hline
(3, 4, 6)
&(2, 3, 7)
\\ \hline
(3, 4, 6)
&(3, 4, 7)
\\ \hline
(3, 4, 6)
&(3, 5, 7)
\\ \hline
(3, 4, 6)
&(3, 6, 7)
\\ \hline
(3, 4, 7)
&(0, 2, 7)
\\ \hline
(3, 4, 7)
&(1, 3, 5)
\\ \hline
(3, 4, 7)
&(3, 4, 6)
\\ \hline
(3, 5, 6)
&(0, 1, 2)
\\ \hline
(3, 5, 6)
&(0, 1, 3)
\\ \hline
(3, 5, 6)
&(0, 2, 3)
\\ \hline
(3, 5, 6)
&(1, 2, 3)
\\ \hline
(3, 5, 6)
&(4, 5, 6)
\\ \hline
(3, 5, 6)
&(4, 5, 7)
\\ \hline
(3, 5, 6)
&(4, 6, 7)
\\ \hline
(3, 5, 6)
&(5, 6, 7)
\\ \hline
(3, 5, 7)
&(0, 1, 6)
\\ \hline
(3, 5, 7)
&(0, 2, 7)
\\ \hline
(3, 5, 7)
&(1, 2, 5)
\\ \hline
(3, 5, 7)
&(3, 4, 6)
\\ \hline
(3, 6, 7)
&(0, 2, 7)
\\ \hline
(3, 6, 7)
&(3, 4, 6)
\\ \hline
(4, 5, 6)
&(0, 3, 5)
\\ \hline
(4, 5, 6)
&(0, 3, 6)
\\ \hline
(4, 5, 6)
&(0, 5, 6)
\\ \hline
(4, 5, 6)
&(1, 2, 4)
\\ \hline
(4, 5, 6)
&(1, 2, 7)
\\ \hline
(4, 5, 6)
&(1, 4, 7)
\\ \hline
(4, 5, 6)
&(2, 4, 7)
\\ \hline
(4, 5, 6)
&(3, 5, 6)
\\ \hline
(4, 5, 7)
&(0, 3, 5)
\\ \hline
(4, 5, 7)
&(0, 3, 6)
\\ \hline
(4, 5, 7)
&(0, 5, 6)
\\ \hline
(4, 5, 7)
&(1, 2, 4)
\\ \hline
(4, 5, 7)
&(1, 2, 7)
\\ \hline
(4, 5, 7)
&(1, 4, 7)
\\ \hline
(4, 5, 7)
&(2, 4, 7)
\\ \hline
(4, 5, 7)
&(3, 5, 6)
\\ \hline
(4, 6, 7)
&(0, 3, 5)
\\ \hline
(4, 6, 7)
&(0, 3, 6)
\\ \hline
(4, 6, 7)
&(0, 5, 6)
\\ \hline
(4, 6, 7)
&(1, 2, 4)
\\ \hline
(4, 6, 7)
&(1, 2, 7)
\\ \hline
(4, 6, 7)
&(1, 4, 7)
\\ \hline
(4, 6, 7)
&(2, 4, 7)
\\ \hline
(4, 6, 7)
&(3, 5, 6)
\\ \hline
(5, 6, 7)
&(0, 3, 5)
\\ \hline
(5, 6, 7)
&(0, 3, 6)
\\ \hline
(5, 6, 7)
&(0, 5, 6)
\\ \hline
(5, 6, 7)
&(1, 2, 4)
\\ \hline
(5, 6, 7)
&(1, 2, 7)
\\ \hline
(5, 6, 7)
&(1, 4, 7)
\\ \hline
(5, 6, 7)
&(2, 4, 7)
\\ \hline
(5, 6, 7)
&(3, 5, 6)
\\ \hline
(0, 1, 2, 3)
&(0, 1, 2, 4)
\\ \hline
(0, 1, 2, 3)
&(0, 1, 2, 7)
\\ \hline
(0, 1, 2, 3)
&(0, 1, 3, 5)
\\ \hline
(0, 1, 2, 3)
&(0, 1, 3, 6)
\\ \hline
(0, 1, 2, 3)
&(0, 1, 4, 7)
\\ \hline
(0, 1, 2, 3)
&(0, 1, 5, 6)
\\ \hline
(0, 1, 2, 3)
&(0, 2, 3, 5)
\\ \hline
(0, 1, 2, 3)
&(0, 2, 3, 6)
\\ \hline
(0, 1, 2, 3)
&(0, 2, 4, 7)
\\ \hline
(0, 1, 2, 3)
&(0, 2, 5, 6)
\\ \hline
(0, 1, 2, 3)
&(0, 3, 4, 5)
\\ \hline
(0, 1, 2, 3)
&(0, 3, 4, 6)
\\ \hline
(0, 1, 2, 3)
&(0, 3, 5, 6)
\\ \hline
(0, 1, 2, 3)
&(0, 3, 5, 7)
\\ \hline
(0, 1, 2, 3)
&(0, 3, 6, 7)
\\ \hline
(0, 1, 2, 3)
&(0, 4, 5, 6)
\\ \hline
(0, 1, 2, 3)
&(0, 5, 6, 7)
\\ \hline
(0, 1, 2, 3)
&(1, 2, 3, 4)
\\ \hline
(0, 1, 2, 3)
&(1, 2, 3, 7)
\\ \hline
(0, 1, 2, 3)
&(1, 2, 4, 5)
\\ \hline
(0, 1, 2, 3)
&(1, 2, 4, 6)
\\ \hline
(0, 1, 2, 3)
&(1, 2, 4, 7)
\\ \hline
(0, 1, 2, 3)
&(1, 2, 5, 7)
\\ \hline
(0, 1, 2, 3)
&(1, 2, 6, 7)
\\ \hline
(0, 1, 2, 3)
&(1, 3, 4, 7)
\\ \hline
(0, 1, 2, 3)
&(1, 3, 5, 6)
\\ \hline
(0, 1, 2, 3)
&(1, 4, 5, 7)
\\ \hline
(0, 1, 2, 3)
&(1, 4, 6, 7)
\\ \hline
(0, 1, 2, 3)
&(2, 3, 4, 7)
\\ \hline
(0, 1, 2, 3)
&(2, 3, 5, 6)
\\ \hline
(0, 1, 2, 3)
&(2, 4, 5, 7)
\\ \hline
(0, 1, 2, 3)
&(2, 4, 6, 7)
\\ \hline
(0, 1, 2, 3)
&(3, 4, 5, 6)
\\ \hline
(0, 1, 2, 3)
&(3, 5, 6, 7)
\\ \hline
(0, 1, 2, 4)
&(0, 1, 2, 3)
\\ \hline
(0, 1, 2, 4)
&(0, 1, 4, 5)
\\ \hline
(0, 1, 2, 4)
&(0, 2, 4, 6)
\\ \hline
(0, 1, 2, 4)
&(0, 3, 5, 6)
\\ \hline
(0, 1, 2, 4)
&(1, 2, 4, 7)
\\ \hline
(0, 1, 2, 4)
&(1, 3, 5, 7)
\\ \hline
(0, 1, 2, 4)
&(2, 3, 6, 7)
\\ \hline
(0, 1, 2, 4)
&(4, 5, 6, 7)
\\ \hline
(0, 1, 2, 5)
&(0, 3, 5, 6)
\\ \hline
(0, 1, 2, 5)
&(1, 2, 4, 7)
\\ \hline
(0, 1, 2, 5)
&(2, 5, 6, 7)
\\ \hline
(0, 1, 2, 6)
&(0, 1, 2, 6)
\\ \hline
(0, 1, 2, 6)
&(0, 3, 5, 6)
\\ \hline
(0, 1, 2, 6)
&(1, 2, 4, 7)
\\ \hline
(0, 1, 2, 7)
&(0, 1, 2, 3)
\\ \hline
(0, 1, 2, 7)
&(0, 1, 6, 7)
\\ \hline
(0, 1, 2, 7)
&(0, 2, 5, 7)
\\ \hline
(0, 1, 2, 7)
&(0, 3, 5, 6)
\\ \hline
(0, 1, 2, 7)
&(1, 2, 4, 7)
\\ \hline
(0, 1, 2, 7)
&(1, 3, 4, 6)
\\ \hline
(0, 1, 2, 7)
&(2, 3, 4, 5)
\\ \hline
(0, 1, 2, 7)
&(4, 5, 6, 7)
\\ \hline
(0, 1, 3, 4)
&(0, 3, 5, 6)
\\ \hline
(0, 1, 3, 4)
&(1, 2, 4, 7)
\\ \hline
(0, 1, 3, 4)
&(3, 4, 6, 7)
\\ \hline
(0, 1, 3, 5)
&(0, 1, 2, 3)
\\ \hline
(0, 1, 3, 5)
&(0, 1, 4, 5)
\\ \hline
(0, 1, 3, 5)
&(0, 2, 4, 6)
\\ \hline
(0, 1, 3, 5)
&(0, 3, 5, 6)
\\ \hline
(0, 1, 3, 5)
&(1, 2, 4, 7)
\\ \hline
(0, 1, 3, 5)
&(1, 3, 5, 7)
\\ \hline
(0, 1, 3, 5)
&(2, 3, 6, 7)
\\ \hline
(0, 1, 3, 5)
&(4, 5, 6, 7)
\\ \hline
(0, 1, 3, 6)
&(0, 1, 2, 3)
\\ \hline
(0, 1, 3, 6)
&(0, 1, 6, 7)
\\ \hline
(0, 1, 3, 6)
&(0, 2, 5, 7)
\\ \hline
(0, 1, 3, 6)
&(0, 3, 5, 6)
\\ \hline
(0, 1, 3, 6)
&(1, 2, 4, 7)
\\ \hline
(0, 1, 3, 6)
&(1, 3, 4, 6)
\\ \hline
(0, 1, 3, 6)
&(2, 3, 4, 5)
\\ \hline
(0, 1, 3, 6)
&(4, 5, 6, 7)
\\ \hline
(0, 1, 3, 7)
&(0, 1, 3, 7)
\\ \hline
(0, 1, 3, 7)
&(0, 3, 5, 6)
\\ \hline
(0, 1, 3, 7)
&(1, 2, 4, 7)
\\ \hline
(0, 1, 4, 5)
&(0, 1, 2, 4)
\\ \hline
(0, 1, 4, 5)
&(0, 1, 3, 5)
\\ \hline
(0, 1, 4, 5)
&(0, 1, 6, 7)
\\ \hline
(0, 1, 4, 5)
&(0, 2, 3, 6)
\\ \hline
(0, 1, 4, 5)
&(0, 4, 5, 6)
\\ \hline
(0, 1, 4, 5)
&(1, 2, 3, 7)
\\ \hline
(0, 1, 4, 5)
&(1, 4, 5, 7)
\\ \hline
(0, 1, 4, 5)
&(2, 3, 4, 5)
\\ \hline
(0, 1, 4, 5)
&(2, 4, 6, 7)
\\ \hline
(0, 1, 4, 5)
&(3, 5, 6, 7)
\\ \hline
(0, 1, 4, 6)
&(0, 2, 4, 7)
\\ \hline
(0, 1, 4, 6)
&(0, 2, 5, 6)
\\ \hline
(0, 1, 4, 7)
&(0, 1, 2, 3)
\\ \hline
(0, 1, 4, 7)
&(4, 5, 6, 7)
\\ \hline
(0, 1, 5, 6)
&(0, 1, 2, 3)
\\ \hline
(0, 1, 5, 6)
&(4, 5, 6, 7)
\\ \hline
(0, 1, 5, 7)
&(1, 3, 4, 7)
\\ \hline
(0, 1, 5, 7)
&(1, 3, 5, 6)
\\ \hline
(0, 1, 6, 7)
&(0, 1, 2, 7)
\\ \hline
(0, 1, 6, 7)
&(0, 1, 3, 6)
\\ \hline
(0, 1, 6, 7)
&(0, 1, 4, 5)
\\ \hline
(0, 1, 6, 7)
&(0, 2, 3, 5)
\\ \hline
(0, 1, 6, 7)
&(0, 5, 6, 7)
\\ \hline
(0, 1, 6, 7)
&(1, 2, 3, 4)
\\ \hline
(0, 1, 6, 7)
&(1, 4, 6, 7)
\\ \hline
(0, 1, 6, 7)
&(2, 3, 6, 7)
\\ \hline
(0, 1, 6, 7)
&(2, 4, 5, 7)
\\ \hline
(0, 1, 6, 7)
&(3, 4, 5, 6)
\\ \hline
(0, 2, 3, 4)
&(0, 2, 3, 4)
\\ \hline
(0, 2, 3, 4)
&(0, 3, 5, 6)
\\ \hline
(0, 2, 3, 4)
&(1, 2, 4, 7)
\\ \hline
(0, 2, 3, 5)
&(0, 1, 2, 3)
\\ \hline
(0, 2, 3, 5)
&(0, 1, 6, 7)
\\ \hline
(0, 2, 3, 5)
&(0, 2, 5, 7)
\\ \hline
(0, 2, 3, 5)
&(0, 3, 5, 6)
\\ \hline
(0, 2, 3, 5)
&(1, 2, 4, 7)
\\ \hline
(0, 2, 3, 5)
&(1, 3, 4, 6)
\\ \hline
(0, 2, 3, 5)
&(2, 3, 4, 5)
\\ \hline
(0, 2, 3, 5)
&(4, 5, 6, 7)
\\ \hline
(0, 2, 3, 6)
&(0, 1, 2, 3)
\\ \hline
(0, 2, 3, 6)
&(0, 1, 4, 5)
\\ \hline
(0, 2, 3, 6)
&(0, 2, 4, 6)
\\ \hline
(0, 2, 3, 6)
&(0, 3, 5, 6)
\\ \hline
(0, 2, 3, 6)
&(1, 2, 4, 7)
\\ \hline
(0, 2, 3, 6)
&(1, 3, 5, 7)
\\ \hline
(0, 2, 3, 6)
&(2, 3, 6, 7)
\\ \hline
(0, 2, 3, 6)
&(4, 5, 6, 7)
\\ \hline
(0, 2, 3, 7)
&(0, 3, 5, 6)
\\ \hline
(0, 2, 3, 7)
&(0, 4, 5, 7)
\\ \hline
(0, 2, 3, 7)
&(1, 2, 4, 7)
\\ \hline
(0, 2, 4, 5)
&(0, 3, 4, 6)
\\ \hline
(0, 2, 4, 5)
&(1, 2, 4, 6)
\\ \hline
(0, 2, 4, 6)
&(0, 1, 2, 4)
\\ \hline
(0, 2, 4, 6)
&(0, 1, 3, 5)
\\ \hline
(0, 2, 4, 6)
&(0, 2, 3, 6)
\\ \hline
(0, 2, 4, 6)
&(0, 4, 5, 6)
\\ \hline
(0, 2, 4, 6)
&(1, 2, 3, 7)
\\ \hline
(0, 2, 4, 6)
&(1, 4, 5, 7)
\\ \hline
(0, 2, 4, 6)
&(2, 4, 6, 7)
\\ \hline
(0, 2, 4, 6)
&(3, 5, 6, 7)
\\ \hline
(0, 2, 4, 7)
&(0, 1, 2, 3)
\\ \hline
(0, 2, 4, 7)
&(0, 1, 4, 6)
\\ \hline
(0, 2, 4, 7)
&(2, 3, 4, 6)
\\ \hline
(0, 2, 4, 7)
&(4, 5, 6, 7)
\\ \hline
(0, 2, 5, 6)
&(0, 1, 2, 3)
\\ \hline
(0, 2, 5, 6)
&(0, 1, 4, 6)
\\ \hline
(0, 2, 5, 6)
&(2, 3, 4, 6)
\\ \hline
(0, 2, 5, 6)
&(4, 5, 6, 7)
\\ \hline
(0, 2, 5, 7)
&(0, 1, 2, 7)
\\ \hline
(0, 2, 5, 7)
&(0, 1, 3, 6)
\\ \hline
(0, 2, 5, 7)
&(0, 2, 3, 5)
\\ \hline
(0, 2, 5, 7)
&(0, 5, 6, 7)
\\ \hline
(0, 2, 5, 7)
&(1, 2, 3, 4)
\\ \hline
(0, 2, 5, 7)
&(1, 4, 6, 7)
\\ \hline
(0, 2, 5, 7)
&(2, 4, 5, 7)
\\ \hline
(0, 2, 5, 7)
&(3, 4, 5, 6)
\\ \hline
(0, 2, 6, 7)
&(0, 3, 4, 6)
\\ \hline
(0, 2, 6, 7)
&(1, 2, 4, 6)
\\ \hline
(0, 3, 4, 5)
&(0, 1, 2, 3)
\\ \hline
(0, 3, 4, 5)
&(4, 5, 6, 7)
\\ \hline
(0, 3, 4, 6)
&(0, 1, 2, 3)
\\ \hline
(0, 3, 4, 6)
&(0, 2, 4, 5)
\\ \hline
(0, 3, 4, 6)
&(0, 2, 6, 7)
\\ \hline
(0, 3, 4, 6)
&(4, 5, 6, 7)
\\ \hline
(0, 3, 5, 6)
&(0, 1, 2, 3)
\\ \hline
(0, 3, 5, 6)
&(0, 1, 2, 4)
\\ \hline
(0, 3, 5, 6)
&(0, 1, 2, 5)
\\ \hline
(0, 3, 5, 6)
&(0, 1, 2, 6)
\\ \hline
(0, 3, 5, 6)
&(0, 1, 2, 7)
\\ \hline
(0, 3, 5, 6)
&(0, 1, 3, 4)
\\ \hline
(0, 3, 5, 6)
&(0, 1, 3, 5)
\\ \hline
(0, 3, 5, 6)
&(0, 1, 3, 6)
\\ \hline
(0, 3, 5, 6)
&(0, 1, 3, 7)
\\ \hline
(0, 3, 5, 6)
&(0, 2, 3, 4)
\\ \hline
(0, 3, 5, 6)
&(0, 2, 3, 5)
\\ \hline
(0, 3, 5, 6)
&(0, 2, 3, 6)
\\ \hline
(0, 3, 5, 6)
&(0, 2, 3, 7)
\\ \hline
(0, 3, 5, 6)
&(0, 4, 5, 6)
\\ \hline
(0, 3, 5, 6)
&(0, 4, 5, 7)
\\ \hline
(0, 3, 5, 6)
&(0, 4, 6, 7)
\\ \hline
(0, 3, 5, 6)
&(0, 5, 6, 7)
\\ \hline
(0, 3, 5, 6)
&(1, 2, 3, 4)
\\ \hline
(0, 3, 5, 6)
&(1, 2, 3, 5)
\\ \hline
(0, 3, 5, 6)
&(1, 2, 3, 6)
\\ \hline
(0, 3, 5, 6)
&(1, 2, 3, 7)
\\ \hline
(0, 3, 5, 6)
&(1, 4, 5, 6)
\\ \hline
(0, 3, 5, 6)
&(1, 4, 5, 7)
\\ \hline
(0, 3, 5, 6)
&(1, 4, 6, 7)
\\ \hline
(0, 3, 5, 6)
&(1, 5, 6, 7)
\\ \hline
(0, 3, 5, 6)
&(2, 4, 5, 6)
\\ \hline
(0, 3, 5, 6)
&(2, 4, 5, 7)
\\ \hline
(0, 3, 5, 6)
&(2, 4, 6, 7)
\\ \hline
(0, 3, 5, 6)
&(2, 5, 6, 7)
\\ \hline
(0, 3, 5, 6)
&(3, 4, 5, 6)
\\ \hline
(0, 3, 5, 6)
&(3, 4, 5, 7)
\\ \hline
(0, 3, 5, 6)
&(3, 4, 6, 7)
\\ \hline
(0, 3, 5, 6)
&(3, 5, 6, 7)
\\ \hline
(0, 3, 5, 6)
&(4, 5, 6, 7)
\\ \hline
(0, 3, 5, 7)
&(0, 1, 2, 3)
\\ \hline
(0, 3, 5, 7)
&(1, 3, 4, 5)
\\ \hline
(0, 3, 5, 7)
&(1, 3, 6, 7)
\\ \hline
(0, 3, 5, 7)
&(4, 5, 6, 7)
\\ \hline
(0, 3, 6, 7)
&(0, 1, 2, 3)
\\ \hline
(0, 3, 6, 7)
&(4, 5, 6, 7)
\\ \hline
(0, 4, 5, 6)
&(0, 1, 2, 3)
\\ \hline
(0, 4, 5, 6)
&(0, 1, 4, 5)
\\ \hline
(0, 4, 5, 6)
&(0, 2, 4, 6)
\\ \hline
(0, 4, 5, 6)
&(0, 3, 5, 6)
\\ \hline
(0, 4, 5, 6)
&(1, 2, 4, 7)
\\ \hline
(0, 4, 5, 6)
&(1, 3, 5, 7)
\\ \hline
(0, 4, 5, 6)
&(2, 3, 6, 7)
\\ \hline
(0, 4, 5, 6)
&(4, 5, 6, 7)
\\ \hline
(0, 4, 5, 7)
&(0, 2, 3, 7)
\\ \hline
(0, 4, 5, 7)
&(0, 3, 5, 6)
\\ \hline
(0, 4, 5, 7)
&(1, 2, 4, 7)
\\ \hline
(0, 4, 6, 7)
&(0, 3, 5, 6)
\\ \hline
(0, 4, 6, 7)
&(0, 4, 6, 7)
\\ \hline
(0, 4, 6, 7)
&(1, 2, 4, 7)
\\ \hline
(0, 5, 6, 7)
&(0, 1, 2, 3)
\\ \hline
(0, 5, 6, 7)
&(0, 1, 6, 7)
\\ \hline
(0, 5, 6, 7)
&(0, 2, 5, 7)
\\ \hline
(0, 5, 6, 7)
&(0, 3, 5, 6)
\\ \hline
(0, 5, 6, 7)
&(1, 2, 4, 7)
\\ \hline
(0, 5, 6, 7)
&(1, 3, 4, 6)
\\ \hline
(0, 5, 6, 7)
&(2, 3, 4, 5)
\\ \hline
(0, 5, 6, 7)
&(4, 5, 6, 7)
\\ \hline
(1, 2, 3, 4)
&(0, 1, 2, 3)
\\ \hline
(1, 2, 3, 4)
&(0, 1, 6, 7)
\\ \hline
(1, 2, 3, 4)
&(0, 2, 5, 7)
\\ \hline
(1, 2, 3, 4)
&(0, 3, 5, 6)
\\ \hline
(1, 2, 3, 4)
&(1, 2, 4, 7)
\\ \hline
(1, 2, 3, 4)
&(1, 3, 4, 6)
\\ \hline
(1, 2, 3, 4)
&(2, 3, 4, 5)
\\ \hline
(1, 2, 3, 4)
&(4, 5, 6, 7)
\\ \hline
(1, 2, 3, 5)
&(0, 3, 5, 6)
\\ \hline
(1, 2, 3, 5)
&(1, 2, 3, 5)
\\ \hline
(1, 2, 3, 5)
&(1, 2, 4, 7)
\\ \hline
(1, 2, 3, 6)
&(0, 3, 5, 6)
\\ \hline
(1, 2, 3, 6)
&(1, 2, 4, 7)
\\ \hline
(1, 2, 3, 6)
&(1, 4, 5, 6)
\\ \hline
(1, 2, 3, 7)
&(0, 1, 2, 3)
\\ \hline
(1, 2, 3, 7)
&(0, 1, 4, 5)
\\ \hline
(1, 2, 3, 7)
&(0, 2, 4, 6)
\\ \hline
(1, 2, 3, 7)
&(0, 3, 5, 6)
\\ \hline
(1, 2, 3, 7)
&(1, 2, 4, 7)
\\ \hline
(1, 2, 3, 7)
&(1, 3, 5, 7)
\\ \hline
(1, 2, 3, 7)
&(2, 3, 6, 7)
\\ \hline
(1, 2, 3, 7)
&(4, 5, 6, 7)
\\ \hline
(1, 2, 4, 5)
&(0, 1, 2, 3)
\\ \hline
(1, 2, 4, 5)
&(4, 5, 6, 7)
\\ \hline
(1, 2, 4, 6)
&(0, 1, 2, 3)
\\ \hline
(1, 2, 4, 6)
&(0, 2, 4, 5)
\\ \hline
(1, 2, 4, 6)
&(0, 2, 6, 7)
\\ \hline
(1, 2, 4, 6)
&(4, 5, 6, 7)
\\ \hline
(1, 2, 4, 7)
&(0, 1, 2, 3)
\\ \hline
(1, 2, 4, 7)
&(0, 1, 2, 4)
\\ \hline
(1, 2, 4, 7)
&(0, 1, 2, 5)
\\ \hline
(1, 2, 4, 7)
&(0, 1, 2, 6)
\\ \hline
(1, 2, 4, 7)
&(0, 1, 2, 7)
\\ \hline
(1, 2, 4, 7)
&(0, 1, 3, 4)
\\ \hline
(1, 2, 4, 7)
&(0, 1, 3, 5)
\\ \hline
(1, 2, 4, 7)
&(0, 1, 3, 6)
\\ \hline
(1, 2, 4, 7)
&(0, 1, 3, 7)
\\ \hline
(1, 2, 4, 7)
&(0, 2, 3, 4)
\\ \hline
(1, 2, 4, 7)
&(0, 2, 3, 5)
\\ \hline
(1, 2, 4, 7)
&(0, 2, 3, 6)
\\ \hline
(1, 2, 4, 7)
&(0, 2, 3, 7)
\\ \hline
(1, 2, 4, 7)
&(0, 4, 5, 6)
\\ \hline
(1, 2, 4, 7)
&(0, 4, 5, 7)
\\ \hline
(1, 2, 4, 7)
&(0, 4, 6, 7)
\\ \hline
(1, 2, 4, 7)
&(0, 5, 6, 7)
\\ \hline
(1, 2, 4, 7)
&(1, 2, 3, 4)
\\ \hline
(1, 2, 4, 7)
&(1, 2, 3, 5)
\\ \hline
(1, 2, 4, 7)
&(1, 2, 3, 6)
\\ \hline
(1, 2, 4, 7)
&(1, 2, 3, 7)
\\ \hline
(1, 2, 4, 7)
&(1, 4, 5, 6)
\\ \hline
(1, 2, 4, 7)
&(1, 4, 5, 7)
\\ \hline
(1, 2, 4, 7)
&(1, 4, 6, 7)
\\ \hline
(1, 2, 4, 7)
&(1, 5, 6, 7)
\\ \hline
(1, 2, 4, 7)
&(2, 4, 5, 6)
\\ \hline
(1, 2, 4, 7)
&(2, 4, 5, 7)
\\ \hline
(1, 2, 4, 7)
&(2, 4, 6, 7)
\\ \hline
(1, 2, 4, 7)
&(2, 5, 6, 7)
\\ \hline
(1, 2, 4, 7)
&(3, 4, 5, 6)
\\ \hline
(1, 2, 4, 7)
&(3, 4, 5, 7)
\\ \hline
(1, 2, 4, 7)
&(3, 4, 6, 7)
\\ \hline
(1, 2, 4, 7)
&(3, 5, 6, 7)
\\ \hline
(1, 2, 4, 7)
&(4, 5, 6, 7)
\\ \hline
(1, 2, 5, 7)
&(0, 1, 2, 3)
\\ \hline
(1, 2, 5, 7)
&(1, 3, 4, 5)
\\ \hline
(1, 2, 5, 7)
&(1, 3, 6, 7)
\\ \hline
(1, 2, 5, 7)
&(4, 5, 6, 7)
\\ \hline
(1, 2, 6, 7)
&(0, 1, 2, 3)
\\ \hline
(1, 2, 6, 7)
&(4, 5, 6, 7)
\\ \hline
(1, 3, 4, 5)
&(0, 3, 5, 7)
\\ \hline
(1, 3, 4, 5)
&(1, 2, 5, 7)
\\ \hline
(1, 3, 4, 6)
&(0, 1, 2, 7)
\\ \hline
(1, 3, 4, 6)
&(0, 1, 3, 6)
\\ \hline
(1, 3, 4, 6)
&(0, 2, 3, 5)
\\ \hline
(1, 3, 4, 6)
&(0, 5, 6, 7)
\\ \hline
(1, 3, 4, 6)
&(1, 2, 3, 4)
\\ \hline
(1, 3, 4, 6)
&(1, 4, 6, 7)
\\ \hline
(1, 3, 4, 6)
&(2, 4, 5, 7)
\\ \hline
(1, 3, 4, 6)
&(3, 4, 5, 6)
\\ \hline
(1, 3, 4, 7)
&(0, 1, 2, 3)
\\ \hline
(1, 3, 4, 7)
&(0, 1, 5, 7)
\\ \hline
(1, 3, 4, 7)
&(2, 3, 5, 7)
\\ \hline
(1, 3, 4, 7)
&(4, 5, 6, 7)
\\ \hline
(1, 3, 5, 6)
&(0, 1, 2, 3)
\\ \hline
(1, 3, 5, 6)
&(0, 1, 5, 7)
\\ \hline
(1, 3, 5, 6)
&(2, 3, 5, 7)
\\ \hline
(1, 3, 5, 6)
&(4, 5, 6, 7)
\\ \hline
(1, 3, 5, 7)
&(0, 1, 2, 4)
\\ \hline
(1, 3, 5, 7)
&(0, 1, 3, 5)
\\ \hline
(1, 3, 5, 7)
&(0, 2, 3, 6)
\\ \hline
(1, 3, 5, 7)
&(0, 4, 5, 6)
\\ \hline
(1, 3, 5, 7)
&(1, 2, 3, 7)
\\ \hline
(1, 3, 5, 7)
&(1, 4, 5, 7)
\\ \hline
(1, 3, 5, 7)
&(2, 4, 6, 7)
\\ \hline
(1, 3, 5, 7)
&(3, 5, 6, 7)
\\ \hline
(1, 3, 6, 7)
&(0, 3, 5, 7)
\\ \hline
(1, 3, 6, 7)
&(1, 2, 5, 7)
\\ \hline
(1, 4, 5, 6)
&(0, 3, 5, 6)
\\ \hline
(1, 4, 5, 6)
&(1, 2, 3, 6)
\\ \hline
(1, 4, 5, 6)
&(1, 2, 4, 7)
\\ \hline
(1, 4, 5, 7)
&(0, 1, 2, 3)
\\ \hline
(1, 4, 5, 7)
&(0, 1, 4, 5)
\\ \hline
(1, 4, 5, 7)
&(0, 2, 4, 6)
\\ \hline
(1, 4, 5, 7)
&(0, 3, 5, 6)
\\ \hline
(1, 4, 5, 7)
&(1, 2, 4, 7)
\\ \hline
(1, 4, 5, 7)
&(1, 3, 5, 7)
\\ \hline
(1, 4, 5, 7)
&(2, 3, 6, 7)
\\ \hline
(1, 4, 5, 7)
&(4, 5, 6, 7)
\\ \hline
(1, 4, 6, 7)
&(0, 1, 2, 3)
\\ \hline
(1, 4, 6, 7)
&(0, 1, 6, 7)
\\ \hline
(1, 4, 6, 7)
&(0, 2, 5, 7)
\\ \hline
(1, 4, 6, 7)
&(0, 3, 5, 6)
\\ \hline
(1, 4, 6, 7)
&(1, 2, 4, 7)
\\ \hline
(1, 4, 6, 7)
&(1, 3, 4, 6)
\\ \hline
(1, 4, 6, 7)
&(2, 3, 4, 5)
\\ \hline
(1, 4, 6, 7)
&(4, 5, 6, 7)
\\ \hline
(1, 5, 6, 7)
&(0, 3, 5, 6)
\\ \hline
(1, 5, 6, 7)
&(1, 2, 4, 7)
\\ \hline
(1, 5, 6, 7)
&(1, 5, 6, 7)
\\ \hline
(2, 3, 4, 5)
&(0, 1, 2, 7)
\\ \hline
(2, 3, 4, 5)
&(0, 1, 3, 6)
\\ \hline
(2, 3, 4, 5)
&(0, 1, 4, 5)
\\ \hline
(2, 3, 4, 5)
&(0, 2, 3, 5)
\\ \hline
(2, 3, 4, 5)
&(0, 5, 6, 7)
\\ \hline
(2, 3, 4, 5)
&(1, 2, 3, 4)
\\ \hline
(2, 3, 4, 5)
&(1, 4, 6, 7)
\\ \hline
(2, 3, 4, 5)
&(2, 3, 6, 7)
\\ \hline
(2, 3, 4, 5)
&(2, 4, 5, 7)
\\ \hline
(2, 3, 4, 5)
&(3, 4, 5, 6)
\\ \hline
(2, 3, 4, 6)
&(0, 2, 4, 7)
\\ \hline
(2, 3, 4, 6)
&(0, 2, 5, 6)
\\ \hline
(2, 3, 4, 7)
&(0, 1, 2, 3)
\\ \hline
(2, 3, 4, 7)
&(4, 5, 6, 7)
\\ \hline
(2, 3, 5, 6)
&(0, 1, 2, 3)
\\ \hline
(2, 3, 5, 6)
&(4, 5, 6, 7)
\\ \hline
(2, 3, 5, 7)
&(1, 3, 4, 7)
\\ \hline
(2, 3, 5, 7)
&(1, 3, 5, 6)
\\ \hline
(2, 3, 6, 7)
&(0, 1, 2, 4)
\\ \hline
(2, 3, 6, 7)
&(0, 1, 3, 5)
\\ \hline
(2, 3, 6, 7)
&(0, 1, 6, 7)
\\ \hline
(2, 3, 6, 7)
&(0, 2, 3, 6)
\\ \hline
(2, 3, 6, 7)
&(0, 4, 5, 6)
\\ \hline
(2, 3, 6, 7)
&(1, 2, 3, 7)
\\ \hline
(2, 3, 6, 7)
&(1, 4, 5, 7)
\\ \hline
(2, 3, 6, 7)
&(2, 3, 4, 5)
\\ \hline
(2, 3, 6, 7)
&(2, 4, 6, 7)
\\ \hline
(2, 3, 6, 7)
&(3, 5, 6, 7)
\\ \hline
(2, 4, 5, 6)
&(0, 3, 5, 6)
\\ \hline
(2, 4, 5, 6)
&(1, 2, 4, 7)
\\ \hline
(2, 4, 5, 6)
&(2, 4, 5, 6)
\\ \hline
(2, 4, 5, 7)
&(0, 1, 2, 3)
\\ \hline
(2, 4, 5, 7)
&(0, 1, 6, 7)
\\ \hline
(2, 4, 5, 7)
&(0, 2, 5, 7)
\\ \hline
(2, 4, 5, 7)
&(0, 3, 5, 6)
\\ \hline
(2, 4, 5, 7)
&(1, 2, 4, 7)
\\ \hline
(2, 4, 5, 7)
&(1, 3, 4, 6)
\\ \hline
(2, 4, 5, 7)
&(2, 3, 4, 5)
\\ \hline
(2, 4, 5, 7)
&(4, 5, 6, 7)
\\ \hline
(2, 4, 6, 7)
&(0, 1, 2, 3)
\\ \hline
(2, 4, 6, 7)
&(0, 1, 4, 5)
\\ \hline
(2, 4, 6, 7)
&(0, 2, 4, 6)
\\ \hline
(2, 4, 6, 7)
&(0, 3, 5, 6)
\\ \hline
(2, 4, 6, 7)
&(1, 2, 4, 7)
\\ \hline
(2, 4, 6, 7)
&(1, 3, 5, 7)
\\ \hline
(2, 4, 6, 7)
&(2, 3, 6, 7)
\\ \hline
(2, 4, 6, 7)
&(4, 5, 6, 7)
\\ \hline
(2, 5, 6, 7)
&(0, 1, 2, 5)
\\ \hline
(2, 5, 6, 7)
&(0, 3, 5, 6)
\\ \hline
(2, 5, 6, 7)
&(1, 2, 4, 7)
\\ \hline
(3, 4, 5, 6)
&(0, 1, 2, 3)
\\ \hline
(3, 4, 5, 6)
&(0, 1, 6, 7)
\\ \hline
(3, 4, 5, 6)
&(0, 2, 5, 7)
\\ \hline
(3, 4, 5, 6)
&(0, 3, 5, 6)
\\ \hline
(3, 4, 5, 6)
&(1, 2, 4, 7)
\\ \hline
(3, 4, 5, 6)
&(1, 3, 4, 6)
\\ \hline
(3, 4, 5, 6)
&(2, 3, 4, 5)
\\ \hline
(3, 4, 5, 6)
&(4, 5, 6, 7)
\\ \hline
(3, 4, 5, 7)
&(0, 3, 5, 6)
\\ \hline
(3, 4, 5, 7)
&(1, 2, 4, 7)
\\ \hline
(3, 4, 5, 7)
&(3, 4, 5, 7)
\\ \hline
(3, 4, 6, 7)
&(0, 1, 3, 4)
\\ \hline
(3, 4, 6, 7)
&(0, 3, 5, 6)
\\ \hline
(3, 4, 6, 7)
&(1, 2, 4, 7)
\\ \hline
(3, 5, 6, 7)
&(0, 1, 2, 3)
\\ \hline
(3, 5, 6, 7)
&(0, 1, 4, 5)
\\ \hline
(3, 5, 6, 7)
&(0, 2, 4, 6)
\\ \hline
(3, 5, 6, 7)
&(0, 3, 5, 6)
\\ \hline
(3, 5, 6, 7)
&(1, 2, 4, 7)
\\ \hline
(3, 5, 6, 7)
&(1, 3, 5, 7)
\\ \hline
(3, 5, 6, 7)
&(2, 3, 6, 7)
\\ \hline
(3, 5, 6, 7)
&(4, 5, 6, 7)
\\ \hline
(4, 5, 6, 7)
&(0, 1, 2, 4)
\\ \hline
(4, 5, 6, 7)
&(0, 1, 2, 7)
\\ \hline
(4, 5, 6, 7)
&(0, 1, 3, 5)
\\ \hline
(4, 5, 6, 7)
&(0, 1, 3, 6)
\\ \hline
(4, 5, 6, 7)
&(0, 1, 4, 7)
\\ \hline
(4, 5, 6, 7)
&(0, 1, 5, 6)
\\ \hline
(4, 5, 6, 7)
&(0, 2, 3, 5)
\\ \hline
(4, 5, 6, 7)
&(0, 2, 3, 6)
\\ \hline
(4, 5, 6, 7)
&(0, 2, 4, 7)
\\ \hline
(4, 5, 6, 7)
&(0, 2, 5, 6)
\\ \hline
(4, 5, 6, 7)
&(0, 3, 4, 5)
\\ \hline
(4, 5, 6, 7)
&(0, 3, 4, 6)
\\ \hline
(4, 5, 6, 7)
&(0, 3, 5, 6)
\\ \hline
(4, 5, 6, 7)
&(0, 3, 5, 7)
\\ \hline
(4, 5, 6, 7)
&(0, 3, 6, 7)
\\ \hline
(4, 5, 6, 7)
&(0, 4, 5, 6)
\\ \hline
(4, 5, 6, 7)
&(0, 5, 6, 7)
\\ \hline
(4, 5, 6, 7)
&(1, 2, 3, 4)
\\ \hline
(4, 5, 6, 7)
&(1, 2, 3, 7)
\\ \hline
(4, 5, 6, 7)
&(1, 2, 4, 5)
\\ \hline
(4, 5, 6, 7)
&(1, 2, 4, 6)
\\ \hline
(4, 5, 6, 7)
&(1, 2, 4, 7)
\\ \hline
(4, 5, 6, 7)
&(1, 2, 5, 7)
\\ \hline
(4, 5, 6, 7)
&(1, 2, 6, 7)
\\ \hline
(4, 5, 6, 7)
&(1, 3, 4, 7)
\\ \hline
(4, 5, 6, 7)
&(1, 3, 5, 6)
\\ \hline
(4, 5, 6, 7)
&(1, 4, 5, 7)
\\ \hline
(4, 5, 6, 7)
&(1, 4, 6, 7)
\\ \hline
(4, 5, 6, 7)
&(2, 3, 4, 7)
\\ \hline
(4, 5, 6, 7)
&(2, 3, 5, 6)
\\ \hline
(4, 5, 6, 7)
&(2, 4, 5, 7)
\\ \hline
(4, 5, 6, 7)
&(2, 4, 6, 7)
\\ \hline
(4, 5, 6, 7)
&(3, 4, 5, 6)
\\ \hline
(4, 5, 6, 7)
&(3, 5, 6, 7)
\\ \hline
(0, 1, 2, 3, 4)
&(0, 1, 2, 4, 7)
\\ \hline
(0, 1, 2, 3, 4)
&(0, 1, 3, 5, 6)
\\ \hline
(0, 1, 2, 3, 4)
&(0, 2, 3, 5, 6)
\\ \hline
(0, 1, 2, 3, 4)
&(0, 3, 4, 5, 6)
\\ \hline
(0, 1, 2, 3, 4)
&(0, 3, 5, 6, 7)
\\ \hline
(0, 1, 2, 3, 4)
&(1, 2, 3, 4, 7)
\\ \hline
(0, 1, 2, 3, 4)
&(1, 2, 4, 5, 7)
\\ \hline
(0, 1, 2, 3, 4)
&(1, 2, 4, 6, 7)
\\ \hline
(0, 1, 2, 3, 5)
&(0, 1, 2, 4, 7)
\\ \hline
(0, 1, 2, 3, 5)
&(0, 1, 3, 5, 6)
\\ \hline
(0, 1, 2, 3, 5)
&(0, 2, 3, 5, 6)
\\ \hline
(0, 1, 2, 3, 5)
&(0, 3, 4, 5, 6)
\\ \hline
(0, 1, 2, 3, 5)
&(0, 3, 5, 6, 7)
\\ \hline
(0, 1, 2, 3, 5)
&(1, 2, 3, 4, 7)
\\ \hline
(0, 1, 2, 3, 5)
&(1, 2, 4, 5, 7)
\\ \hline
(0, 1, 2, 3, 5)
&(1, 2, 4, 6, 7)
\\ \hline
(0, 1, 2, 3, 6)
&(0, 1, 2, 4, 7)
\\ \hline
(0, 1, 2, 3, 6)
&(0, 1, 3, 5, 6)
\\ \hline
(0, 1, 2, 3, 6)
&(0, 2, 3, 5, 6)
\\ \hline
(0, 1, 2, 3, 6)
&(0, 3, 4, 5, 6)
\\ \hline
(0, 1, 2, 3, 6)
&(0, 3, 5, 6, 7)
\\ \hline
(0, 1, 2, 3, 6)
&(1, 2, 3, 4, 7)
\\ \hline
(0, 1, 2, 3, 6)
&(1, 2, 4, 5, 7)
\\ \hline
(0, 1, 2, 3, 6)
&(1, 2, 4, 6, 7)
\\ \hline
(0, 1, 2, 3, 7)
&(0, 1, 2, 4, 7)
\\ \hline
(0, 1, 2, 3, 7)
&(0, 1, 3, 5, 6)
\\ \hline
(0, 1, 2, 3, 7)
&(0, 2, 3, 5, 6)
\\ \hline
(0, 1, 2, 3, 7)
&(0, 3, 4, 5, 6)
\\ \hline
(0, 1, 2, 3, 7)
&(0, 3, 5, 6, 7)
\\ \hline
(0, 1, 2, 3, 7)
&(1, 2, 3, 4, 7)
\\ \hline
(0, 1, 2, 3, 7)
&(1, 2, 4, 5, 7)
\\ \hline
(0, 1, 2, 3, 7)
&(1, 2, 4, 6, 7)
\\ \hline
(0, 1, 2, 4, 5)
&(0, 1, 2, 5, 7)
\\ \hline
(0, 1, 2, 4, 5)
&(1, 3, 4, 5, 6)
\\ \hline
(0, 1, 2, 4, 6)
&(0, 1, 2, 5, 7)
\\ \hline
(0, 1, 2, 4, 6)
&(0, 3, 4, 6, 7)
\\ \hline
(0, 1, 2, 4, 6)
&(1, 3, 4, 5, 6)
\\ \hline
(0, 1, 2, 4, 6)
&(2, 3, 4, 5, 7)
\\ \hline
(0, 1, 2, 4, 7)
&(0, 1, 2, 3, 4)
\\ \hline
(0, 1, 2, 4, 7)
&(0, 1, 2, 3, 5)
\\ \hline
(0, 1, 2, 4, 7)
&(0, 1, 2, 3, 6)
\\ \hline
(0, 1, 2, 4, 7)
&(0, 1, 2, 3, 7)
\\ \hline
(0, 1, 2, 4, 7)
&(0, 4, 5, 6, 7)
\\ \hline
(0, 1, 2, 4, 7)
&(1, 4, 5, 6, 7)
\\ \hline
(0, 1, 2, 4, 7)
&(2, 4, 5, 6, 7)
\\ \hline
(0, 1, 2, 4, 7)
&(3, 4, 5, 6, 7)
\\ \hline
(0, 1, 2, 5, 6)
&(0, 1, 2, 5, 7)
\\ \hline
(0, 1, 2, 5, 6)
&(0, 2, 4, 6, 7)
\\ \hline
(0, 1, 2, 5, 6)
&(1, 3, 4, 5, 6)
\\ \hline
(0, 1, 2, 5, 7)
&(0, 1, 2, 4, 5)
\\ \hline
(0, 1, 2, 5, 7)
&(0, 1, 2, 4, 6)
\\ \hline
(0, 1, 2, 5, 7)
&(0, 1, 2, 5, 6)
\\ \hline
(0, 1, 2, 5, 7)
&(0, 1, 4, 5, 6)
\\ \hline
(0, 1, 2, 5, 7)
&(0, 2, 4, 5, 6)
\\ \hline
(0, 1, 2, 5, 7)
&(1, 2, 4, 5, 6)
\\ \hline
(0, 1, 2, 6, 7)
&(1, 3, 4, 5, 7)
\\ \hline
(0, 1, 3, 4, 5)
&(0, 1, 3, 4, 6)
\\ \hline
(0, 1, 3, 4, 5)
&(0, 2, 4, 5, 7)
\\ \hline
(0, 1, 3, 4, 6)
&(0, 1, 3, 4, 5)
\\ \hline
(0, 1, 3, 4, 6)
&(0, 1, 3, 4, 7)
\\ \hline
(0, 1, 3, 4, 6)
&(0, 1, 3, 5, 7)
\\ \hline
(0, 1, 3, 4, 6)
&(0, 1, 4, 5, 7)
\\ \hline
(0, 1, 3, 4, 6)
&(0, 3, 4, 5, 7)
\\ \hline
(0, 1, 3, 4, 6)
&(1, 3, 4, 5, 7)
\\ \hline
(0, 1, 3, 4, 7)
&(0, 1, 3, 4, 6)
\\ \hline
(0, 1, 3, 4, 7)
&(0, 2, 4, 5, 7)
\\ \hline
(0, 1, 3, 4, 7)
&(1, 3, 5, 6, 7)
\\ \hline
(0, 1, 3, 5, 6)
&(0, 1, 2, 3, 4)
\\ \hline
(0, 1, 3, 5, 6)
&(0, 1, 2, 3, 5)
\\ \hline
(0, 1, 3, 5, 6)
&(0, 1, 2, 3, 6)
\\ \hline
(0, 1, 3, 5, 6)
&(0, 1, 2, 3, 7)
\\ \hline
(0, 1, 3, 5, 6)
&(0, 4, 5, 6, 7)
\\ \hline
(0, 1, 3, 5, 6)
&(1, 4, 5, 6, 7)
\\ \hline
(0, 1, 3, 5, 6)
&(2, 4, 5, 6, 7)
\\ \hline
(0, 1, 3, 5, 6)
&(3, 4, 5, 6, 7)
\\ \hline
(0, 1, 3, 5, 7)
&(0, 1, 3, 4, 6)
\\ \hline
(0, 1, 3, 5, 7)
&(0, 2, 4, 5, 7)
\\ \hline
(0, 1, 3, 5, 7)
&(1, 2, 5, 6, 7)
\\ \hline
(0, 1, 3, 5, 7)
&(2, 3, 4, 5, 6)
\\ \hline
(0, 1, 3, 6, 7)
&(0, 2, 4, 5, 6)
\\ \hline
(0, 1, 4, 5, 6)
&(0, 1, 2, 5, 7)
\\ \hline
(0, 1, 4, 5, 6)
&(1, 3, 4, 5, 6)
\\ \hline
(0, 1, 4, 5, 7)
&(0, 1, 3, 4, 6)
\\ \hline
(0, 1, 4, 5, 7)
&(0, 2, 4, 5, 7)
\\ \hline
(0, 1, 4, 6, 7)
&(1, 2, 3, 5, 7)
\\ \hline
(0, 1, 5, 6, 7)
&(0, 2, 3, 4, 6)
\\ \hline
(0, 2, 3, 4, 5)
&(1, 3, 5, 6, 7)
\\ \hline
(0, 2, 3, 4, 6)
&(0, 1, 5, 6, 7)
\\ \hline
(0, 2, 3, 4, 6)
&(0, 2, 3, 5, 7)
\\ \hline
(0, 2, 3, 4, 6)
&(1, 2, 4, 5, 6)
\\ \hline
(0, 2, 3, 4, 6)
&(1, 3, 4, 6, 7)
\\ \hline
(0, 2, 3, 4, 7)
&(0, 2, 3, 5, 7)
\\ \hline
(0, 2, 3, 4, 7)
&(0, 2, 4, 5, 6)
\\ \hline
(0, 2, 3, 4, 7)
&(1, 3, 4, 6, 7)
\\ \hline
(0, 2, 3, 5, 6)
&(0, 1, 2, 3, 4)
\\ \hline
(0, 2, 3, 5, 6)
&(0, 1, 2, 3, 5)
\\ \hline
(0, 2, 3, 5, 6)
&(0, 1, 2, 3, 6)
\\ \hline
(0, 2, 3, 5, 6)
&(0, 1, 2, 3, 7)
\\ \hline
(0, 2, 3, 5, 6)
&(0, 4, 5, 6, 7)
\\ \hline
(0, 2, 3, 5, 6)
&(1, 4, 5, 6, 7)
\\ \hline
(0, 2, 3, 5, 6)
&(2, 4, 5, 6, 7)
\\ \hline
(0, 2, 3, 5, 6)
&(3, 4, 5, 6, 7)
\\ \hline
(0, 2, 3, 5, 7)
&(0, 2, 3, 4, 6)
\\ \hline
(0, 2, 3, 5, 7)
&(0, 2, 3, 4, 7)
\\ \hline
(0, 2, 3, 5, 7)
&(0, 2, 3, 6, 7)
\\ \hline
(0, 2, 3, 5, 7)
&(0, 2, 4, 6, 7)
\\ \hline
(0, 2, 3, 5, 7)
&(0, 3, 4, 6, 7)
\\ \hline
(0, 2, 3, 5, 7)
&(2, 3, 4, 6, 7)
\\ \hline
(0, 2, 3, 6, 7)
&(0, 2, 3, 5, 7)
\\ \hline
(0, 2, 3, 6, 7)
&(1, 3, 4, 6, 7)
\\ \hline
(0, 2, 4, 5, 6)
&(0, 1, 2, 5, 7)
\\ \hline
(0, 2, 4, 5, 6)
&(0, 1, 3, 6, 7)
\\ \hline
(0, 2, 4, 5, 6)
&(0, 2, 3, 4, 7)
\\ \hline
(0, 2, 4, 5, 6)
&(1, 3, 4, 5, 6)
\\ \hline
(0, 2, 4, 5, 7)
&(0, 1, 3, 4, 5)
\\ \hline
(0, 2, 4, 5, 7)
&(0, 1, 3, 4, 7)
\\ \hline
(0, 2, 4, 5, 7)
&(0, 1, 3, 5, 7)
\\ \hline
(0, 2, 4, 5, 7)
&(0, 1, 4, 5, 7)
\\ \hline
(0, 2, 4, 5, 7)
&(0, 3, 4, 5, 7)
\\ \hline
(0, 2, 4, 5, 7)
&(1, 3, 4, 5, 7)
\\ \hline
(0, 2, 4, 6, 7)
&(0, 1, 2, 5, 6)
\\ \hline
(0, 2, 4, 6, 7)
&(0, 2, 3, 5, 7)
\\ \hline
(0, 2, 4, 6, 7)
&(1, 2, 3, 4, 5)
\\ \hline
(0, 2, 4, 6, 7)
&(1, 3, 4, 6, 7)
\\ \hline
(0, 2, 5, 6, 7)
&(1, 2, 3, 5, 6)
\\ \hline
(0, 2, 5, 6, 7)
&(1, 2, 3, 5, 7)
\\ \hline
(0, 2, 5, 6, 7)
&(1, 2, 3, 6, 7)
\\ \hline
(0, 2, 5, 6, 7)
&(1, 2, 5, 6, 7)
\\ \hline
(0, 2, 5, 6, 7)
&(1, 3, 5, 6, 7)
\\ \hline
(0, 2, 5, 6, 7)
&(2, 3, 5, 6, 7)
\\ \hline
(0, 3, 4, 5, 6)
&(0, 1, 2, 3, 4)
\\ \hline
(0, 3, 4, 5, 6)
&(0, 1, 2, 3, 5)
\\ \hline
(0, 3, 4, 5, 6)
&(0, 1, 2, 3, 6)
\\ \hline
(0, 3, 4, 5, 6)
&(0, 1, 2, 3, 7)
\\ \hline
(0, 3, 4, 5, 6)
&(0, 4, 5, 6, 7)
\\ \hline
(0, 3, 4, 5, 6)
&(1, 4, 5, 6, 7)
\\ \hline
(0, 3, 4, 5, 6)
&(2, 4, 5, 6, 7)
\\ \hline
(0, 3, 4, 5, 6)
&(3, 4, 5, 6, 7)
\\ \hline
(0, 3, 4, 5, 7)
&(0, 1, 3, 4, 6)
\\ \hline
(0, 3, 4, 5, 7)
&(0, 2, 4, 5, 7)
\\ \hline
(0, 3, 4, 5, 7)
&(1, 2, 3, 5, 7)
\\ \hline
(0, 3, 4, 6, 7)
&(0, 1, 2, 4, 6)
\\ \hline
(0, 3, 4, 6, 7)
&(0, 2, 3, 5, 7)
\\ \hline
(0, 3, 4, 6, 7)
&(1, 3, 4, 6, 7)
\\ \hline
(0, 3, 5, 6, 7)
&(0, 1, 2, 3, 4)
\\ \hline
(0, 3, 5, 6, 7)
&(0, 1, 2, 3, 5)
\\ \hline
(0, 3, 5, 6, 7)
&(0, 1, 2, 3, 6)
\\ \hline
(0, 3, 5, 6, 7)
&(0, 1, 2, 3, 7)
\\ \hline
(0, 3, 5, 6, 7)
&(0, 4, 5, 6, 7)
\\ \hline
(0, 3, 5, 6, 7)
&(1, 4, 5, 6, 7)
\\ \hline
(0, 3, 5, 6, 7)
&(2, 4, 5, 6, 7)
\\ \hline
(0, 3, 5, 6, 7)
&(3, 4, 5, 6, 7)
\\ \hline
(0, 4, 5, 6, 7)
&(0, 1, 2, 4, 7)
\\ \hline
(0, 4, 5, 6, 7)
&(0, 1, 3, 5, 6)
\\ \hline
(0, 4, 5, 6, 7)
&(0, 2, 3, 5, 6)
\\ \hline
(0, 4, 5, 6, 7)
&(0, 3, 4, 5, 6)
\\ \hline
(0, 4, 5, 6, 7)
&(0, 3, 5, 6, 7)
\\ \hline
(0, 4, 5, 6, 7)
&(1, 2, 3, 4, 7)
\\ \hline
(0, 4, 5, 6, 7)
&(1, 2, 4, 5, 7)
\\ \hline
(0, 4, 5, 6, 7)
&(1, 2, 4, 6, 7)
\\ \hline
(1, 2, 3, 4, 5)
&(0, 2, 4, 6, 7)
\\ \hline
(1, 2, 3, 4, 6)
&(1, 2, 3, 5, 6)
\\ \hline
(1, 2, 3, 4, 6)
&(1, 2, 3, 5, 7)
\\ \hline
(1, 2, 3, 4, 6)
&(1, 2, 3, 6, 7)
\\ \hline
(1, 2, 3, 4, 6)
&(1, 2, 5, 6, 7)
\\ \hline
(1, 2, 3, 4, 6)
&(1, 3, 5, 6, 7)
\\ \hline
(1, 2, 3, 4, 6)
&(2, 3, 5, 6, 7)
\\ \hline
(1, 2, 3, 4, 7)
&(0, 1, 2, 3, 4)
\\ \hline
(1, 2, 3, 4, 7)
&(0, 1, 2, 3, 5)
\\ \hline
(1, 2, 3, 4, 7)
&(0, 1, 2, 3, 6)
\\ \hline
(1, 2, 3, 4, 7)
&(0, 1, 2, 3, 7)
\\ \hline
(1, 2, 3, 4, 7)
&(0, 4, 5, 6, 7)
\\ \hline
(1, 2, 3, 4, 7)
&(1, 4, 5, 6, 7)
\\ \hline
(1, 2, 3, 4, 7)
&(2, 4, 5, 6, 7)
\\ \hline
(1, 2, 3, 4, 7)
&(3, 4, 5, 6, 7)
\\ \hline
(1, 2, 3, 5, 6)
&(0, 2, 5, 6, 7)
\\ \hline
(1, 2, 3, 5, 6)
&(1, 2, 3, 4, 6)
\\ \hline
(1, 2, 3, 5, 6)
&(1, 3, 4, 5, 7)
\\ \hline
(1, 2, 3, 5, 7)
&(0, 1, 4, 6, 7)
\\ \hline
(1, 2, 3, 5, 7)
&(0, 2, 5, 6, 7)
\\ \hline
(1, 2, 3, 5, 7)
&(0, 3, 4, 5, 7)
\\ \hline
(1, 2, 3, 5, 7)
&(1, 2, 3, 4, 6)
\\ \hline
(1, 2, 3, 6, 7)
&(0, 2, 5, 6, 7)
\\ \hline
(1, 2, 3, 6, 7)
&(1, 2, 3, 4, 6)
\\ \hline
(1, 2, 4, 5, 6)
&(0, 1, 2, 5, 7)
\\ \hline
(1, 2, 4, 5, 6)
&(0, 2, 3, 4, 6)
\\ \hline
(1, 2, 4, 5, 6)
&(1, 3, 4, 5, 6)
\\ \hline
(1, 2, 4, 5, 7)
&(0, 1, 2, 3, 4)
\\ \hline
(1, 2, 4, 5, 7)
&(0, 1, 2, 3, 5)
\\ \hline
(1, 2, 4, 5, 7)
&(0, 1, 2, 3, 6)
\\ \hline
(1, 2, 4, 5, 7)
&(0, 1, 2, 3, 7)
\\ \hline
(1, 2, 4, 5, 7)
&(0, 4, 5, 6, 7)
\\ \hline
(1, 2, 4, 5, 7)
&(1, 4, 5, 6, 7)
\\ \hline
(1, 2, 4, 5, 7)
&(2, 4, 5, 6, 7)
\\ \hline
(1, 2, 4, 5, 7)
&(3, 4, 5, 6, 7)
\\ \hline
(1, 2, 4, 6, 7)
&(0, 1, 2, 3, 4)
\\ \hline
(1, 2, 4, 6, 7)
&(0, 1, 2, 3, 5)
\\ \hline
(1, 2, 4, 6, 7)
&(0, 1, 2, 3, 6)
\\ \hline
(1, 2, 4, 6, 7)
&(0, 1, 2, 3, 7)
\\ \hline
(1, 2, 4, 6, 7)
&(0, 4, 5, 6, 7)
\\ \hline
(1, 2, 4, 6, 7)
&(1, 4, 5, 6, 7)
\\ \hline
(1, 2, 4, 6, 7)
&(2, 4, 5, 6, 7)
\\ \hline
(1, 2, 4, 6, 7)
&(3, 4, 5, 6, 7)
\\ \hline
(1, 2, 5, 6, 7)
&(0, 1, 3, 5, 7)
\\ \hline
(1, 2, 5, 6, 7)
&(0, 2, 5, 6, 7)
\\ \hline
(1, 2, 5, 6, 7)
&(1, 2, 3, 4, 6)
\\ \hline
(1, 3, 4, 5, 6)
&(0, 1, 2, 4, 5)
\\ \hline
(1, 3, 4, 5, 6)
&(0, 1, 2, 4, 6)
\\ \hline
(1, 3, 4, 5, 6)
&(0, 1, 2, 5, 6)
\\ \hline
(1, 3, 4, 5, 6)
&(0, 1, 4, 5, 6)
\\ \hline
(1, 3, 4, 5, 6)
&(0, 2, 4, 5, 6)
\\ \hline
(1, 3, 4, 5, 6)
&(1, 2, 4, 5, 6)
\\ \hline
(1, 3, 4, 5, 7)
&(0, 1, 2, 6, 7)
\\ \hline
(1, 3, 4, 5, 7)
&(0, 1, 3, 4, 6)
\\ \hline
(1, 3, 4, 5, 7)
&(0, 2, 4, 5, 7)
\\ \hline
(1, 3, 4, 5, 7)
&(1, 2, 3, 5, 6)
\\ \hline
(1, 3, 4, 6, 7)
&(0, 2, 3, 4, 6)
\\ \hline
(1, 3, 4, 6, 7)
&(0, 2, 3, 4, 7)
\\ \hline
(1, 3, 4, 6, 7)
&(0, 2, 3, 6, 7)
\\ \hline
(1, 3, 4, 6, 7)
&(0, 2, 4, 6, 7)
\\ \hline
(1, 3, 4, 6, 7)
&(0, 3, 4, 6, 7)
\\ \hline
(1, 3, 4, 6, 7)
&(2, 3, 4, 6, 7)
\\ \hline
(1, 3, 5, 6, 7)
&(0, 1, 3, 4, 7)
\\ \hline
(1, 3, 5, 6, 7)
&(0, 2, 3, 4, 5)
\\ \hline
(1, 3, 5, 6, 7)
&(0, 2, 5, 6, 7)
\\ \hline
(1, 3, 5, 6, 7)
&(1, 2, 3, 4, 6)
\\ \hline
(1, 4, 5, 6, 7)
&(0, 1, 2, 4, 7)
\\ \hline
(1, 4, 5, 6, 7)
&(0, 1, 3, 5, 6)
\\ \hline
(1, 4, 5, 6, 7)
&(0, 2, 3, 5, 6)
\\ \hline
(1, 4, 5, 6, 7)
&(0, 3, 4, 5, 6)
\\ \hline
(1, 4, 5, 6, 7)
&(0, 3, 5, 6, 7)
\\ \hline
(1, 4, 5, 6, 7)
&(1, 2, 3, 4, 7)
\\ \hline
(1, 4, 5, 6, 7)
&(1, 2, 4, 5, 7)
\\ \hline
(1, 4, 5, 6, 7)
&(1, 2, 4, 6, 7)
\\ \hline
(2, 3, 4, 5, 6)
&(0, 1, 3, 5, 7)
\\ \hline
(2, 3, 4, 5, 7)
&(0, 1, 2, 4, 6)
\\ \hline
(2, 3, 4, 6, 7)
&(0, 2, 3, 5, 7)
\\ \hline
(2, 3, 4, 6, 7)
&(1, 3, 4, 6, 7)
\\ \hline
(2, 3, 5, 6, 7)
&(0, 2, 5, 6, 7)
\\ \hline
(2, 3, 5, 6, 7)
&(1, 2, 3, 4, 6)
\\ \hline
(2, 4, 5, 6, 7)
&(0, 1, 2, 4, 7)
\\ \hline
(2, 4, 5, 6, 7)
&(0, 1, 3, 5, 6)
\\ \hline
(2, 4, 5, 6, 7)
&(0, 2, 3, 5, 6)
\\ \hline
(2, 4, 5, 6, 7)
&(0, 3, 4, 5, 6)
\\ \hline
(2, 4, 5, 6, 7)
&(0, 3, 5, 6, 7)
\\ \hline
(2, 4, 5, 6, 7)
&(1, 2, 3, 4, 7)
\\ \hline
(2, 4, 5, 6, 7)
&(1, 2, 4, 5, 7)
\\ \hline
(2, 4, 5, 6, 7)
&(1, 2, 4, 6, 7)
\\ \hline
(3, 4, 5, 6, 7)
&(0, 1, 2, 4, 7)
\\ \hline
(3, 4, 5, 6, 7)
&(0, 1, 3, 5, 6)
\\ \hline
(3, 4, 5, 6, 7)
&(0, 2, 3, 5, 6)
\\ \hline
(3, 4, 5, 6, 7)
&(0, 3, 4, 5, 6)
\\ \hline
(3, 4, 5, 6, 7)
&(0, 3, 5, 6, 7)
\\ \hline
(3, 4, 5, 6, 7)
&(1, 2, 3, 4, 7)
\\ \hline
(3, 4, 5, 6, 7)
&(1, 2, 4, 5, 7)
\\ \hline
(3, 4, 5, 6, 7)
&(1, 2, 4, 6, 7)
\\ \hline
(0, 1, 2, 3, 4, 6)
&(0, 1, 3, 4, 5, 7)
\\ \hline
(0, 1, 2, 3, 4, 6)
&(1, 2, 3, 5, 6, 7)
\\ \hline
(0, 1, 2, 3, 5, 7)
&(0, 1, 2, 4, 5, 6)
\\ \hline
(0, 1, 2, 3, 5, 7)
&(0, 2, 3, 4, 6, 7)
\\ \hline
(0, 1, 2, 4, 5, 6)
&(0, 1, 2, 3, 5, 7)
\\ \hline
(0, 1, 2, 4, 5, 6)
&(0, 1, 2, 4, 5, 7)
\\ \hline
(0, 1, 2, 4, 5, 6)
&(0, 1, 2, 5, 6, 7)
\\ \hline
(0, 1, 2, 4, 5, 6)
&(0, 1, 3, 4, 5, 6)
\\ \hline
(0, 1, 2, 4, 5, 6)
&(1, 2, 3, 4, 5, 6)
\\ \hline
(0, 1, 2, 4, 5, 6)
&(1, 3, 4, 5, 6, 7)
\\ \hline
(0, 1, 2, 4, 5, 7)
&(0, 1, 2, 4, 5, 6)
\\ \hline
(0, 1, 2, 4, 5, 7)
&(0, 1, 3, 4, 5, 7)
\\ \hline
(0, 1, 2, 5, 6, 7)
&(0, 1, 2, 4, 5, 6)
\\ \hline
(0, 1, 2, 5, 6, 7)
&(1, 2, 3, 5, 6, 7)
\\ \hline
(0, 1, 3, 4, 5, 6)
&(0, 1, 2, 4, 5, 6)
\\ \hline
(0, 1, 3, 4, 5, 6)
&(0, 1, 3, 4, 5, 7)
\\ \hline
(0, 1, 3, 4, 5, 7)
&(0, 1, 2, 3, 4, 6)
\\ \hline
(0, 1, 3, 4, 5, 7)
&(0, 1, 2, 4, 5, 7)
\\ \hline
(0, 1, 3, 4, 5, 7)
&(0, 1, 3, 4, 5, 6)
\\ \hline
(0, 1, 3, 4, 5, 7)
&(0, 1, 3, 4, 6, 7)
\\ \hline
(0, 1, 3, 4, 5, 7)
&(0, 2, 3, 4, 5, 7)
\\ \hline
(0, 1, 3, 4, 5, 7)
&(0, 2, 4, 5, 6, 7)
\\ \hline
(0, 1, 3, 4, 6, 7)
&(0, 1, 3, 4, 5, 7)
\\ \hline
(0, 1, 3, 4, 6, 7)
&(0, 2, 3, 4, 6, 7)
\\ \hline
(0, 2, 3, 4, 5, 7)
&(0, 1, 3, 4, 5, 7)
\\ \hline
(0, 2, 3, 4, 5, 7)
&(0, 2, 3, 4, 6, 7)
\\ \hline
(0, 2, 3, 4, 6, 7)
&(0, 1, 2, 3, 5, 7)
\\ \hline
(0, 2, 3, 4, 6, 7)
&(0, 1, 3, 4, 6, 7)
\\ \hline
(0, 2, 3, 4, 6, 7)
&(0, 2, 3, 4, 5, 7)
\\ \hline
(0, 2, 3, 4, 6, 7)
&(0, 2, 3, 5, 6, 7)
\\ \hline
(0, 2, 3, 4, 6, 7)
&(1, 2, 3, 4, 6, 7)
\\ \hline
(0, 2, 3, 4, 6, 7)
&(1, 3, 4, 5, 6, 7)
\\ \hline
(0, 2, 3, 5, 6, 7)
&(0, 2, 3, 4, 6, 7)
\\ \hline
(0, 2, 3, 5, 6, 7)
&(1, 2, 3, 5, 6, 7)
\\ \hline
(0, 2, 4, 5, 6, 7)
&(0, 1, 3, 4, 5, 7)
\\ \hline
(0, 2, 4, 5, 6, 7)
&(1, 2, 3, 5, 6, 7)
\\ \hline
(1, 2, 3, 4, 5, 6)
&(0, 1, 2, 4, 5, 6)
\\ \hline
(1, 2, 3, 4, 5, 6)
&(1, 2, 3, 5, 6, 7)
\\ \hline
(1, 2, 3, 4, 6, 7)
&(0, 2, 3, 4, 6, 7)
\\ \hline
(1, 2, 3, 4, 6, 7)
&(1, 2, 3, 5, 6, 7)
\\ \hline
(1, 2, 3, 5, 6, 7)
&(0, 1, 2, 3, 4, 6)
\\ \hline
(1, 2, 3, 5, 6, 7)
&(0, 1, 2, 5, 6, 7)
\\ \hline
(1, 2, 3, 5, 6, 7)
&(0, 2, 3, 5, 6, 7)
\\ \hline
(1, 2, 3, 5, 6, 7)
&(0, 2, 4, 5, 6, 7)
\\ \hline
(1, 2, 3, 5, 6, 7)
&(1, 2, 3, 4, 5, 6)
\\ \hline
(1, 2, 3, 5, 6, 7)
&(1, 2, 3, 4, 6, 7)
\\ \hline
(1, 3, 4, 5, 6, 7)
&(0, 1, 2, 4, 5, 6)
\\ \hline
(1, 3, 4, 5, 6, 7)
&(0, 2, 3, 4, 6, 7)
\\ \hline
\caption{Whirlwind $M_1$ (\ref{mat:whirlwind-m1}) singular submatrices}\label{tbl:m1-singular}
\end{longtable}
\end{footnotesize}


The fact that we have found singular submatrices for Whirlwind's $m_0$ and $m_1$ matrices shows they are not MDS in $GF(2^4)$. However, it is relevant to note that, in \cite{Whirlwind2010}, the authors mention the usage of the $GF(2^{16})$ field with \emph{decompositions} to $GF(2^4)$, as well as the usage of a \emph{normal basis} representation for the finite fields. It is possible that, despite not MDS in $GF(2^4)$ with $p(x) = x^4+x+1$ as the irreducible polynomial and a polynomial basis $\{1, x, x^2, x^3, x^4\}$, matrices (\ref{mat:whirlwind-m0}) and (\ref{mat:whirlwind-m1}) be MDS for the normal basis representation. We are still studying and evaluating this in our research. This report will be updated with our future findings. For this reason, we place Whirlwind's matrices in Table \ref{tbl:maybe-non-mds-list}, while we are still investigating.

\begin{footnotesize}
\begin{longtable}[c]{|l|l|l|l|l|l|l|l|l|l|}
\hline
\textbf{Year} & \textbf{Ord} & \textbf{Type} & \textbf{Inv} & \textbf{Use} & \textbf{Bib} & \textbf{$GF(2)[x]/(p(x))$} & \textbf{\#xor} & \textbf{\#xtime} & \textbf{Matrices} \\ \hline
\endfirsthead
\endhead

% Whirlwind
2010 & 8 & dyadic & no & \shortstack{Whirlwind} & \cite{Whirlwind2010} & $x^4+x+1$ & \shortstack{104\\128} & \shortstack{136\\136} & \shortstack{(\ref{mat:whirlwind-m0}) \\ (\ref{mat:whirlwind-m0-inv})} \\ \hline
2010 & 8 & dyadic & no & \shortstack{Whirlwind} & \cite{Whirlwind2010} & $x^4+x+1$ & \shortstack{128\\128} & \shortstack{128\\128} & \shortstack{(\ref{mat:whirlwind-m1}) \\ (\ref{mat:whirlwind-m1-inv})} \\ \hline
\caption{Non-MDS matrices under further investigation: parameters, usage and cost}\label{tbl:maybe-non-mds-list}
\end{longtable}
\end{footnotesize}

\begin{equation}\label{mat:whirlwind-m0}
\begin{bmatrix}
5_x & 4_x & a_x & 6_x & 2_x & d_x & 8_x & 3_x\\
4_x & 5_x & 6_x & a_x & d_x & 2_x & 3_x & 8_x\\
a_x & 6_x & 5_x & 4_x & 8_x & 3_x & 2_x & d_x\\
6_x & a_x & 4_x & 5_x & 3_x & 8_x & d_x & 2_x\\
2_x & d_x & 8_x & 3_x & 5_x & 4_x & a_x & 6_x\\
d_x & 2_x & 3_x & 8_x & 4_x & 5_x & 6_x & a_x\\
8_x & 3_x & 2_x & d_x & a_x & 6_x & 5_x & 4_x\\
3_x & 8_x & d_x & 2_x & 6_x & a_x & 4_x & 5_x
\end{bmatrix}
\end{equation}

\begin{equation}\label{mat:whirlwind-m0-inv}
\begin{bmatrix}
7_x & 3_x & e_x & b_x & 8_x & 1_x & 6_x & c_x\\
3_x & 7_x & b_x & e_x & 1_x & 8_x & c_x & 6_x\\
e_x & b_x & 7_x & 3_x & 6_x & c_x & 8_x & 1_x\\
b_x & e_x & 3_x & 7_x & c_x & 6_x & 1_x & 8_x\\
8_x & 1_x & 6_x & c_x & 7_x & 3_x & e_x & b_x\\
1_x & 8_x & c_x & 6_x & 3_x & 7_x & b_x & e_x\\
6_x & c_x & 8_x & 1_x & e_x & b_x & 7_x & 3_x\\
c_x & 6_x & 1_x & 8_x & b_x & e_x & 3_x & 7_x
\end{bmatrix}
\end{equation}

\begin{equation}\label{mat:whirlwind-m1}
\begin{bmatrix}
5_x & e_x & 4_x & 7_x & 1_x & 3_x & f_x & 8_x\\
e_x & 5_x & 7_x & 4_x & 3_x & 1_x & 8_x & f_x\\
4_x & 7_x & 5_x & e_x & f_x & 8_x & 1_x & 3_x\\
7_x & 4_x & e_x & 5_x & 8_x & f_x & 3_x & 1_x\\
1_x & 3_x & f_x & 8_x & 5_x & e_x & 4_x & 7_x\\
3_x & 1_x & 8_x & f_x & e_x & 5_x & 7_x & 4_x\\
f_x & 8_x & 1_x & 3_x & 4_x & 7_x & 5_x & e_x\\
8_x & f_x & 3_x & 1_x & 7_x & 4_x & e_x & 5_x
\end{bmatrix}
\end{equation}

\begin{equation}\label{mat:whirlwind-m1-inv}
\begin{bmatrix}
f_x & 1_x & c_x & 9_x & 3_x & 5_x & 2_x & b_x\\
1_x & f_x & 9_x & c_x & 5_x & 3_x & b_x & 2_x\\
c_x & 9_x & f_x & 1_x & 2_x & b_x & 3_x & 5_x\\
9_x & c_x & 1_x & f_x & b_x & 2_x & 5_x & 3_x\\
3_x & 5_x & 2_x & b_x & f_x & 1_x & c_x & 9_x\\
5_x & 3_x & b_x & 2_x & 1_x & f_x & 9_x & c_x\\
2_x & b_x & 3_x & 5_x & c_x & 9_x & f_x & 1_x\\
b_x & 2_x & 5_x & 3_x & 9_x & c_x & 1_x & f_x
\end{bmatrix}
\end{equation}
