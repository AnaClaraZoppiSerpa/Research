Matrix (\ref{mat:shark}) and its inverse (\ref{mat:shark-inv}) are used in the SHARK \cite{SHARK1996} cipher.

\begin{equation}\label{mat:shark}
\begin{bmatrix}
ce_x & 95_x & 57_x & 82_x & 8a_x & 19_x & b0_x & 01_x\\
e7_x & fe_x & 05_x & d2_x & 52_x & c1_x & 88_x & f1_x\\
b9_x & da_x & 4d_x & d1_x & 9e_x & 17_x & 83_x & 86_x\\
d0_x & 9d_x & 26_x & 2c_x & 5d_x & 9f_x & 6d_x & 75_x\\
52_x & a9_x & 07_x & 6c_x & b9_x & 8f_x & 70_x & 17_x\\
87_x & 28_x & 3a_x & 5a_x & f4_x & 33_x & 0b_x & 6c_x\\
74_x & 51_x & 15_x & cf_x & 09_x & a4_x & 62_x & 09_x\\
0b_x & 31_x & 7f_x & 86_x & be_x & 05_x & 83_x & 34_x
\end{bmatrix}
\end{equation}

\begin{equation}\label{mat:shark-inv}
\begin{bmatrix}
e7_x &30_x &90_x &85_x &d0_x &4b_x &91_x &41_x\\
53_x &95_x &9b_x &a5_x &96_x &bc_x &a1_x &68_x\\
02_x &45_x &f7_x &65_x &5c_x &1f_x &b6_x &52_x\\
a2_x &ca_x &22_x &94_x &44_x &63_x &2a_x &a2_x\\
fc_x &67_x &8e_x &10_x &29_x &75_x &85_x &71_x\\
24_x &45_x &a2_x &cf_x &2f_x &22_x &c1_x &0e_x\\
a1_x &f1_x &71_x &40_x &91_x &27_x &18_x &a5_x\\
56_x &f4_x &af_x &32_x &d2_x &a4_x &dc_x &71_x
\end{bmatrix}
\end{equation}

Matrix (\ref{mat:square}) and its inverse (\ref{mat:square-inv}) are used in the SQUARE \cite{SQUARE1997} cipher. They are right-circulant.

\begin{equation}\label{mat:square}
\begin{bmatrix}
02_x & 01_x & 01_x & 03_x\\
03_x & 02_x & 01_x & 01_x\\
01_x & 03_x & 02_x & 01_x\\
01_x & 01_x & 03_x & 02_x
\end{bmatrix}
\end{equation}

\begin{equation}\label{mat:square-inv}
\begin{bmatrix}
0e_x & 09_x & 0d_x & 0b_x\\
0b_x & 0e_x & 09_x & 0d_x\\
0d_x & 0b_x & 0e_x & 09_x\\
09_x & 0d_x & 0b_x & 0e_x
\end{bmatrix}
\end{equation}

Matrix (\ref{mat:tavares}) is involutory, and was obtained by \cite{Youssef1997} with a Cauchy construction.

\begin{equation}\label{mat:tavares}
\begin{bmatrix}
93_x & 13_x & 57_x & da_x & 58_x & 47_x & 0c_x & 1f_x\\
13_x & 93_x & da_x & 57_x & 47_x & 58_x & 1f_x & 0c_x\\
57_x & da_x & 93_x & 13_x & 0c_x & 1f_x & 58_x & 47_x\\
da_x & 57_x & 13_x & 93_x & 1f_x & 0c_x & 47_x & 58_x\\
58_x & 47_x & 0c_x & 1f_x & 93_x & 13_x & 57_x & da_x\\
47_x & 58_x & 1f_x & 0c_x & 13_x & 93_x & da_x & 57_x\\
0c_x & 1f_x & 58_x & 47_x & 57_x & da_x & 93_x & 13_x\\
1f_x & 0c_x & 47_x & 58_x & da_x & 57_x & 13_x & 93_x
\end{bmatrix}
\end{equation}

Matrix (\ref{mat:khazad}) is Hadamard and involutory. It is used in the KHAZAD \cite{KHAZAD2000} cipher.

\begin{equation}\label{mat:khazad}
\begin{bmatrix}
01_x & 03_x & 04_x & 05_x & 06_x & 08_x & 0b_x & 07_x\\
03_x & 01_x & 05_x & 04_x & 08_x & 06_x & 07_x & 0b_x\\
04_x & 05_x & 01_x & 03_x & 0b_x & 07_x & 06_x & 08_x\\
05_x & 04_x & 03_x & 01_x & 07_x & 0b_x & 08_x & 06_x\\
06_x & 08_x & 0b_x & 07_x & 01_x & 03_x & 04_x & 05_x\\
08_x & 06_x & 07_x & 0b_x & 03_x & 01_x & 05_x & 04_x\\
0b_x & 07_x & 06_x & 08_x & 04_x & 05_x & 01_x & 03_x\\
07_x & 0b_x & 08_x & 06_x & 05_x & 04_x & 03_x & 01_x
\end{bmatrix}
\end{equation}

Matrix (\ref{mat:anubis}) is Hadamard and involutory. It is used in the ANUBIS \cite{ANUBIS2000} cipher.

\begin{equation}\label{mat:anubis}
\begin{bmatrix}
01_x & 02_x & 04_x & 06_x\\
02_x & 01_x & 06_x & 04_x\\
04_x & 06_x & 01_x & 02_x\\
06_x & 04_x & 02_x & 01_x
\end{bmatrix}
\end{equation}

Still regarding the ANUBIS cipher, while (\ref{mat:anubis}) is used as its linear transformation layer, (\ref{mat:anubis-ke}) is used in the key extraction. It is a Vandermonde construction. When $N = 4$, it is an MDS matrix (see Theorem \ref{teo:mds}).

\begin{equation}\label{mat:anubis-ke}
\begin{bmatrix}
01_x & 01_x & 01_x & ... & 01_x\\
01_x & 02_x & 02_x^2 & ... & 02_x^{N-1}\\
01_x & 06_x & 06_x^2 & ... & 06_x^{N-1}\\
01_x & 08_x & 08_x^2 & ... & 08_x^{N-1}
\end{bmatrix}
=
\begin{bmatrix}
01_x & 01_x & 01_x & 01_x\\
01_x & 02_x & 04_x & 08_x\\
01_x & 06_x & 14_x & 78_x\\
01_x & 08_x & 40_x & 3a_x
\end{bmatrix} \text{for } N=4
\end{equation}

Matrix (\ref{mat:rijndael}) and its inverse (\ref{mat:rijndael-inv}) are used in the Rijndael \cite{DesignOfRijndael2002} cipher, which was selected to become AES. They are right-circulant. We show the hexadecimal notation and the corresponding polynomials to emphasize that, albeit stored as integers in cryptographic software implementation, all matrix elements are actually polynomials in a Finite Field. This applies not only to the Rijndael cipher's matrices but to all matrices listed in this work.

\begin{equation}\label{mat:rijndael}
\begin{bmatrix}
02_x & 03_x & 01_x & 01_x\\
01_x & 02_x & 03_x & 01_x\\
01_x & 01_x & 02_x & 03_x\\
03_x & 01_x & 01_x & 02_x
\end{bmatrix}
=
\begin{bmatrix}
x & x+1 & 1 & 1\\
1 & x & x+1 & 1\\
1 & 1 & x & x+1\\
x+1 & 1 & 1 & x
\end{bmatrix}
\end{equation}

\begin{equation}\label{mat:rijndael-inv}
\begin{bmatrix}
0e_x & 0b_x & 0d_x & 09_x\\
09_x & 0e_x & 0b_x & 0d_x\\
0d_x & 09_x & 0e_x & 0b_x\\
0b_x & 0d_x & 09_x & 0e_x
\end{bmatrix}
=
\begin{bmatrix}
x^3+x^2+x & x^3+x+1 & x^3+x^2+1 & x^3+1\\
x^3+1 & x^3+x^2+x & x^3+x+1 & x^3+x^2+1\\
x^3+x^2+1 & x^3+1 & x^3+x^2+x & x^3+x+1\\
x^3+x+1 & x^3+x^2+1 & x^3+1 & x^3+x^2+x
\end{bmatrix}
\end{equation}

Furthermore, it is interesting to note that Rijndael's matrix (\ref{mat:rijndael}) is the transpose of SQUARE's matrix (\ref{mat:square}) and this also happens to the inverses (matrix (\ref{mat:square-inv}) is the transpose of (\ref{mat:rijndael-inv})).

Matrices (\ref{mat:bksq}) and its inverse (\ref{mat:bksq-inv}) are used in the BKSQ \cite{BKSQ1998} cipher. They are right-circulant.

\begin{equation}\label{mat:bksq}
\begin{bmatrix}
03_x & 02_x & 02_x\\
02_x & 03_x & 02_x\\
02_x & 02_x & 03_x
\end{bmatrix}
\end{equation}

\begin{equation}\label{mat:bksq-inv}
\begin{bmatrix}
ac_x & ad_x & ad_x\\
ad_x & ac_x & ad_x\\
ad_x & ad_x & ac_x
\end{bmatrix}
\end{equation}

The Hierocrypt-3 cipher makes use of two MDS matrices, one for lower level diffusion and another for higher level diffusion on the cipher, which follows a nested Substitution Permutation Network design (for more detail the reader may refer to \cite{Hierocrypt2000}). Matrix (\ref{mat:hierocrypt-3-lower}) and its inverse (\ref{mat:hierocrypt-3-lower-inv}) are used for lower level diffusion. (\ref{mat:hierocrypt-3-higher}) and (\ref{mat:hierocrypt-3-higher-inv}) (the inverse) are used for higher level diffusion. It is worth noting that, for lower level diffusion, the finite field is $GF(2^8)$, whilst, for higher level diffusion, the authors choose $GF(2^4)$.

\begin{equation}\label{mat:hierocrypt-3-lower}
\begin{bmatrix}
c4_x & 65_x & c8_x & 8b_x\\
8b_x & c4_x & 65_x & c8_x\\
c8_x & 8b_x & c4_x & 65_x\\
65_x & c8_x & 8b_x & c4_x
\end{bmatrix}
\end{equation}

\begin{equation}\label{mat:hierocrypt-3-lower-inv}
\begin{bmatrix}
82_x & c4_x & 34_x & f6_x\\
f6_x & 82_x & c4_x & 34_x\\
34_x & f6_x & 82_x & c4_x\\
c4_x & 34_x & f6_x & 82_x
\end{bmatrix}
\end{equation}

\begin{equation}\label{mat:hierocrypt-3-higher}
\begin{bmatrix}
5_x & 5_x & a_x & e_x\\
e_x & 5_x & 5_x & a_x\\
a_x & e_x & 5_x & 5_x\\
5_x & a_x & e_x & 5_x
\end{bmatrix}
\end{equation}

\begin{equation}\label{mat:hierocrypt-3-higher-inv}
\begin{bmatrix}
b_x & e_x & e_x & 6_x\\
6_x & b_x & e_x & e_x\\
e_x & 6_x & b_x & e_x\\
e_x & e_x & 6_x & b_x
\end{bmatrix}
\end{equation}

The Hierocrypt-L1 cipher too uses matrix (\ref{mat:hierocrypt-3-lower}) and the inverse (\ref{mat:hierocrypt-3-lower-inv}) in its lower diffusion layer. However, for the higher layer, (\ref{mat:hierocrypt-l1-higher}) and (\ref{mat:hierocrypt-l1-higher-inv}) (inverse) are used. Analogously to Hierocrypt-3, the higher layer uses $GF(2^4)$.

\begin{equation}\label{mat:hierocrypt-l1-higher}
\begin{bmatrix}
5_x & 7_x\\
a_x & b_x
\end{bmatrix}
\end{equation}

\begin{equation}\label{mat:hierocrypt-l1-higher-inv}
\begin{bmatrix}
c_x & a_x\\
5_x & b_x
\end{bmatrix}
\end{equation}

Matrices (\ref{mat:fox-mu4}) (inverse: (\ref{mat:fox-mu4-inv})) and (\ref{mat:fox-mu8}) (inverse: (\ref{mat:fox-mu8-inv})) are used in the FOX block cipher family, with $z = x^7+x^6+x^5+x^4+x^3+x^2+1, a = x+1, b = x^7+x, c = x, d = x^2, e = x^7+x^6+x^5+x^4+x^3+x^2$ and $f = x^6+x^5+x^4+x^3+x^2+x$.

\begin{equation}\label{mat:fox-mu4}
\begin{bmatrix}
1 & 1 & 1 & x\\
1 & z & x & 1\\
z & x & 1 & 1\\
x & 1 & z & 1
\end{bmatrix}
=
\begin{bmatrix}
01_x & 01_x & 01_x & 02_x\\
01_x & fd_x & 02_x & 01_x\\
fd_x & 02_x & 01_x & 01_x\\
02_x & 01_x & fd_x & 01_x
\end{bmatrix}
\end{equation}

\begin{equation}\label{mat:fox-mu8}
\begin{bmatrix}
1 & 1 & 1 & 1 & 1 & 1 & 1 & a\\
1 & a & b & c & d & e & f & 1\\
a & b & c & d & e & f & 1 & 1\\
b & c & d & e & f & 1 & a & 1\\
c & d & e & f & 1 & a & b & 1\\
d & e & f & 1 & a & b & c & 1\\
e & f & 1 & a & b & c & d & 1\\
f & 1 & a & b & c & d & e & 1
\end{bmatrix}
=
\begin{bmatrix}
01_x & 01_x & 01_x & 01_x & 01_x & 01_x & 01_x & 03_x\\
01_x & 03_x & 82_x & 02_x & 04_x & fc_x & 7e_x & 01_x\\
03_x & 82_x & 02_x & 04_x & fc_x & 7e_x & 01_x & 01_x\\
82_x & 02_x & 04_x & fc_x & 7e_x & 01_x & 03_x & 01_x\\
02_x & 04_x & fc_x & 7e_x & 01_x & 03_x & 82_x & 01_x\\
04_x & fc_x & 7e_x & 01_x & 03_x & 82_x & 02_x & 01_x\\
fc_x & 7e_x & 01_x & 03_x & 82_x & 02_x & 04_x & 01_x\\
7e_x & 01_x & 03_x & 82_x & 02_x & 04_x & fc_x & 01_x
\end{bmatrix}
\end{equation}

\begin{equation}\label{mat:fox-mu4-inv}
\begin{bmatrix}
7e_x & e1_x & ad_x & b0_x\\
7e_x & ad_x & b0_x & e1_x\\
7e_x & b0_x & e1_x & ad_x\\
c3_x & 7e_x & 7e_x & 7e_x
\end{bmatrix}
\end{equation}

\begin{equation}\label{mat:fox-mu8-inv}
\begin{bmatrix}
c6_x & fe_x & 3a_x & 73_x & 6d_x & 0c_x & d2_x & b7_x\\
c6_x & 3a_x & 73_x & 6d_x & 0c_x & d2_x & b7_x & fe_x\\
c6_x & 73_x & 6d_x & 0c_x & d2_x & b7_x & fe_x & 3a_x\\
c6_x & 6d_x & 0c_x & d2_x & b7_x & fe_x & 3a_x & 73_x\\
c6_x & 0c_x & d2_x & b7_x & fe_x & 3a_x & 73_x & 6d_x\\
c6_x & d2_x & b7_x & fe_x & 3a_x & 73_x & 6d_x & 0c_x\\
c6_x & b7_x & fe_x & 3a_x & 73_x & 6d_x & 0c_x & d2_x\\
ea_x & c6_x & c6_x & c6_x & c6_x & c6_x & c6_x & c6_x
\end{bmatrix}
\end{equation}

Matrix (\ref{mat:curupira}) is used in Curupira's diffusion layer. It is involutory.

\begin{equation}\label{mat:curupira}
\begin{bmatrix}
03_x & 02_x & 02_x\\
04_x & 05_x & 04_x\\
06_x & 06_x & 07_x
\end{bmatrix}
\end{equation}

Matrix (\ref{mat:curupira-ke}) is used in Curupira's key scheduling process, with $c(x) = x^4 + x^3 + x^2$.

\begin{equation}\label{mat:curupira-ke}
\begin{bmatrix}
1+c(x) & c(x) & c(x)\\
c(x) & 1+c(x) & c(x)\\
c(x) & c(x) & 1+c(x)
\end{bmatrix}
\end{equation}

Matrix (\ref{mat:grostl}) is used in the Gr{\o}stl hash function. It is right-circulant.

\begin{equation}\label{mat:grostl}
\begin{bmatrix}
02_x & 02_x & 03_x & 04_x & 05_x & 03_x & 05_x & 07_x\\
07_x & 02_x & 02_x & 03_x & 04_x & 05_x & 03_x & 05_x\\
05_x & 07_x & 02_x & 02_x & 03_x & 04_x & 05_x & 03_x\\
03_x & 05_x & 07_x & 02_x & 02_x & 03_x & 04_x & 05_x\\
05_x & 03_x & 05_x & 07_x & 02_x & 02_x & 03_x & 04_x\\
04_x & 05_x & 03_x & 05_x & 07_x & 02_x & 02_x & 03_x\\
03_x & 04_x & 05_x & 03_x & 05_x & 07_x & 02_x & 02_x\\
02_x & 03_x & 04_x & 05_x & 03_x & 05_x & 07_x & 02_x
\end{bmatrix}
\end{equation}

Matrices (\ref{mat:whirlwind-m0}) and (\ref{mat:whirlwind-m1}) are used in the Whirlwind hash function.

\begin{equation}\label{mat:whirlwind-m0}
\begin{bmatrix}
5_x & 4_x & a_x & 6_x & 2_x & d_x & 8_x & 3_x\\
4_x & 5_x & 6_x & a_x & d_x & 2_x & 3_x & 8_x\\
a_x & 6_x & 5_x & 4_x & 8_x & 3_x & 2_x & d_x\\
6_x & a_x & 4_x & 5_x & 3_x & 8_x & d_x & 2_x\\
2_x & d_x & 8_x & 3_x & 5_x & 4_x & a_x & 6_x\\
d_x & 2_x & 3_x & 8_x & 4_x & 5_x & 6_x & a_x\\
8_x & 3_x & 2_x & d_x & a_x & 6_x & 5_x & 4_x\\
3_x & 8_x & d_x & 2_x & 6_x & a_x & 4_x & 5_x
\end{bmatrix}
\end{equation}

\begin{equation}\label{mat:whirlwind-m1}
\begin{bmatrix}
5_x & e_x & 4_x & 7_x & 1_x & 3_x & f_x & 8_x\\
e_x & 5_x & 7_x & 4_x & 3_x & 1_x & 8_x & f_x\\
4_x & 7_x & 5_x & e_x & f_x & 8_x & 1_x & 3_x\\
7_x & 4_x & e_x & 5_x & 8_x & f_x & 3_x & 1_x\\
1_x & 3_x & f_x & 8_x & 5_x & e_x & 4_x & 7_x\\
3_x & 1_x & 8_x & f_x & e_x & 5_x & 7_x & 4_x\\
f_x & 8_x & 1_x & 3_x & 4_x & 7_x & 5_x & e_x\\
8_x & f_x & 3_x & 1_x & 7_x & 4_x & e_x & 5_x
\end{bmatrix}
\end{equation}
