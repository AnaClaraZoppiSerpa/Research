\documentclass{report}
\usepackage[utf8]{inputenc}
\usepackage{amsmath}
\usepackage{amsfonts}
\usepackage{hyperref}
\usepackage{tcolorbox}
\usepackage{breqn}
\usepackage{adjustbox}
\usepackage{changepage}
\usepackage{rotating}
\usepackage{algorithm}
\usepackage{algpseudocode}
\usepackage{ntheorem}
\usepackage[table,xcdraw]{xcolor}
\usepackage{longtable}
\usepackage{listings}

% Definition
\newtheorem{definition}{Definition}{\bfseries}{\itshape}
\newtheorem*{definition*}{Definition}{\bfseries}{\itshape}

% Theorem
\newtheorem{theorem}{Theorem}{\bfseries}{\itshape}

% Concept
\newtheorem*{concept}{}{\bfseries}{\itshape}

\title{titulo}
\author{Ana Clara Zoppi Serpa\\ Prof. Dr. Ricardo Dahab \\ Dr. Jorge Nakahara Jr.}
\date{\today}

\begin{document}

\maketitle

\tableofcontents

\chapter{titulo}

[chapter intro]

We assume the reader is familiar with:
\begin{itemize}
    \item a
    \item b
    \item c
\end{itemize}

\section{Notation}
\begin{itemize}
    \item a
    \item b
    \item c
\end{itemize}

\section{Acronyms}
\begin{itemize}
    \item a
    \item b
    \item c
\end{itemize}

\section{[título apropriado aqui]}

[escrever o report de fato]

\section{Conclusions}

\bibliographystyle{plain}
\bibliography{refs}

\end{document}
